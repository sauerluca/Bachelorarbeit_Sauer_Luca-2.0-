\chapter*{Abstract}
\begingroup
\begin{table}[h!]
\setlength\tabcolsep{0pt}
\begin{tabular}{p{3.7cm}p{11.7cm}}
Titel & \DerTitelDerArbeit \\
Verfasser/in: & \DerAutorDerArbeit \\
Kurs: & \DieKursbezeichnung \\
Ausbildungsstätte: & \DerNameDerFirma\\
\end{tabular}
\end{table}
\endgroup

Zur besseren Lesbarkeit wird in dieser Studie durchgehend das generischen Maskulinum genutzt
(z. B. Mitarbeiter). Dies gibt keinerlei Auskunft über das Geschlecht und stellt keine implizierte
Geschlechterdiskriminierung des weiblichen Geschlechts dar. Frauen und Männer mögen sich
gleichermaßen angesprochen fühlen.