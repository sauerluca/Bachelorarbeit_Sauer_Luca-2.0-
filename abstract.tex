\chapter*{Kurzfassung}
 \thispagestyle{empty}
 \begingroup
\begin{table}[h!]
\setlength\tabcolsep{0pt}
\begin{tabular}{p{3.7cm}p{11.7cm}}
Titel & \DerTitelDerArbeit \\
Verfasser/in: & \DerAutorDerArbeit \\
Kurs: & \DieKursbezeichnung \\
Ausbildungsstätte: & \DerNameDerFirma\\
\end{tabular}
\end{table}
\endgroup
Die Fertigungsdurchführung der diskreteten Fertigung muss sich heute schnell an veränderte Situationen anpassen können, um im Wettbewerb weiter zu bestehen. Die Verfügbarkeit von aktuellen, situativen Informationen bei Entscheidungen im laufenden Betrieb ist ein wesentlicher Faktor, um Veränderungen schneller ausführen zu können und Verschwendungen infolge fehlender Informationen zu reduzieren. Zur Unterstützung kommt heute eine Vielzahl aufgabenspezifischer IT-Systeme zum Einsatz, auf welche sich die Informationen verteilen. Diese verursachen während der Betriebsdatenerfassung nicht wertschöpfende Zeiten. Hieraus ergibt sich die Motivation neue, übergreifende Ansätze auf Basis von Ereignisverarbeitung zu entwickeln, welche innerhalb einer diskreten Fertigung das operative Personal in der Erfassung sowie Beschaffung der Informationen unmittelbar und effizient unterstützen.

Den Schwerpunkt bildet das Gesamtkonzept des fertigungsnahen Anwendungssystems, das aus dem Geschäftsprozessmodell und der Ereignisverarbeitungsebene besteht. 
Das Geschäftsprozessmodell beschreibt und vernetzt enthaltene Ereignisse, Aktivitäten und Zusammenhänge der Fertigungsdurchführung in der diskreten Fertigung. Die Ereignisverarbeitungsebene ist als Webservice aufgebaut. Der Webservice verarbeitet eingehende Ereignisse und unterstützt durch unmittelbare Reaktionen auf diese Ereignisse die automatische und manuelle Betriebsdatenerfassung.
Im Rahmen dieser Arbeit wird ein Ansatz zur ereignisgesteuerten Betriebsdatenerfassung von Werkern und Fertigungssteuerern vorgestellt, der es ermöglicht, aus der auftretenden Situation heraus, auf aktuelle notwendige Informationen und die Zusammenhänge in der Fertigungsdurchführung in der diskreten Fertigung zuzugreifen.

Zur Erprobung des Geschäftsprozesses wurde eine softwaretechnische Implementierung in Form eines Prototyps durchgeführt und anhand zuvor definierter Anforderungen evaluiert und validiert.