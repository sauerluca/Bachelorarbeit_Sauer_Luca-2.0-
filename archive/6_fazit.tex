\chapter{Fazit und Ausblick}
Betrachtet man die aktuelle Situation des stark wachsenden Cloud-Computing-Marktes und der Digitalisierung werden Cloud-Anwendungen als Erweiterungs- und Integrationsmöglichkeit eine gewichtige Rolle bei der Verbreitung von betriebswirtschaftlicher Cloud-Standardsoftware spielen. Verwandt mit der Standardisierung im Allgemeinen ist das Konzept einer digitalen Plattform, das eine Standardisierung auf Unternehmensebene darstellt. Im Kern haben digitale Plattformen zum Ziel, auf Basis wesentlicher Gemeinsamkeiten unterschiedliche individuelle Produkte zu entwickeln. Ein offenes \ac{PaaS}-Angebot für Schnittstellen und Erweiterungsmöglichkeiten von und zu \ac{SaaS}-Angeboten zielt darauf ab, Vertrauen zu etablieren und die Akzeptanz von Cloud-Computing zu erhöhen. Kunden und Partner stimmen einer solchen Plattformstrategie jedoch nur zu, wenn eine nahtlose Integration gewährleistet wird. Eine nahtlose Integration und eine kompatible Erweiterbarkeit, die von der Standardsoftware entkoppelt ist, müssen gleichzeitig erreicht werden, was sich wie ein Widerspruch anhört, aber dies ist tatsächlich die Voraussetzung. \autocite{Herzig.2018}

Die Idee, dieser Problemstellung entgegenzutreten, war es in der vorliegenden Arbeit, durch die Erarbeitung und Evaluation einer Konzeption zur Erweiterung von SAP S/4HANA Cloud via SAP Cloud Platform, das Potential einer digitalen Plattform im Cloud-Computing-Umfeld darzustellen und anhand eines konkreten Anwendungsszenarios zu demonstrieren. Aufgrund der steigenden Notwendigkeit von Benutzerunterstützung im Rahmen von Cloud-Computing und der Digitalisierung galt es ein Konzept zur Erweiterung des ausgewählten Geschäftsszenarios von SAP S/4HANA Cloud unter Berücksichtigung einer optimalen Benutzbarkeit zu entwerfen.

Der Konzeptentwurf folgt der grundlegenden Methodik einer Anforderungsanalyse nach \citeauthor{Balzert.2011} und beinhaltet einige \ac{UCD}-Techniken. Mit Beginn des User Researchs wurde eine Persona erstellt, Use Cases definiert und die grundlegende Handlung in einem Storyboard zusammengefasst. Aus dem Storyboard resultierte das fundamentale Verständnis der Benutzerbedürfnisse, welches als Voraussetzung zur Analyse der Anforderungen eingesetzt wurde. Mit der Auswertung der Ergebnisse konnte eine Vielzahl an Verbesserungspotentialen festgestellt werden, wie der Prozess und das Design zu optimieren und in die Architektur zu integrieren sind.

Der Anforderungsanalyse des Ist-Zustandes schloss sich der zweite praktische Teil der Arbeit an. In diesem wurden für alle Anforderungen Lösungskonzepte erstellt. Das resultierende Konzept ist in seinem Ausmaß weitestgehend eindeutig und muss lediglich im Kontext spezifischer Aspekte evaluiert werden. Insgesamt lässt sich das Ergebnis als Folge der ersten Konzeption durchaus als erfolgreich beurteilen. Das wesentliche Grundgerüst wurde gelegt und zentrale Fragestellungen, wie zum Beispiel zur Möglichkeit der Erweiterung von \ac{SaaS}-Angeboten mittels \ac{PaaS}-Angeboten, der vom Standard entkoppelten Stammdatenpflege oder auch der Prozessoptimierung durch eine hohe Benutzbarkeit beantwortet. Lediglich einige wenige Bereiche, wie zum Beispiel die Persistenz der Daten in der Cloud-Anwendung, sind in einer zweiten Konzeption erneut zu validieren.

Diese erarbeitete Konzeption wurde schließlich realisiert. Prinzipiell lässt sich sagen, dass der Prototyp funktionsfähig ist und alle Anforderungen erfüllt. Um die Eingangsfrage zu beantworten: Ja, eine digitale Plattform im Umfeld von Cloud-Comuting ist durchaus in der Lage den Erweiterungs- und Individualisierungsbedürfnissen eines \ac{SaaS}-Benutzers Folge zu leisten. Vor allem aufgrund der Erweiterbarkeit mit grundlegenden Funktionen, wie einer sicheren Authentifizierung, der Integration über standardisierte Schnittstellen und benutzerfreundlicheren Eigenschaften bewegt sich das Konzept in die richtige Richtung. Dennoch gibt es noch einige Funktionalitäten, die in dieser Arbeit nicht behandelt werden konnten.

Wie vorher schon kurz angedeutet, gibt es weitaus mehr Alternativen, als den Vorschlag, der in dieser Arbeit wahrgenommen wurde. Da noch nicht viele Lösungen für diese Problemstellung existieren, könnte es sicherlich gewinnbringend sein, sich der anderen Alternativen, wie der In-App-Erweiterbarkeit, zumindest sporadisch anzunehmen. Auch fehlt eine gründliche Evaluierung mit einem Endbenutzer. Daher wird der aus dem Konzept entstandene Prototyp durch die Veröffentlichung auf dem \textit{SAP Extensibility Explorer}, die mit dem S/4HANA-Cloud-Release 1902 erfolgt, inklusive einer ausführlichen Dokumentation sowie dem Quellcode, für die Öffentlichkeit verfügbar gemacht. Die vorliegende methodische Vorgehensweise sowie das Konzept der Cloud-Anwendung auf der SAP Cloud Platform können in diesem Kontext als Vorlage zur konkreten Umsetzung eines realen Geschäftsszenarios im SAP S/4HANA-Cloud-Umfeld als Cloud-Anwendung dienen. 

Zusammenfassend kann mit dieser Arbeit ein Einblick in dieses, noch eher unerschlossene, Gebiet gewährt und die strategische Relevanz von Schnittstellen und Plattformen im Cloud-Computing-Umfeld demonstriert werden. Mit einer prototypischen Umsetzung ist ein erster Grundstein gelegt, auf deren Basis Optimierungen und Weiterentwicklungen auf dem Weg zu einer entkoppelten und dennoch nahtlosen Integration von Erweiterungen in \ac{SaaS}-Angebote durchgeführt werden können. Weiterhin wurden alternative Geschäftsszenarien und Technologien angesprochen, die in einer weiterführenden Arbeit als Diskussionsgrundlage genutzt werden können. So haben wir beispielsweise eine auf Microservice basierende Architektur als eine wichtige Säule der Entwicklung von Cloud-Anwendungen eingeführt. Dennoch hört die Relevanz der Arbeit hier nicht auf. Tatsächlich gibt es zahlreiche Ideen, um digitale Plattformen noch leistungsfähiger zu machen, die Integration zwischen Cloud-Anwendungen noch einfacher und die Anwendungsentwicklung noch effizienter zu gestalten.