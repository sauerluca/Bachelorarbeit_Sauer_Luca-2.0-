\chapter{Einleitung}

Die Digitale Transformation führt dazu, dass betriebswirtschaftliche Prozesse, industrielle Produktion und digitale Produkte und Dienstleistungen mehr und mehr zusammenwachsen. Um Angebot und Nachfrage verschiedener Akteure mithilfe digitaler Technologien zu verbinden und ihre Interaktionen zu vereinfachen, sind Erweiterungs- und Integrationsmöglichkeiten notwendig. Digitale Plattformen sind solche zentralen Schnittstellen. Sie verändern bestehende Kunden-Anbieter-Beziehungen und erschließen neue Geschäftsmodelle.
\autocite{Densborn.2018}

% \paragraph{Ziel dieser Arbeit}
\epigraph{
\enquote{Die Digitale Transformation führt zu starken Veränderungen in der ERP-Welt. Der Trend geht klar weg von den zentralen monolithischen Systemen hin zu vernetzten digitalen Plattformen.}
% \autocite[15]{Schneider.2018}
\autocite{bitkom.2018}
}{}

Diese Entwicklung erfährt durch Cloud Computing und in Folge von digitalen Plattformen eine ungeahnte Dynamik. In dem Maße nämlich, in dem Funktionen als Services oder Anwendungen über die Cloud angeboten werden, erhalten Benutzer die Freiheit, zusätzliche Anwendungen und Funktionen über Schnittstellen zur bestehenden Cloud-Anwendung zu ergänzen. Die zugrundeliegende Technologie ist nicht neu, doch die Konsequenzen für die Geschäftsmodelle von Anbietern und Benutzern von \acs{IT}-Systemen können kaum überschätzt werden.
\autocite{bitkom.2018}

In diesem Rahmen liegt die Frage nach der betriebswirtschaftlichen Standardsoftware der Zukunft nahe:
\enquote{Ist die Zukunft von betriebswirtschaftlicher Standardsoftware ein großes monolithisches System oder ist es ein schlanker Kern mit einer Reihe von betriebswirtschaftlichen Cloud-Anwendungen, die es umgeben?}
Diese Frage geht davon aus, dass ein Kompromiss erforderlich ist, um von einem Muster zum anderen zu gelangen. In einem monolithischen System ist die einfache Integration standardmäßig von Vorteil, sie ist jedoch nur eingeschränkt flexibel. Andererseits lässt sich das Best-of-Breed nicht immer leicht integrieren. Doch das Bild wandelt sich: Standardschnittstellen und offene Technologien erlauben es heute, Software und Systeme über digitale Plattformen nahtlos miteinander zu vernetzen.
\autocite{Henneberger.2010}

Ziel dieser Arbeit ist daher der Entwurf und die Evaluierung eines Konzepts zur Erweiterung einer Cloud-Anwendung unter Berücksichtigung einer optimalen Benutzbarkeit. Durch die Konzeption sollen relevante Aspekte der Erweiterbarkeit von Cloud-Anwendungen aufgezeigt werden, um die Lücke zwischen betriebswirtschaftlicher Cloud-Standardsoftware und der Integration durch digitale Plattformen zu schließen, sodass ein Kunde im Betrieb einer Cloud-Anwendung bestmöglich dabei unterstützt wird seine individuellen Anforderungen umzusetzen. Aufgrund der zeitlichen Begrenzung dieser Arbeit, soll sich bei der Konzeption auf für die Abteilung wesentliche Technologien, wie SAP S/4HANA Cloud und SAP Cloud Platform, beschränkt werden. Damit soll die Arbeit eine Antwort auf die Frage liefern, inwiefern ein Benutzer die Möglichkeit hat betriebswirtschaftliche Cloud-Standardsoftware durch Erweiterungen über digitale Plattformen an seine Bedürfnisse anzupassen.

% \paragraph{Aufbau und Inhalt}
Die vorliegende Arbeit ist in sechs Kapitel untergliedert. Auf dieses einleitende Kapitel folgt das zweite Kapitel, welches sich den Grundlagen widmet und dem Leser Einblicke in die wesentlichen theoretischen Fundamente zu den Themen Stammdaten, Cloud-Computing und den Techniken von  Grundlegende User Centered Design, gibt. Spezielle Produkte, wie SAP S/4HANA Cloud und die SAP Cloud Platform werden hier ebenfalls vorgestellt.

Anschließend folgt das erste praxisorientierte Kapitel, in welchem die Anforderungsanalyse dokumentiert ist. Zunächst wird der IST-Zustand beleuchtet und ein Gesamtüberblick über die Ausgangslage, die Architektur, den gegebenen Prozess und das Design des zuvor ausgewählten Geschäftsszenarios gegeben. Entsprechend werden dann im zweiten Schritt Schwachstellen und Optimierungsbedarfe ausgearbeitet, um erste richtungweisende Anhaltspunkte zu erhalten. In diesem Kontext wird auch der Lieferumfang festgelegt. Des Weiteren werden die wesentlichen funktionalen und nicht-funktionalen Anforderungen beschrieben.

Darauf folgt die Konzeption eines Lösungsvorschlages für die ermittelten Schwachstellen. Dabei wird auf alle definierten Anforderungen umfassend eingegangen und mittels geeigneter Alternativen ein Lösungsvorschlag dargestellt. Zu diesem Kapitel gehören auch Konzepte für die in Kapitel 3 beschriebenen Schwachstellen, wie ein Authentifizierungskonzept, ein Architekturkonzept und ein Ansatz für das Stammdatenmodell. Die zu verwendeten Frameworks werden ebenso kurz erläutert.

Es schließt sich das Evaluationskapitel an, welches hauptsächlich die Ergebnisse der vorangegangenen Entwicklung präsentiert. Mithilfe von einem Abgleich mit dem IST-Zustand, sowie aller Anforderungen, wird das Konzept geprüft und evaluiert. Mit dem Fazit, das eine Zusammenfassung und einen Ausblick in die Zukunft wagt, kommt die Arbeit zum Ende.
% \paragraph{Einschränkungen}



