% **************************************************************************************
% * Prozessexperten                                                               
% **************************************************************************************
\tocless\section{Interview mit einem Prozessexperten}\label{ah:interviewCON}

\spaceparagraph{Experte für das Experteninterview}
\textit{Herr Otto Normalverbraucher \footnote{Der Name \textit{Otto Normalverbraucher} repräsentiert den Namen des Prozessexperten.}} wird für das Experteninterview befragt.
Er ist einer der Experten des Beratungsbereiches \enquote{Manufacturing Industries} der \textit{ERP SE \footnote{Die Firma \textit{ERP SE} repräsentiert den Arbeitgeber des Prozessexperten.}}, welcher Betrieben aus der Automobilindustrie, im Maschinen- und Anlagenbau sowie der Hightech Industrie, bei der Gestaltung branchenspezifischer Prozessabläufe und der Umsetzung mit SAP-Software unterstützt.
In seiner Rolle als \enquote{Senior Business Process Consultant} bewertet er Kundenanforderungen, stimmt sie auf die Geschäftsprozesse ab, konzeptioniert die Abbildung mittels SAP Lösungen in einer Gesamtlösungsarchitektur und sorgt für die Implementierung aus Sicht der Lösungsarchitektur.
Basierend auf den ihm vorliegenden Systemlandschaftsdaten erarbeitet er detaillierte Konzepte durchgängiger, branchenspezifischer Geschäftsprozesse und identifiziert dabei Faktoren, die einen negativen Einfluss auf den Geschäftsprozess haben.
Im Dialog mit den Kunden hilft er diesen, die kritischen Faktoren nachhaltig zu beseitigen. 
 
Diese Tätigkeit übt \textit{Herr Otto Normalverbraucher} seit dem Jahr 1999 (seit dem Jahr 2004 bei der \textit{ERP SE}) aus und besitzt zum Zeitpunkt der Erstellung dieser Bachelorarbeit damit eine Berufserfahrung von 20 Jahren.
Aufgrund dieser besitzt er einen großen Erfahrungsschatz hinsichtlich der Merkmale von etablierten Arbeitsprozessen und Hilfsmitteln zur Fertigungsdurchführung im SAP-Umfeld, welche anhand dieses Experteninterviews greifbar gemacht werden soll.

\spaceparagraph{Kernaussagen des Experteninterviews}

\begin{definitionForm}[KA-P-1]
Die Fertigungsdurchführung ist, beispielhaft an diversen von ihm betreuten Industiebetrieben. durch \textbf{geringe Flexibilität und Geschwindigkeit} gekennzeichnet. Der Auslöser einer Fertigungsdurchführung ist die persönliche Abholung eines Fertigungsauftrags vom Werker beim Fertigungssteuerer. Die Übergabe erfolgt in ausgedruckter Form. Dabei entstehen unnötige Laufwege und Wartezeiten in der Produktion. Bei kurzfristigen Änderungen aufgrund von aufgetretenen Ereignissen, wie einer Veränderung der Mengen im Fertigungsauftrag, besteht keine Möglichkeit dies ohne weitere Latenzzeiten zu erreichen.
\end{definitionForm}

\begin{definitionForm}[KA-P-2]
Die Fertigungsdurchführung ist, beispielhaft an diversen von ihm betreuten Industiebetrieben, durch \textbf{geringe Transparenz} gekennzeichnet. Durch fehlende Einsichten in den Zwischenstand eines Fertigunsauftrags, hat ein Fertigungssteuerer aufgrund mangelnder Informationen kaum Möglichkeiten in Echtzeit auf veränderte Situationen zu reagieren. Folgen können beispielsweise Auslastungslücken oder Verzögerungen in der Produktion sein. Diese Transparenz kann nur aufwändig über persönliche Nachfrage in der Produktion hergestellt werden, da der Rücklauf der papierbasierten Rückmeldungen erst am Folgetag erfolgt. Eine andere Möglichkeit sind die Rückmeldungen über Terminals in den Werken, da diese meist lange Laufzeiten mit sich bringen, werden sie jedoch in der Regel erst kurz vor Schichtende in gesammelter Form getätigt. 
\end{definitionForm}

\begin{definitionForm}[KA-P-3]
Die Fertigungsdurchführung ist, beispielhaft an diversen von ihm betreuten Industiebetrieben, durch \textbf{hohen Erfassungsaufwand} gekennzeichnet. Durch die manuelle Erfassung auf Papier und der anschließenden Übertragung in das ERP-System erhöht sich der notwendige Aufwand bis die Daten digital vorliegen. Die Rücknahme fehlerhafter Rückmeldungen oder Änderungen an getätigten Rückmeldungen sind durch die papierbasierte Rückmeldung schwer durchführbar. Auch die Rückmeldung an Terminals in den Werken führt zu zusätzlichen Laufzeiten. Die Informationen die zurückgemeldet werden müssen, werden in der Zwischenzeit dennoch in Papierform gehalten. 
\end{definitionForm}

\begin{definitionForm}[KA-P-4]
Die Fertigungsdurchführung ist, beispielhaft an diversen von ihm betreuten Industiebetrieben, durch \textbf{mangelhafte Datenqualität} gekennzeichnet. Durch die Abgabe der Rückmeldungen in Papierform kommt es oft zu unvollständigen oder gar fehlerhaften Angaben. \textbf{Fehlende Plausibilitätsprüfungen} erhöhen das Risiko hierfür. Führt der Werker die Rückmeldung ins ERP-System selbst über ein Terminal in den Werken durch, führt das im Normalfall zu keinen Problemen, da er die Daten korrigieren kann. 
\end{definitionForm}

\begin{definitionForm}[KA-P-5]
Die Erfassung von \textbf{Fertigungsrückmeldungen in digitaler Form} wird von diversen von ihm betreuten Industiebetrieben getestet und angestrebt. Die Strategie einer papierlosen Fertigung wird ist in der Praxis noch nicht weit verbreitet, wird aber von vielen Industiebetrieben angestrebt.
\end{definitionForm}

\begin{definitionForm}[KA-P-6]
Die Rückmeldung von Fertigungsaufträgen erfolgt in der Regel auf der Vorgangsebene. Die minimal notwendigen Informationen sind: \textbf{Auftragsnummer, Vorgangsnummer, Gut-, Ausschuss- und Nacharbeitsmenge und Durchlaufszeit.}
Anhänge in Form von Dateien sind eine sinnvolle Ergänzung. Die Meldung von Defekten erfolgt parallel zur Rückmeldung. 
Die Soll-Werte sind als Vorschlag vorgegeben.
\end{definitionForm}

\begin{definitionForm}[KA-P-7]
Die \textbf{Erfassung von Defekten} ist ein wesentlicher Bestandteil der Rückmeldung in der Fertigungsdurchführung. Diese Funktionalität ist im Modul Produktionsplanung- und steuerung von SAP S/4HANA jedoch nicht integriert.
\end{definitionForm}

\begin{definitionForm}[KA-P-8]
Der \textbf{Einsatz von mobilen Geräten} hat ein enormes Potenzial die Fertigungsdurchführung effizienter zu gestalten. Eine Durchführung der Betriebsdatenerfassung über einen Barcode-Scanner oder über Text- bzw. Formularerkunng ist ein Anwendnungsfall der ebenfalls ein sehr hohes Potenzial für der diskreten Fertigung hat.
\end{definitionForm}

\begin{definitionForm}[KA-P-9]
Eine \textbf{Arbeitsplatz-spezifische Vorgangsliste} zur Übersicht der anstehenden Fertigungsaufträge pro Werker, ist eine weit verbreitete kundenspezifische Erweiterung. Diese dient der einfacheren Arbeitseinteilung indem der Werker die Übersicht über die ihm zugewiesenen Aufträge behalten kann und diese selbst in gewünschter Form, digital oder in Papierform, abarbeiten kann.
\end{definitionForm}