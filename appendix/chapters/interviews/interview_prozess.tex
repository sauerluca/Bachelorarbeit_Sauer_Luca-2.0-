% **************************************************************************************
% * Prozessexperten                                                               
% **************************************************************************************
\section*{Interview mit einem Prozessexperten}

\paragraph{Experte für das Experteninterview}
\textit{Herr Otto Normalverbraucher \footnote{Der Name \textit{Otto Normalverbraucher} repräsentiert den Namen des Prozessexperten.}} wird für das Experteninterview befragt.
Er ist einer der Experten des Beratungsbereiches \enquote{Manufacturing Industries} der \textit{ERP SE \footnote{Die Firma \textit{ERP SE} repräsentiert den Arbeitgeber des Prozessexperten.}}, welcher Betriebe, aus der Automobilindustrie, im Maschinen- und Anlagenbau sowie der High-Tech Industrie, bei der Gestaltung branchenspezifischer Prozessabläufe und ihrer Umsetzung mit SAP-Software unterstützt.
In seiner Rolle als \enquote{Senior Business Process Consultant} bewertet er Kundenanforderungen, stimmt sie auf die Geschäftsprozesse ab, konzeptioniert die Abbildung mittels SAP Lösungen in einer Gesamtlösungsarchitektur und sorgt für die Implementierung aus Sicht der Lösungsarchitektur.
Basierend auf den ihm vorliegenden Systemlandschaftsdaten erarbeitet er detaillierte Konzepte durchgängiger, branchenspezifischer Geschäftsprozesse und identifiziert dabei mögliche Ursachen, die zur Entstehung der kritischen Defizite im Geschäftsprozess führen.
Im Dialog mit den Kunden hilft sie diesen, die kritischen Defizite nachhaltig zu beseitigen. 
 
Diese Tätigkeit übt \textit{Herr Otto Normalverbraucher} seit dem Jahr 1999 (seit dem Jahr 2004 bei der \textit{ERP SE}) aus und besitzt zum Zeitpunkt der Erstellung dieser Bachelorarbeit damit eine Berufserfahrung von 20 Jahren.
Aufgrund dieser besitzt er einen großen Erfahrungsschatz hinsichtlich der Charakteristika von etablierten Arbeitsprozessen und Hilfsmitteln zur Fertigungsdurchführung im SAP-Umfeld, welche anhand dieses Experteninterviews greifbar gemacht werden soll.

\paragraph{Transkription des Experteninterviews}
\paragraph{Auswertung der Kernaussagen}