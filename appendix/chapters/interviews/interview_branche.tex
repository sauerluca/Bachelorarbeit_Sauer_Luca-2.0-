% **************************************************************************************
% * Anwenderperspektive                                                               
% **************************************************************************************
\tocless\section{Interview mit einem Branchenexperten}

\spaceparagraph{Experte für das Experteninterview}
\textit{Prof. Dr.-Ing. Max Mustermann \footnote{Der Name \textit{Max Mustermann} repräsentiert den Namen des Branchenexperten.}} wird für das Experteninterview befragt.
Er ist Institutsleiter des Instituts für Produktionsmanagement, Technologie und Werkzeugmaschinen der \textit{Technische Universität Musterstadt \footnote{Die \textit{Technische Universität Musterstadt} repräsentiert das Bildungsinstitut des Branchenexperten.}}, in dessen Zuständigkeit die folgenden Forschungsbereiche fallen: Digitalisierung und Vernetzung von Fertigungsprozessen, Echtzeitdatenerfassung und Automatisierungstechnik für die diskrete Fertigung, Werkzeugmaschinen und Industrierobotik \footnote{Die Aufzählung der Forschungsbereiche dient lediglich als Überblick und ist nicht vollständig.}

Die Tätigkeit des Institutsleiters übt \textit{Prof. Dr.-Ing. Max Mustermann} seit dem Jahr 2019 an der \textit{Technischen Universität Musterstadt} aus. Seit dem Jahr 2002 hatte er diverse Leitungsfunktionen mit den Themenschwerpunkten Werkzeugtechnologie, Produktionsplanung, Prototypen- und Serienfertigung, Betriebsmittelkonstruktion und Automatisierungstechnik sowie Engineering und Innovationsmanagement bei der \textit{Musterberger Druckmaschinen AG \footnote{Die \textit{Musterberger Druckmaschinen AG} repräsentiert einen Arbeitgeber des Branchenexperten.}}, sowie der \textit{ERP SE \footnote{Die Firma \textit{ERP SE} repräsentiert einen Arbeitgeber des Branchenexperten.}} inne. Zum Zeitpunkt der Erstellung dieser Bachelorarbeit arbeitet er damit seit 17 Jahren im Fachbereich Produktionsmanagement. Aufgrund dessen besitzt er einen großen Erfahrungsschatz im Umfeld der industrienahen Forschung und ein ausgeprägtes Verständnis der tatsächlichen Kundenbedürfnisse an die Fertigungsdurchführung, welches anhand dieses Experteninterviews greifbar gemacht werden soll.

\spaceparagraph{Transkription des Experteninterviews}
\textit{Interviewer}: Guten Tag Herr Mustermann. Danke, dass Sie sich für dieses Experteninterview Zeit genommen haben.
\newline
\textit{Max Mustermann}: Guten Tag Herr Sauer. Sehr gerne.

\textit{Interviewer}: : Das Experteninterview wird sich so strukturieren, dass wir erst formale Aspekte klären. Danach werde ich Ihnen Fragen zu den Merkmalen und Defiziten von Prozessschritten in der Fertigungsdurchführung stellen. Für das Experteninterview sind 60 Minuten angesetzt. Falls Sie Pausen oder kurze Unterbrechungen benötigen, geben Sie mir bitte kurz Bescheid. Sind das Vorgehen und der Rahmen aus Ihrer Sicht in Ordnung?
\newline
\textit{Max Mustermann}: Danke Ihrer Nachfrage, das passt.

\textit{Interviewer}: Das Experteninterview führe ich im Rahmen meiner Bachelorarbeit durch. Darf ich alle Aussagen und Erkenntnisse aus dem Experteninterview auswerten und in meine Bachelorarbeit einbeziehen?
\newline
\textit{Max Mustermann}: Ja.

% \textit{Interviewer}: Danke dafür. Wie lange beschäftigen Sie sich schon mit den Geschäftsprozessen der Produktionsplanung und -steuerung? Führen Sie ausschließlich Analysen zu kritischen Systemzuständen durch oder setzen Sie Vorschläge Ihrerseits auch direkt an Kundensystemen um?
% \newline
% \textit{Max Mustermann}: Meine Tätigkeit übe ich aus, seitdem die Datenbank SAP HANA 2010 auf den Markt gekommen ist. Das heißt, mittlerweile beschäftige ich mich schon acht Jahre mit der Thematik in Bezug auf die Systemlandschaftsdaten. Dabei arbeite ich im Rahmen des SAP EWA Next Generation des SAP S/4HANA Service and Support. Ich besitze den Status als Datenbankadministrator. Neben meiner Analysetätigkeit behebe ich kritische Sytemzustände in Zusammenarbeit mit dem Kunden.

% \textit{Interviewer}: Guten Tag Herr Mustermann. Danke, dass Sie sich für dieses Experteninterview Zeit genommen haben.
% \newline
% \textit{Max Mustermann}: Guten Tag Herr Sauer. Sehr gerne.
