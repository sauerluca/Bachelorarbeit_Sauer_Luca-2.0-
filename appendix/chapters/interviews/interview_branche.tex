% **************************************************************************************
% * Anwenderperspektive                                                               
% **************************************************************************************
\tocless\section{Interview mit einem Branchenexperten}\label{ah:interviewTUD}

\spaceparagraph{Experte für das Experteninterview}
\textit{Prof. Dr.-Ing. Max Mustermann \footnote{Der Name \textit{Max Mustermann} repräsentiert den Namen des Branchenexperten.}} wird für das Experteninterview befragt.
Er ist Institutsleiter des Instituts für Produktionsmanagement, Technologie und Werkzeugmaschinen der \textit{Technische Universität Musterstadt \footnote{Die \textit{Technische Universität Musterstadt} repräsentiert das Bildungsinstitut des Branchenexperten.}}, in dessen Zuständigkeit die folgenden Forschungsbereiche fallen: Digitalisierung und Vernetzung von Fertigungsprozessen, Echtzeitdatenerfassung und Automatisierungstechnik für die diskrete Fertigung, Werkzeugmaschinen und Industrierobotik \footnote{Die Aufzählung der Forschungsbereiche dient lediglich als Überblick und ist nicht vollständig.}

Die Tätigkeit des Institutsleiters übt \textit{Prof. Dr.-Ing. Max Mustermann} seit dem Jahr 2019 an der \textit{Technischen Universität Musterstadt} aus. Seit dem Jahr 2002 hatte er diverse Leitungsfunktionen mit den Themenschwerpunkten Werkzeugtechnologie, Produktionsplanung, Prototypen- und Serienfertigung, Betriebsmittelkonstruktion und Automatisierungstechnik sowie Engineering und Innovationsmanagement bei der \textit{Musterberger Druckmaschinen AG \footnote{Die \textit{Musterberger Druckmaschinen AG} repräsentiert einen Arbeitgeber des Branchenexperten.}}, sowie der \textit{ERP SE \footnote{Die Firma \textit{ERP SE} repräsentiert einen Arbeitgeber des Branchenexperten.}} inne. Zum Zeitpunkt der Erstellung dieser Bachelorarbeit ist er damit seit 17 Jahren im Fachbereich Produktionsmanagement tätig. Aufgrund dessen besitzt er einen großen Erfahrungsschatz im Umfeld der industrienahen Forschung und ein ausgeprägtes Verständnis für die tatsächlichen Abläufe einer Fertigungsdurchführung in der diskreten Industrie, welches anhand dieses Experteninterviews greifbar gemacht werden soll.

\spaceparagraph{Transkription des Experteninterviews}
\textit{Interviewer}: Guten Tag Herr Mustermann. Danke, dass Sie sich für dieses Experteninterview Zeit genommen haben.
\newline
\textit{Max Mustermann}: Guten Tag Herr Sauer. Sehr gerne.

\textit{Interviewer}: : Das Experteninterview wird sich so strukturieren, dass wir erst formale Aspekte klären. Danach werde ich Ihnen Fragen zu Merkmalen und Defiziten von Aktivitäten in der Fertigungsdurchführung stellen. Für das Experteninterview sind 60 Minuten angesetzt. Falls Sie Pausen oder kurze Unterbrechungen benötigen, geben Sie mir bitte kurz Bescheid. Sind das Vorgehen und der Rahmen aus Ihrer Sicht in Ordnung?
\newline
\textit{Max Mustermann}: Danke Ihrer Nachfrage, das passt.

\textit{Interviewer}: Das Experteninterview führe ich im Rahmen meiner Bachelorarbeit durch. Darf ich alle Aussagen und Erkenntnisse aus dem Experteninterview auswerten und in meine Bachelorarbeit einbeziehen?
\newline
\textit{Max Mustermann}: Ja.

% Herr Bourlauf, auch in der Bankenbranche ist die Industrialisierung von
% Geschäftsprozessen ein viel diskutiertes Thema. Wie schätzen Sie den aktuellen
% Stand ein?

% es  wird viel geschrieben über die Industrialisierung bei Banken. Von
% Standardisierung, Taylorisierung, Automatisierung, Modularisierung, kontinuierlicher
% Verbesserung, Konzentration auf Kernkompetenzen und Reduzierung der Fertigungstiefe ist die Rede. Die Automobilbranche, die unterschiedliche Fabrikate auf einer Fertigungsstraße produziert, wird oft als Vorbild für die Finanzdienstleistungsbranche
% genommen. Green, Yellow, Black Belts und Ähnliches etablieren sich im Rahmen von
% Six Sigma als neue Berufsgruppen. Eine SOA hat heutzutage angeblich jedes Softwarehaus und jede IT-Abteilung, die etwas auf sich hält. Business Process Management (BPM) betreibt inzwischen jede moderne Organisation, sei es mit Visio, ARIS
% oder anderen »Maltools«. Kreditfabriken existieren und eigentlich sind die Banken
% doch schon sehr weit – oder?

% Welchen Weg sind Sie gegangen?

%  Die Degussa Bank ist eine im Geschäftsmodell streng fokussierte Privatbank, die mit über 600 Mitarbeitern in über 210 Zweigstellen das Privatkundengeschäft in Deutschland betreibt. Wir haben in den letzten beiden Jahren schrittweise
% begonnen, zwei Bereiche zu industrialisieren und zu reorganisieren: den Immobilienkreditbereich und erhebliche Teile des Kundenservices mit über 70 Prozessen aus
% Konto- und Kundenservice. Die Geschäftsprozesse beider Bereiche haben unterschiedliche Charakteristika und Anforderungen. Während es im Immobilienkreditbereich um relativ wenige, aber dafür komplexe Prozessarten geht, gibt es im Kundenservice viele unterschiedliche Prozessarten mit hoher Agilität.
% Bei der Umgestaltung haben wir uns von zwei Grundgedanken leiten lassen:
% Erstens werden alle Prozesse »End-to-End« vom Kunden aus betrachtet. Wir sprechen hier von kundenfokussierter Fertigung, zum Kunden hin individuell, in der Fertigung weitestgehend standardisiert. Zweitens denken wir betriebswirtschaftlich – analog zur Automobilindustrie – in Fertigungsstraßen und Standards, damit die Lösungen
% als »Blueprint« einsetzbar sind.
% Im Rahmen unserer serviceorientierten Architektur sind heute alle Prozesse in
% einer BPM-Plattform abgebildet und werden dort orchestriert und überwacht. Die
% BPM-Plattform bildet das Bindeglied zwischen unserer SOA und der Fachabteilung.
% Es gibt nun in allen Bereichen Produktionsleitstände, mit deren Hilfe wir u.a. eine
% deutlich flexiblere Arbeitsorganisation erreicht haben. Je nach tatsächlicher oder
% erwarteter Auslastung der Teams werden Finanzprodukte von einer Fertigungsstraße
% auf eine andere Fertigungsstraße verschoben oder Kollegen aus einem Team virtuell
% einem anderen Team zugeordnet, sofern sie über die entsprechenden Fähigkeiten
% verfügen. Zusätzlich besteht die Möglichkeit, Zweigstellen anderen Betreuungsteams
% zuzuordnen.

% : Mit der Industrialisierung von Geschäftsprozessen verbunden sind sicherlich erhebliche Auswirkungen auf die Organisation, denen aber auch große Nutzenpotenziale gegenüberstehen. Können Sie uns diese kurz beschreiben?


% In der Tat hat die Industrialisierung erhebliche Auswirkungen auf unsere
% Gesamtorganisation und führt zu deutlichen Verbesserungen für unsere Servicequalität und Kostenposition. Die wichtigsten Aspekte aus meiner Sicht sind folgende:
% ■ Geschäftlich relevante Prozesse laufen nun automatisiert in der BPM-Plattform ab
% und sind transparent für jeden verfügbar. Das ist die Grundlage für alle weiteren
% Möglichkeiten, die mit der Industrialisierung einhergehen. Voraussetzung hierfür
% war neben der Einführung der BPM-Plattform in unserem Fall auch die Digitalisierung der gesamten Kundenkorrespondenz und damit die Einführung der elektronischen Kredit- und Kundenakte.
% ■ Da alle Informationen im System transparent vorhanden sind, können wir – egal
% über welchen Vertriebskanal der Kunde uns anspricht – direkt Auskunft über den
% Vorgang geben.
% ■ Alle betroffenen Prozesse wurden standardisiert und können nun leicht verändert
% werden. Das bedeutet, dass wir jederzeit Prozessabschnitte taylorisieren oder
% zusammenfassen können, abhängig vom Mengenaufkommen und den Fähigkeiten
% der verfügbaren Mitarbeiter. Uns steht ein Pool an praxisgetesteten Prozessabschnitten zur Verfügung, die wir als Vorlage nutzen und rasch an sich schnell verändernde Marktbedingungen anpassen können. Außerdem können wir auf diese
% Weise unser Personal flexibler einsetzen, die erforderlichen Skill-Levels in einzelnen Prozessschritten senken und auf Engpässe zeitnah reagieren. Damit wird ein
% einheitlicher Qualitätsstandard gesichert, volle Transparenz erzeugt und die Einhaltung definierter Service Levels über alle Arbeitsprozesse hinweg überprüfbar.
% ■ Die Fertigungstiefe kann die Bank jederzeit selbst bestimmen. Damit können wir
% z.B. einfachere Tätigkeiten im Rahmen der Kampagnenbearbeitung an externe
% Dienstleister auslagern und unsere hoch qualifizierten Mitarbeiter von monotonen
% Tätigkeiten entlasten.
% ■ Die Strukturen der betroffenen Abteilungen wurden im Alignment mit den Prozessen aufgestellt. Durch das Aufsetzen eines kontinuierlichen Verbesserungsprozesses unter Verwendung der statistischen Kennzahlen aus dem BPM-System kann
% die Fachabteilung die Prozesse und ihre eigene Organisation optimieren. Die
% Reorganisation wurde aus der Fachabteilung initiiert, bedurfte keiner externen
% Beratungsgesellschaft, und die Organisationsstruktur passte sich Schritt für Schritt
% dem optimierten Prozessmodell an.
% ■ Das Know-how der Mitarbeiter ist nun in der BPM-Plattform abgebildet und allgemein zugänglich. Dadurch ist die ursprüngliche Profession der Mitarbeiter teilweise
% verloren gegangen, neue Rollen und Fähigkeiten mussten erlernt und ausgebildet
% werden.
% ■ Last but not least: Wir haben durch diese Maßnahmen Effizienzsteigerungen von
% 30 Prozent und mehr erreicht.

% \textit{Interviewer}: Danke dafür. Wie lange beschäftigen Sie sich schon mit den Geschäftsprozessen der Produktionsplanung und -steuerung? Führen Sie ausschließlich Analysen zu kritischen Systemzuständen durch oder setzen Sie Vorschläge Ihrerseits auch direkt an Kundensystemen um?
% \newline
% \textit{Max Mustermann}: Meine Tätigkeit übe ich aus, seitdem die Datenbank SAP HANA 2010 auf den Markt gekommen ist. Das heißt, mittlerweile beschäftige ich mich schon acht Jahre mit der Thematik in Bezug auf die Systemlandschaftsdaten. Dabei arbeite ich im Rahmen des SAP EWA Next Generation des SAP S/4HANA Service and Support. Ich besitze den Status als Datenbankadministrator. Neben meiner Analysetätigkeit behebe ich kritische Sytemzustände in Zusammenarbeit mit dem Kunden.

% \textit{Interviewer}: Guten Tag Herr Mustermann. Danke, dass Sie sich für dieses Experteninterview Zeit genommen haben.
% \newline
% \textit{Max Mustermann}: Guten Tag Herr Sauer. Sehr gerne.
