% **************************************************************************************
% * Anwenderperspektive                                                               
% **************************************************************************************
\tocless\section{Interview mit einem Branchenexperten}\label{ah:interviewTUD}

\spaceparagraph{Experte für das Experteninterview}
\textit{Prof. Dr.-Ing. Max Mustermann \footnote{Der Name \textit{Max Mustermann} repräsentiert den Namen des Branchenexperten.}} wird für das Experteninterview befragt.
Er ist Institutsleiter des Instituts für Produktionsmanagement, Technologie und Werkzeugmaschinen der \textit{Technische Universität Musterstadt \footnote{Die \textit{Technische Universität Musterstadt} repräsentiert das Bildungsinstitut des Branchenexperten.}}, in dessen Zuständigkeit die folgenden Forschungsbereiche fallen: Digitalisierung und Vernetzung von Fertigungsprozessen, Echtzeitdatenerfassung und Automatisierungstechnik für die diskrete Fertigung, Werkzeugmaschinen und Industrierobotik \footnote{Die Aufzählung der Forschungsbereiche dient lediglich als Überblick und ist nicht vollständig.}

Die Tätigkeit des Institutsleiters übt \textit{Prof. Dr.-Ing. Max Mustermann} seit dem Jahr 2019 an der \textit{Technischen Universität Musterstadt} aus. Seit dem Jahr 2002 hatte er diverse Leitungsfunktionen mit den Themenschwerpunkten Werkzeugtechnologie, Produktionsplanung, Prototypen- und Serienfertigung, Betriebsmittelkonstruktion und Automatisierungstechnik sowie Engineering und Innovationsmanagement bei der \textit{Musterberger Druckmaschinen AG \footnote{Die \textit{Musterberger Druckmaschinen AG} repräsentiert einen Arbeitgeber des Branchenexperten.}}, sowie der \textit{ERP SE \footnote{Die Firma \textit{ERP SE} repräsentiert einen Arbeitgeber des Branchenexperten.}} inne. Zum Zeitpunkt der Erstellung dieser Bachelorarbeit ist er damit seit 17 Jahren im Fachbereich Produktionsmanagement tätig. Aufgrund dessen besitzt er einen großen Erfahrungsschatz im Umfeld der industrienahen Forschung und ein ausgeprägtes Verständnis für die tatsächlichen Abläufe einer Fertigungsdurchführung in der diskreten Industrie, welches anhand dieses Experteninterviews greifbar gemacht werden soll.

\spaceparagraph{Kernaussagen des Experteninterviews}

\begin{definitionForm}[KA-B-1]
Die Fertigungsdurchführung ist, beispielhaft an seinem früheren Arbeitgeber \textit{Musterberger Druckmaschinen AG}, durch \textbf{geringe Flexibilität und Geschwindigkeit} gekennzeichnet. Der Auslöser einer Fertigungsdurchführung ist die persönliche Übergabe eines Fertigungsauftrags vom Fertigungssteuerer an den Werker. Die Übergabe erfolgt mit dem Ausdrucken von Papieren und einem anschließenden Zusammenkommen. Dabei entstehen unnötige Laufwege und Wartezeiten in der Produktion. Bei kurzfristigen Änderungen aufgrund von aufgetretenen Ereignissen, wie einer Veränderung der Mengen im Fertigungsauftrag, besteht keine Möglichkeit dies ohne weitere Latenzzeiten zu erreichen.
\end{definitionForm}

\begin{definitionForm}[KA-B-2]
Die Fertigungsdurchführung ist, beispielhaft an seinem früheren Arbeitgeber \textit{Musterberger Druckmaschinen AG}, durch \textbf{geringe Transparenz} gekennzeichnet. Durch fehlende Einsichten in den Zwischenstand eines Fertigunsauftrags, hat ein Fertigungssteuerer aufgrund mangelnder Informationen kaum Möglichkeiten in Echtzeit auf veränderte Situationen zu reagiern. Auch Meldungen von Defekten (nicht einsetzbare Hilfsmittel oder Maschinen, Qualitätsprobleme etc.) werden oft erst am Ende des Tages getätigt. Folgen können beispielsweise Auslastungslücken oder Verzögerungen in der Produktion sein. Diese Transparenz kann nur aufwändig über persönliche Nachfrage in der Produktion hergestellt werden, da der Rücklauf der papierbasierten Rückmeldungen erst am Folgetag erfolgt. Eine andere Möglichkeit sind die Rückmeldungen über Terminals in den Werken. Da diese meist lange Laufzeiten mit sich bringen, werden sie jedoch in der Regel erst kurz vor Schichtende in gesammelter Form getätigt. 
\end{definitionForm}

\begin{definitionForm}[KA-B-3]
Die Fertigungsdurchführung ist, beispielhaft an seinem früheren Arbeitgeber \textit{Musterberger Druckmaschinen AG}, durch \textbf{hohen Erfassungsaufwand} gekennzeichnet. Durch die manuelle Erfassung auf Papier und der anschließenden Übertragung in das ERP-System erhöht sich der notwendige Aufwand bis die Daten digital vorliegen. Die Rücknahme fehlerhafter Rückmeldungen oder Änderungen an getätigten Rückmeldungen sind durch die papierbasierte Rückmeldung schwer durchführbar. Auch die Rückmeldung an Terminals in den Werken führt zu zusätzlichen Laufzeiten. Die Informationen die zurückgemeldet werden müssen, werden in der Zwischenzeit dennoch in Papierform gehalten. 
\end{definitionForm}

\begin{definitionForm}[KA-B-4]
Die Fertigungsdurchführung ist, beispielhaft an seinem früheren Arbeitgeber \textit{Musterberger Druckmaschinen AG}, durch \textbf{mangelhafte Datenqualität} gekennzeichnet. Durch die Abgabe der Rückmeldungen in Papierform kommt es oft zu unvollständigen oder gar fehlerhaften Angaben. \textbf{Fehlende Plausibilitätsprüfungen} erhöhen das Risiko hierfür. Führt der Werker die Rückmeldung ins ERP-System selbst über ein Terminal in den Werken durch, führt das im Normalfall zu keinen Problemen, da er die Daten korrigieren kann. In seiner Rolle als Produktionsleiter bei der \textit{Musterberger Druckmaschinen AG} machte er dabei die Erfahrung, das ca. 5-15\% aller Rückmeldungen fehlerhaft sind, je nach Auftragstyp.
\end{definitionForm}

\begin{definitionForm}[KA-B-5]
Die Erfassung von Fertigungsrückmeldungen in Papierform ist zum jetzigen Zeitpunkt in den meisten ihm bekannten Industriebetrieben \textbf{nicht vollständig oder gar nicht ersetzbar}. Alternative Ansätze bringen entscheidende Vorteile und können zu einer gesteigerten Produktivität führen. Eine vollautomatisierte Rückmeldung der Fertigungsdurchführung ist in der diskreten Fertigung nicht möglich.
\end{definitionForm}

\begin{definitionForm}[KA-B-6]
Die Rückmeldung von Fertigungsaufträgen erfolgt in der Regel auf der Vorgangsebene. Die minimal notwendigen Informationen sind: \textbf{Auftragsnummer, Vorgangsnummer, Gut-, Ausschuss- und Nacharbeitsmenge und Durchlaufszeit.}
Anhänge in Form von Dateien sind eine sinnvolle Ergänzung.
\end{definitionForm}

\begin{definitionForm}[KA-B-7]
Der \textbf{Einsatz von mobilen Geräten} hat ein enormes Potenzial die Fertigungsdurchführung effizienter zu gestalten. Eine Durchführung der Betriebsdatenerfassung über eine Spachsteuerung ist ein Anwendnungsfall der ebenfalls Vorteile in der diskreten Fertigung bringen kann. Bei der \textit{Musterberger Druckmaschinen AG} haben nicht alle Betriebsmittel einen Strichcode, weswegen ein Scanner nicht optimal eingesetzt werden kann.
\end{definitionForm}