% **************************************************************************************
% * Anwenderperspektive                                                               
% **************************************************************************************
\section*{Interview mit einem Branchenexperten}

\paragraph{Experte für das Experteninterview}
\textit{Prof. Dr.-Ing. Max Mustermann \footnote{Der Name \textit{Max Mustermann} repräsentiert den Namen des Branchenexperten.}} wird für das Experteninterview befragt.
Er ist Institutsleiter des Instituts für Produktionsmanagement, Technologie und Werkzeugmaschinen der \textit{Technische Universität Musterstadt \footnote{Die \textit{Technische Universität Musterstadt} repräsentiert das Bildungsinstitut des Branchenexperten.}}, in dessen Zuständigkeit die folgenden Forschungsbereiche fallen: Digitalisierung und Vernetzung von Fertigungsprozessen, Echtzeitdatenerfassung und Automatisierungstechnik für die diskrete Fertigung, Werkzeugmaschinen und Industrierobotik \footnote{Die Aufzählung der Forschungsbereiche dient lediglich als Überblick und ist nicht vollständig.}

Die Tätigkeit des Institutsleiters übt \textit{Prof. Dr.-Ing. Max Mustermann} seit dem Jahr 2019 an der \textit{Technischen Universität Musterstadt} aus. Seit dem Jahr 2002 hatte er diverse Leitungsfunktionen mit den Themenschwerpunkten Werkzeugtechnologie, Produktionsplanung, Prototypen- und Serienfertigung, Betriebsmittelkonstruktion und Automatisierungstechnik sowie Engineering und Innovationsmanagement bei der \textit{Musterberger Druckmaschinen AG \footnote{Die \textit{Musterberger Druckmaschinen AG} repräsentiert einen Arbeitgeber des Branchenexperten.}}, sowie der \textit{ERP SE \footnote{Die Firma \textit{ERP SE} repräsentiert einen Arbeitgeber des Branchenexperten.}} inne. Zum Zeitpunkt der Erstellung dieser Bachelorarbeit arbeitet er damit seit 17 Jahren im Fachbereich Produktionsmanagement. Aufgrund dessen besitzt er einen großen Erfahrungsschatz im Umfeld der industrienahen Forschung und ein ausgeprägtes Verständnis der tatsächlichen Kundenbedürfnisse an die Fertigungsdurchführung, welches anhand dieses Experteninterviews greifbar gemacht werden soll.

\paragraph{Transkription des Experteninterviews}
\paragraph{Auswertung der Kernaussagen}