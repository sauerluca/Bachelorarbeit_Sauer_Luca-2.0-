% **************************************************************************************
% * Technologieexperten                                                               
% **************************************************************************************
\section*{Interview mit einem Technologieexperten}

\paragraph{Experte für das Experteninterview}
\textit{Herr John Doe \footnote{Der Name \textit{John Doe} repräsentiert den Namen des Technologieexperten.}} wird für das Experteninterview befragt.
 Er ist einer der Experten des Entwicklungsbereiches \enquote{S/4HANA Cloud Produce - Manufacturing} der \textit{ERP SE \footnote{Die Firma \textit{ERP SE} repräsentiert den Arbeitgeber des Technologieexperten.}}, in dessen Verantwortungsbereich die Anwendungen und Technologien zur Produktionsplanung und -steuerung der ERP-Software SAP S/4HANA entworfen und entwickelt werden.
 In seiner Rolle als \enquote{Development Expert} eruiert er Anforderungen an die ERP-Software, transformiert bestehende Geschäftsprozesse in Anwendungen, konzeptioniert die technologischen Hilfsmittel und sorgt für die Implementierung aus Sicht der Systemarchitektur. Basierend auf den ihm vorliegenden Geschäftsprozessen erarbeitet er detaillierte Konzepte standardisierter, prozessspezifischer Anwendungen und evaluiert dabei mögliche Technologien, die zur Unterstützung der Geschäftsprozesse fungieren. Im Dialog mit den Kunden, dem Beratungsbereich und dem Produktmanagment werden geeignete Technologien identifiziert, um die Geschäftsprozesse nachhaltig zu verbessern. 
 
Diese Tätigkeit übt \textit{Herr John Doe} seit dem Jahr 2000 bei der \textit{ERP SE} aus und besitzt zum Zeitpunkt der Erstellung dieser Bachelorarbeit damit eine Berufserfahrung von 19 Jahren. Aufgrund dieser besitzt er einen großen Erfahrungsschatz hinsichtlich der Anforderungen an geeignete Technologien der angebotenen SAP-Software zur Fertigungsdurchführung, welche anhand dieses Experteninterviews greifbar gemacht werden soll.

\paragraph{Transkription des Experteninterviews}
\paragraph{Auswertung der Kernaussagen}

