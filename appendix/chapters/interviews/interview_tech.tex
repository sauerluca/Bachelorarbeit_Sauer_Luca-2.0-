% **************************************************************************************
% * Technologieexperten                                                               
% **************************************************************************************
\tocless\section{Interview mit einem Technologieexperten}\label{ah:interviewDev}

\spaceparagraph{Experte für das Experteninterview}
\textit{Herr John Doe \footnote{Der Name \textit{John Doe} repräsentiert den Namen des Technologieexperten.}} wird für das Experteninterview befragt.
 Er ist einer der Experten des Entwicklungsbereiches \enquote{S/4HANA Cloud Produce - Manufacturing} der \textit{ERP SE \footnote{Die Firma \textit{ERP SE} repräsentiert den Arbeitgeber des Technologieexperten.}}, in dessen Verantwortungsbereich die Anwendungen und Technologien zur Produktionsplanung und -steuerung der \ac{ERP}-Software SAP S/4HANA entworfen und entwickelt werden.
 In seiner Rolle als \enquote{Development Expert} eruiert er Anforderungen an die \ac{ERP}-Software, transformiert bestehende Geschäftsprozesse in Anwendungen, konzeptioniert die technologischen Hilfsmittel und sorgt für die Implementierung aus Sicht der Systemarchitektur. Basierend auf den ihm vorliegenden Geschäftsprozessen erarbeitet er detaillierte Konzepte standardisierter, prozessspezifischer Anwendungen und evaluiert dabei mögliche Technologien, die zur Unterstützung der Geschäftsprozesse fungieren. Im Dialog mit den Kunden, dem Beratungsbereich und dem Produktmanagement werden geeignete Technologien identifiziert, um die Geschäftsprozesse nachhaltig zu verbessern. 
 
Diese Tätigkeit übt \textit{Herr John Doe} seit dem Jahr 2000 bei der \textit{ERP SE} aus und besitzt zum Zeitpunkt der Erstellung dieser Bachelorarbeit damit eine Berufserfahrung von 19 Jahren. Aufgrund dieser besitzt er einen großen Erfahrungsschatz hinsichtlich der Anforderungen an geeignete Technologien der angebotenen SAP-Software zur Fertigungsdurchführung, welche anhand dieses Experteninterviews greifbar gemacht werden soll.

\paragraph{Kernaussagen des Experteninterviews}

\begin{definitionForm}[KA-T-1]
Die Fertigungsdurchführung in SAP S/4HANA Cloud ist durch \textbf{mangelhafte Datenqualität} gekennzeichnet. Durch den Zeitdruck in der Produktion werden Rückmeldungen so schnell wie möglich erfasst. Warnungen werden hierbei ignoriert. Des weiteren nutzen nicht alle Kunden die Customizing-Funktionalität Toleranzregeln, für die Rückmeldung festzulegen.
\end{definitionForm}

\begin{definitionForm}[KA-T-2]
Die Erfassung von \textbf{Fertigungsrückmeldungen in digitaler Form} wird von Seiten des Entwicklungsbereiches \enquote{S/4HANA Cloud Produce - Manufacturing} empfohlen. Die Strategie einer papierlosen Fertigung findet in der Praxis noch keine weite Verbreitung.
\end{definitionForm}

\begin{definitionForm}[KA-T-3]
Die Rückmeldung von Fertigungsaufträgen erfolgt in der Regel auf der Vorgangsebene. Die minimal notwendigen Informationen sind: \textbf{Auftragsnummer, Vorgangsnummer, Gut-, Ausschuss- und Nacharbeitsmenge, Rüstzeit, Bearbeitungszeit und Abrüstzeit}. Die Angabe von Personalnummern ist eher unüblich und meist nicht notwendig.
\end{definitionForm}

\begin{definitionForm}[KA-T-4]
Der \textbf{Einsatz von mobilen Geräten} hat ein großes Potenzial die Fertigungsdurchführung effizienter zu gestalten. Eine Durchführung der Betriebsdatenerfassung über einen Barcode-Scanner oder über Text- bzw. Formularerkunng ist ein Anwendnungsfall der ebenfalls ein sehr hohes Potenzial für die diskrete Fertigung hat.
\end{definitionForm}

\begin{definitionForm}[KA-T-5]
Die Fertigungsdurchführung in SAP S/4HANA Cloud wird oft nur zur \textbf{Materialbedarfsplanung und Abbrechnung} genutzt. Dabei kommen zur eigentlichen Fertigungsdurchführung sogenannte MES-Systeme oder andere IT-Systeme zum Einsatz.
\end{definitionForm}

