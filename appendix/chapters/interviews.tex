\chapter{Interviewprotokolle}

\spaceparagraph{Vorwort zu den Experteninterviews}
Die folgenden Ausführungen fassen die Ergebnisse der im Rahmen dieser Bachelorarbeit durchgeführten Experteninterviews zusammen.
Die Auswertung erfolgt summarisch und wird nicht wortwörtlich transkribiert.
Dieses Vorgehen ist mit dem wissenschaftlichen Betreuer dieser Bachelorarbeit abgestimmt.
Die Interviewpartnerinnen und -partner haben der Veröffentlichung ihres Interviews in der Audio- oder Textversion zugestimmt.
Aus datenschutzrechtlichen Gründen wurden in den folgenden Interviewprotokollen alle Namen, Unternehmen und Institutionen durch Pseudonyme ersetzt und dementsprechend gekennzeichnet.

\begin{itemize}
  \item
        \textit{Thema der Experteninterviews:} 
        Die Experteninterviews beziehen sich auf die Herausarbeitung kritischer Defizite beziehungsweise kritischer Prozessstrukturen des Geschäftsprozesses zur Fertigungsdurchführung im SAP-Umfeld.
        Wodurch sich Faktoren eliminieren lassen, die in Form von Verschwendung negativ Einfluss auf den Prozess nehmen, wird im Rahmen des Experteninterviews ebenfalls eruiert. 
        Im Fokus stehen die Feststellung der etablierten Arbeitsprozesse und Hilfsmittel einer Fertigungsdurchführung und die Identifikation von alternativen Arbeitsprozessen und Hilfsmitteln, deren Merkmale und Defizite.
  \item 
        \textit{Ziel der Experteninterviews:}
        Mit den Experteninterviews sollen formulierbare Verbesserungspotenziale erhoben werden, anhand derer Anforderungen an den Geschäftsprozess zur Fertigungsdurchführung festgemacht werden.
        Diese sollen einen Kontext von Defiziten der Fertigungsdurchführung im SAP-Umfeld beschreiben und als Ausgangslage für die Evaluation zu untersuchender Maßnahmen dienen.
\end{itemize}

\newpage
% **************************************************************************************
% * Anwenderperspektive                                                               
% **************************************************************************************
\tocless\section{Interview mit einem Branchenexperten}\label{ah:interviewTUD}

\spaceparagraph{Experte für das Experteninterview}
\textit{Prof. Dr.-Ing. Max Mustermann \footnote{Der Name \textit{Max Mustermann} repräsentiert den Namen des Branchenexperten.}} wird für das Experteninterview befragt.
Er ist Institutsleiter des Instituts für Produktionsmanagement, Technologie und Werkzeugmaschinen der \textit{Technische Universität Musterstadt \footnote{Die \textit{Technische Universität Musterstadt} repräsentiert das Bildungsinstitut des Branchenexperten.}}, in dessen Zuständigkeit die folgenden Forschungsbereiche fallen: Digitalisierung und Vernetzung von Fertigungsprozessen, Echtzeitdatenerfassung und Automatisierungstechnik für die diskrete Fertigung, Werkzeugmaschinen und Industrierobotik \footnote{Die Aufzählung der Forschungsbereiche dient lediglich als Überblick und ist nicht vollständig.}

Die Tätigkeit des Institutsleiters übt \textit{Prof. Dr.-Ing. Max Mustermann} seit dem Jahr 2019 an der \textit{Technischen Universität Musterstadt} aus. Seit dem Jahr 2002 hatte er diverse Leitungsfunktionen mit den Themenschwerpunkten Werkzeugtechnologie, Produktionsplanung, Prototypen- und Serienfertigung, Betriebsmittelkonstruktion und Automatisierungstechnik sowie Engineering und Innovationsmanagement bei der \textit{Musterberger Druckmaschinen AG \footnote{Die \textit{Musterberger Druckmaschinen AG} repräsentiert einen Arbeitgeber des Branchenexperten.}}, sowie der \textit{ERP SE \footnote{Die Firma \textit{ERP SE} repräsentiert einen Arbeitgeber des Branchenexperten.}} inne. Zum Zeitpunkt der Erstellung dieser Bachelorarbeit ist er damit seit 17 Jahren im Fachbereich Produktionsmanagement tätig. Aufgrund dessen besitzt er einen großen Erfahrungsschatz im Umfeld der industrienahen Forschung und ein ausgeprägtes Verständnis für die tatsächlichen Abläufe einer Fertigungsdurchführung in der diskreten Industrie, welches anhand dieses Experteninterviews greifbar gemacht werden soll.

\spaceparagraph{Kernaussagen des Experteninterviews}

\begin{definitionForm}[KA-B-1]
Die Fertigungsdurchführung ist, beispielhaft an seinem früheren Arbeitgeber \textit{Musterberger Druckmaschinen AG}, durch \textbf{geringe Flexibilität und Geschwindigkeit} gekennzeichnet. Der Auslöser einer Fertigungsdurchführung ist die persönliche Übergabe eines Fertigungsauftrags vom Fertigungssteuerer an den Werker. Die Übergabe erfolgt mit dem Ausdrucken von Papieren und einem anschließenden Zusammenkommen. Dabei entstehen unnötige Laufwege und Wartezeiten in der Produktion. Bei kurzfristigen Änderungen aufgrund von aufgetretenen Ereignissen, wie einer Veränderung der Mengen im Fertigungsauftrag, besteht keine Möglichkeit dies ohne weitere Latenzzeiten zu erreichen.
\end{definitionForm}

\begin{definitionForm}[KA-B-2]
Die Fertigungsdurchführung ist, beispielhaft an seinem früheren Arbeitgeber \textit{Musterberger Druckmaschinen AG}, durch \textbf{geringe Transparenz} gekennzeichnet. Durch fehlende Einsichten in den Zwischenstand eines Fertigunsauftrags, hat ein Fertigungssteuerer aufgrund mangelnder Informationen kaum Möglichkeiten in Echtzeit auf veränderte Situationen zu reagiern. Auch Meldungen von Defekten (Nicht einsetzbare Hilfsmittel oder Maschinen, Qualitätsprobleme etc.) werden oft erst am Ende des Tages getätigt. Folgen können beispielsweise Auslastungslücken oder Verzögerungen in der Produktion sein. Diese Transparenz kann nur aufwändig über persönliche Nachfrage in der Produktion hergestellt werden, da der Rücklauf der papierbasierten Rückmeldungen,erst am Folgetag erfolgen. Eine andere Möglichkeit sind die Rückmeldungen über Terminals in den Werken, da diese meist lange Laufzeiten mit sich bringen, werden sie jeodch in der Regel erst kurz vor Schichtende in gesammelter Form getätigt. 
\end{definitionForm}

\begin{definitionForm}[KA-B-3]
Die Fertigungsdurchführung ist, beispielhaft an seinem früheren Arbeitgeber \textit{Musterberger Druckmaschinen AG}, durch \textbf{hohen Erfassungsaufwand} gekennzeichnet. Durch die manuelle Erfassung auf Papier und der anschließenden Übertragung in das ERP-System erhöht sich der notwendige Aufwand bis die Daten digital vorliegen. Die Rücknahme fehlerhafter Rückmeldungen oder Änderungen an getätigten Rückmeldungen sind durch die papierbasierte Rückmeldung schwer durchführbar. Auch die Rückmeldung an Terminals in den Werken führt zu zusätzlichen Laufzeiten, die Informationen die zurückgemeldet werden müssen, werden in der Zwischenzeit dennoch in Papierform gehalten. 
\end{definitionForm}

\begin{definitionForm}[KA-B-4]
Die Fertigungsdurchführung ist, beispielhaft an seinem früheren Arbeitgeber \textit{Musterberger Druckmaschinen AG}, durch \textbf{mangelhafte Datenqualität} gekennzeichnet. Durch die Abgabe der Rückmeldungen in Papierform kommt es oft zu unvollständigen oder gar fehlerhaften Angaben. \textbf{Fehlende Plausibilitätsprüfungen} erhöhen das Risiko hierfür. Führt der Werker die Rückmeldung ins ERP-System selbst über ein Terminal in den Werken durch, führt das im Normalfall zu keinen Problemen, da er die Daten korrigieren kann. In seiner Rolle als Produktionsleiter bei der \textit{Musterberger Druckmaschinen AG} machte er dabei die Erfahrung, das ca. 5-15\% aller Rückmeldungen fehlerhaft sind, je nach Auftragstyp.
\end{definitionForm}

\begin{definitionForm}[KA-B-5]
Die Erfassung von Fertigungsrückmeldungen in Papierform ist zum jetzigen Zeitpunkt in den meisten ihm bekannten Industriebetrieben \textbf{nicht vollständig oder gar nicht ersetzbar}. Alternative Ansätze bringen entscheidende Vorteile und können zu einer gesteigerten Produktivität führen. Eine vollautomatisierte Rückmeldung der Fertigungsdurchführung ist in der diskreten Fertigung nicht möglich.
\end{definitionForm}

\begin{definitionForm}[KA-B-6]
Die Rückmeldung von Fertigungsaufträgen erfolgt in der Regel auf der Vorgangsebene. Die minimal notwendigen Informationen sind: \textbf{Auftragsnummer, Vorgangsnummer, Gut-, Ausschuss- und Nacharbeitsmenge und Durchlaufszeit.}
Anhänge in Form von Dateien sind eine sinnvolle Ergänzung.
\end{definitionForm}

\begin{definitionForm}[KA-B-7]
Der \textbf{Einsatz von mobilen Geräten} hat ein enormes Potenzial die Fertigungsdurchführung effizienter zu gestalten. Eine Durchführung der Betriebsdatenerfassung über eine Spachsteuerung ist ein Anwendnungsfall der ebenfalls Vorteile in der diskreten Fertigung bringen kann. Bei der \textit{Musterberger Druckmaschinen AG} haben nicht alle Betriebsmittel einen Strichcode, weswegen ein Scanner nicht optimal eingesetzt werden kann.
\end{definitionForm}
\newpage
% **************************************************************************************
% * Prozessexperten                                                               
% **************************************************************************************
\tocless\section{Interview mit einem Prozessexperten}\label{ah:interviewCON}

\spaceparagraph{Experte für das Experteninterview}
\textit{Herr Otto Normalverbraucher \footnote{Der Name \textit{Otto Normalverbraucher} repräsentiert den Namen des Prozessexperten.}} wird für das Experteninterview befragt.
Er ist einer der Experten des Beratungsbereiches \enquote{Manufacturing Industries} der \textit{ERP SE \footnote{Die Firma \textit{ERP SE} repräsentiert den Arbeitgeber des Prozessexperten.}}, welcher Betrieben aus der Automobilindustrie, im Maschinen- und Anlagenbau sowie der Hightech Industrie, bei der Gestaltung branchenspezifischer Prozessabläufe und der Umsetzung mit SAP-Software unterstützt.
In seiner Rolle als \enquote{Senior Business Process Consultant} bewertet er Kundenanforderungen, stimmt sie auf die Geschäftsprozesse ab, konzeptioniert die Abbildung mittels SAP Lösungen in einer Gesamtlösungsarchitektur und sorgt für die Implementierung aus Sicht der Lösungsarchitektur.
Basierend auf den ihm vorliegenden Systemlandschaftsdaten erarbeitet er detaillierte Konzepte durchgängiger, branchenspezifischer Geschäftsprozesse und identifiziert dabei Faktoren, die einen negativen Einfluss auf den Geschäftsprozess haben.
Im Dialog mit den Kunden hilft er diesen, die kritischen Faktoren nachhaltig zu beseitigen. 
 
Diese Tätigkeit übt \textit{Herr Otto Normalverbraucher} seit dem Jahr 1999 (seit dem Jahr 2004 bei der \textit{ERP SE}) aus und besitzt zum Zeitpunkt der Erstellung dieser Bachelorarbeit damit eine Berufserfahrung von 20 Jahren.
Aufgrund dieser besitzt er einen großen Erfahrungsschatz hinsichtlich der Merkmale von etablierten Arbeitsprozessen und Hilfsmitteln zur Fertigungsdurchführung im SAP-Umfeld, welche anhand dieses Experteninterviews greifbar gemacht werden soll.

\spaceparagraph{Kernaussagen des Experteninterviews}

\begin{definitionForm}[KA-P-1]
Die Fertigungsdurchführung ist, beispielhaft an diversen von ihm betreuten Industiebetrieben. durch \textbf{geringe Flexibilität und Geschwindigkeit} gekennzeichnet. Der Auslöser einer Fertigungsdurchführung ist die persönliche Abholung eines Fertigungsauftrags vom Werker beim Fertigungssteuerer. Die Übergabe erfolgt in ausgedruckter Form. Dabei entstehen unnötige Laufwege und Wartezeiten in der Produktion. Bei kurzfristigen Änderungen aufgrund von aufgetretenen Ereignissen, wie einer Veränderung der Mengen im Fertigungsauftrag, besteht keine Möglichkeit dies ohne weitere Latenzzeiten zu erreichen.
\end{definitionForm}

\begin{definitionForm}[KA-P-2]
Die Fertigungsdurchführung ist, beispielhaft an diversen von ihm betreuten Industiebetrieben, durch \textbf{geringe Transparenz} gekennzeichnet. Durch fehlende Einsichten in den Zwischenstand eines Fertigunsauftrags, hat ein Fertigungssteuerer aufgrund mangelnder Informationen kaum Möglichkeiten in Echtzeit auf veränderte Situationen zu reagieren. Folgen können beispielsweise Auslastungslücken oder Verzögerungen in der Produktion sein. Diese Transparenz kann nur aufwändig über persönliche Nachfrage in der Produktion hergestellt werden, da der Rücklauf der papierbasierten Rückmeldungen erst am Folgetag erfolgt. Eine andere Möglichkeit sind die Rückmeldungen über Terminals in den Werken, da diese meist lange Laufzeiten mit sich bringen, werden sie jedoch in der Regel erst kurz vor Schichtende in gesammelter Form getätigt. 
\end{definitionForm}

\begin{definitionForm}[KA-P-3]
Die Fertigungsdurchführung ist, beispielhaft an diversen von ihm betreuten Industiebetrieben, durch \textbf{hohen Erfassungsaufwand} gekennzeichnet. Durch die manuelle Erfassung auf Papier und der anschließenden Übertragung in das ERP-System erhöht sich der notwendige Aufwand bis die Daten digital vorliegen. Die Rücknahme fehlerhafter Rückmeldungen oder Änderungen an getätigten Rückmeldungen sind durch die papierbasierte Rückmeldung schwer durchführbar. Auch die Rückmeldung an Terminals in den Werken führt zu zusätzlichen Laufzeiten. Die Informationen die zurückgemeldet werden müssen, werden in der Zwischenzeit dennoch in Papierform gehalten. 
\end{definitionForm}

\begin{definitionForm}[KA-P-4]
Die Fertigungsdurchführung ist, beispielhaft an diversen von ihm betreuten Industiebetrieben, durch \textbf{mangelhafte Datenqualität} gekennzeichnet. Durch die Abgabe der Rückmeldungen in Papierform kommt es oft zu unvollständigen oder gar fehlerhaften Angaben. \textbf{Fehlende Plausibilitätsprüfungen} erhöhen das Risiko hierfür. Führt der Werker die Rückmeldung ins ERP-System selbst über ein Terminal in den Werken durch, führt das im Normalfall zu keinen Problemen, da er die Daten korrigieren kann. 
\end{definitionForm}

\begin{definitionForm}[KA-P-5]
Die Erfassung von \textbf{Fertigungsrückmeldungen in digitaler Form} wird von diversen von ihm betreuten Industiebetrieben getestet und angestrebt. Die Strategie einer papierlosen Fertigung wird ist in der Praxis noch nicht weit verbreitet, wird aber von vielen Industiebetrieben angestrebt.
\end{definitionForm}

\begin{definitionForm}[KA-P-6]
Die Rückmeldung von Fertigungsaufträgen erfolgt in der Regel auf der Vorgangsebene. Die minimal notwendigen Informationen sind: \textbf{Auftragsnummer, Vorgangsnummer, Gut-, Ausschuss- und Nacharbeitsmenge und Durchlaufszeit.}
Anhänge in Form von Dateien sind eine sinnvolle Ergänzung. Die Meldung von Defekten erfolgt parallel zur Rückmeldung. 
Die Soll-Werte sind als Vorschlag vorgegeben.
\end{definitionForm}

\begin{definitionForm}[KA-P-7]
Die \textbf{Erfassung von Defekten} ist ein wesentlicher Bestandteil der Rückmeldung in der Fertigungsdurchführung. Diese Funktionalität ist im Modul Produktionsplanung- und steuerung von SAP S/4HANA jedoch nicht integriert.
\end{definitionForm}

\begin{definitionForm}[KA-P-8]
Der \textbf{Einsatz von mobilen Geräten} hat ein enormes Potenzial die Fertigungsdurchführung effizienter zu gestalten. Eine Durchführung der Betriebsdatenerfassung über einen Barcode-Scanner oder über Text- bzw. Formularerkunng ist ein Anwendnungsfall der ebenfalls ein sehr hohes Potenzial für der diskreten Fertigung hat.
\end{definitionForm}

\begin{definitionForm}[KA-P-9]
Eine \textbf{Arbeitsplatz-spezifische Vorgangsliste} zur Übersicht der anstehenden Fertigungsaufträge pro Werker, ist eine weit verbreitete kundenspezifische Erweiterung. Diese dient der einfacheren Arbeitseinteilung indem der Werker die Übersicht über die ihm zugewiesenen Aufträge behalten kann und diese selbst in gewünschter Form, digital oder in Papierform, abarbeiten kann.
\end{definitionForm}
\newpage
% **************************************************************************************
% * Technologieexperten                                                               
% **************************************************************************************
\section*{Interview mit einem Technologieexperten}

\paragraph{Experte für das Experteninterview}
\textit{Herr John Doe \footnote{Der Name \textit{John Doe} repräsentiert den Namen des Technologieexperten.}} wird für das Experteninterview befragt.
 Er ist einer der Experten des Entwicklungsbereiches \enquote{S/4HANA Cloud Produce - Manufacturing} der \textit{ERP SE \footnote{Die Firma \textit{ERP SE} repräsentiert den Arbeitgeber des Technologieexperten.}}, in dessen Verantwortungsbereich die Anwendungen und Technologien zur Produktionsplanung und -steuerung der ERP-Software SAP S/4HANA entworfen und entwickelt werden.
 In seiner Rolle als \enquote{Development Expert} eruiert er Anforderungen an die ERP-Software, transformiert bestehende Geschäftsprozesse in Anwendungen, konzeptioniert die technologischen Hilfsmittel und sorgt für die Implementierung aus Sicht der Systemarchitektur. Basierend auf den ihm vorliegenden Geschäftsprozessen erarbeitet er detaillierte Konzepte standardisierter, prozessspezifischer Anwendungen und evaluiert dabei mögliche Technologien, die zur Unterstützung der Geschäftsprozesse fungieren. Im Dialog mit den Kunden, dem Beratungsbereich und dem Produktmanagment werden geeignete Technologien identifiziert, um die Geschäftsprozesse nachhaltig zu verbessern. 
 
Diese Tätigkeit übt \textit{Herr John Doe} seit dem Jahr 2000 bei der \textit{ERP SE} aus und besitzt zum Zeitpunkt der Erstellung dieser Bachelorarbeit damit eine Berufserfahrung von 19 Jahren. Aufgrund dieser besitzt er einen großen Erfahrungsschatz hinsichtlich der Anforderungen an geeignete Technologien der angebotenen SAP-Software zur Fertigungsdurchführung, welche anhand dieses Experteninterviews greifbar gemacht werden soll.

\paragraph{Transkription des Experteninterviews}
\paragraph{Auswertung der Kernaussagen}

