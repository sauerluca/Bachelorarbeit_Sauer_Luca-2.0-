\chapter{Interviewprotokolle}
Die folgenden Ausführungen fassen die Ergebnisse der im Rahmen dieser Bachelorarbeit durchgeführten Experteninterviews zusammen. Die Auswertung erfolgt summarisch und wird nicht wortwörtlich transkribiert. Dieses Vorgehen ist mit dem wissenschaftlichen Betreuer dieser Bachelorarbeit abgestimmt.

\section{Experteninterview aus Anwenderperspektive}
\paragraph{Thema des Experteninterviews}
Das Experteninterview bezieht sich auf die Erkennung kritischer beziehungsweise anomaler Datenbankzustände der Datenbank SAP HANA. Wodurch sich normale Zustände der SAP HANA bestimmten lassen, wird im Rahmen des Experteninterviews ebenfalls eruiert. Im Fokus stehen „CPU Utilization Data“ und „Memory Consumption Data“, deren Charakteristika und optische Muster.

\paragraph{Ziel des Experteninterviews}
Mit dem Experteninterview sollen mathematisch
formulierbare Kriterien erhoben werden, anhand derer Anomalien in „CPU Utilization Data“ und „Memory Consumption Data“ festgemacht werden. Diese sollen einen Kontext von Anomalien der Datenbank SAP HANA beschreiben und als Ausgangslage für die Evaluation zu testender Algorithmen dienen.

\paragraph{Experte für das Experteninterview}
 Für das Experteninterview wird Frau Susanne Glänzer befragt. Sie ist eine der Expertinnen des „Center of Excellence“, welche Kunden beim Monitoring ihrer System- und Datenbankarchitektur unterstützt. Dabei prüft sie die Datenbank SAP HANA anhand verschiedener Systemlandschaftsdaten, wie dem Verlauf der „CPU Utilization“, der „Memory Consumption“, Ausführungszeiten, Column Unloads et cetera, auf die Existenz kritischer Systemzustände. Basierend auf den ihr vorliegenden Systemlandschaftsdaten ergründet sie mögliche Ursachen, die zur Entstehung der kritischen Systemzustände führen. Im Dialog mit den Kunden hilft sie diesen, die kritischen Systemzustände nachhaltig zu beseitigen. Zudem ist sie Product Ownerin des SAP EarlyWatch Alert.
 
Diese Tätigkeit übt Frau Glänzer seit dem Jahr 2010 aus und besitzt zum Zeitpunkt der Erstellung dieser Bachelorarbeit damit eine Berufserfahrung von acht Jahren. Aufgrund dieser besitzt sie einen großen Erfahrungsschatz hinsichtlich der Erkennung von Anomalien in Systemlandschaftsdaten, welcher anhand dieses Experteninterviews greifbar gemacht werden soll.

\paragraph{Transkription des Experteninterviews}
\paragraph{Auswertung der Kernaussagen}


\section{Experteninterview aus Prozessperspektive}

\section{Experteninterview aus Technolgieperspektive}
