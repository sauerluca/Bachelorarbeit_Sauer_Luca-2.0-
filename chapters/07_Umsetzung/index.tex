\chapter{Softwaretechnische Implementierung}\label{ch:Implementierung}
Zur Unterstützung der konzipierten dynamischen Fertigungsdurchführung in der diskreten Industrie wurden im Rahmen dieser Bachelorarbeit die wesentlichen Mindestanforderungen prototypisch implementiert. 
Dabei handelt es sich zum einen um eine softwaretechnische Implementierung des Ereignisverarbeitungsmodells das die Durchführung einer dynamischen Fertigungsdurchführung auf Basis von Ereignisverarbeitung ermöglicht. 
Zum anderen wurde eine zugehörige Benutzerschnittstelle zur Benutzung der dynamischen Fertigungsdurchführung implementiert. Diese beiden Implementierungen werden im anschließenden Abschnitte integriert.

Die SAP Cloud Platform ist ein offenes Platform as a Service-Angebot der SAP SE, auf der neue Anwendungen Form von Webservices gebaut werden und bestehende Anwendungssysteme, wie SAP S/4HANA Cloud,  integriert oder erweitert werden können.
\cite{Schneider.2018}
Zur Erweiterung von SAP S/4HANA Cloud stehen auf der SAP Cloud Platform verschiedene Werkzeuge zur Verfügung, um SAPUI5-Benutzerschnittstellen zu erstellen und um eigene
Anwendungssoftware zu entwickeln, die die Funktionalität von SAP S/4HANA Cloud
erweitern. Die Anwendungssoftware wird in diesem Rahmen häufig in der Programmiersprache Java entwickelt.
\cite{Schneider.2018}

\section{Transformation des Ereignisverarbeitungsmodells in ausführbaren Code}\label{sec:Transformation}

\subsection{Implementierte Komponenten}
\todo{Java Coding hier platzieren}

\subsection{Softwarearchitektur}


\subsection{Benutzung}

\todo{JavaScript Coding hier platzieren}

\section{Implementierung der Benutzerschnittstelle}

\subsection{Softwarearchitektur}

\subsection{Benutzung}
\section{Technische Anbindung von Diensten und Ereignisquellen}

\subsection{Integration mit SAP S/4HANA Cloud}

\subsection{Anbindung von SAP Enterprise Messaging}

\subsection{Erweiterung durch Google Cloud Vision}