\section{Kritische Würdigung}\label{sec:Kritik}
Die vorliegende Bachelorarbeit liefert einen umfassenden Einblick in die Möglichkeiten der Dynamisierung von Geschäftsprozessen mit Hilfe der Ereignisverarbeitung. 
Durch die Anwendung genereller Konzepte und Prinzipien lassen sich derartige Geschäftsprozesse optimieren und teilweise automatisiert ausführen, die mit einer Vielzahl von zur Laufzeit auftretenden Ereignissen aus dem Umfeld des Geschäftsprozesses in Echtzeit interagieren können, sodass sich ihr Ablauf dynamisch auf die jeweils aktuellen Informationen aus den Ereignissen ausrichtet. 

Die vorgestellte Methode aus Kapitel \ref{ch:Methode} unterstützt eine zielführende Vorgehensweise bei der schrittweisen Entwicklung von automatisierten und dynamischen Geschäftsprozessen. Für die praktische Durchführung der einzelnen Aktivitäten des Vorgehens werden Forschungsansätze dargelegt und angewendet und es werden Experten interviewt sowie deren fachliches und methodisches Wissen in den Anforderungen an die zu erledigenden Aufgaben entspricht.
Die überschaubare Fallzahl an Interviews erhebt jedoch keinen Anspruch auf Repräsentativität, sondern stellt eine qualitative Forschungsstudie dar. 

Das entwickelte Geschäftsprozessmodell und die softwaretechnische Implementierung kann die Anwendbarkeit der Dynamisierung von Geschäftsprozessen mit ereignisorientierten Konzepten in einer diskreten Fertigung aufzeigen. Dabei konnten auf fachlicher und technischer Ebene die gestellten Anforderungen hinsichtlich der Automatisierung, der Flexibilität, der Erweiterbarkeit zur Integration in SAP S/4HANA Cloud und der Dynamik bezüglich der Ereignisverarbeitung dargestellt werden. 

Die Herausforderungen lagen in der mangelnden Standardisierung von Verfahrensweisen und der Kombination der gewählten Technologien und Diensten, sodass eine ereignisgesteuerte Architektur in Form einer \ac{SOA} in bestehende Systeme integriert werden kann. 
Gemeinsam mit den umgesetzten Methoden zur Betriebsdatenbeschaffungs- und erfassungs ermöglicht die Ereignisverarbeitung die Demonstration der dynamischen Abbildung der Fertigungsdurchführung auf der logischen Grundlage des Geschäftsprozessmodells.

Resümierend konnte die Eignung der realisierten dynamischen Unterstützung bei der Betreibsdatenerfassung und -beschaffung in ungeplant auftretenden Entscheidungsprozessen mit kurzfristigem Planungshorizont und deren Ausführungen dargestellt werden. Als Hauptnutzen wird daraus eine zeiteffizientere Betreibsdatenerfassung ermöglicht.