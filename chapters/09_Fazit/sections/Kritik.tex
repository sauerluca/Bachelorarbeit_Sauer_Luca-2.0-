\section{Kritische Würdigung}\label{sec:Kritik}

% Die überschaubare Fallzahl an Interviews erhebt keinen Anspruch auf Repräsentativität, sondern stellt eine qualitative Forschungsstudie dar. 

% Mit den Ergebnissen der Experteninterviews sollten Informationen gesammelt, analysiert, in Beziehung gesetzt und verglichen werden, um Handlungsempfehlungen für den Einsatz von bürgerschaftlichem Engagement als Instrument der Kulturförderung durch Stiftungen formulieren und entwickeln zu können. Dafür wurden die Interviews mit Hilfe der qualitativen Inhaltsanalyse ausgewertet.

% Die gesammelten Daten behandeln nicht nur die Grundhaltung der Stiftungen sowohl zu bürgerschaftlichem Engagement und seiner Förderung allgemein, sondern auch deren persönliche Einschätzung zum Einsatz von bürgerschaftlichem Engagement in der Zukunft. Zudem wurden differenzierte Aspekte rund um das Thema befragt, die im weiteren Verlauf näher vorgestellt werden.

% Für die Interviews wurden 17 Experten aus u.a. 14 Stiftungen befragt (siehe Kapitel 1.3.3). Für die vorliegende Arbeit sollten die Experteninterviews dabei nicht nur als Ergänzung und Überprüfung von theoretischem Wissen dienen. Da bürgerschaftliches Engagement im Stiftungsbereich, wie bereits beschrieben, bislang wenig untersucht wurde, dienten die Aussagen der Experten auch dazu, neue Erkenntnisse für den Stiftungsbereich zu gewinnen, indem sie Auskunft über ihr eigenes Handlungsfeld und damit verbundene Eindrücke bzw. Erfahrungen in die Gespräche mit einbrachten. Damit wurden die Experteninterviews zu einer zusätzlichen Datenquelle neben den theoretischen Recherchen.

% Aufgrund dieser angelesenen sowie durch Tagungen und durch eigene Berufserfahrung erlangten Kompetenz konnte die offene Führung der Interviews umgesetzt werden. Die Reihenfolge des Leitfragebogens gab keinen zwingenden Ablauf der Diskussion vor. Die Offenheit in der Führungsstruktur der Interviews stellte sich als Bereicherung dar, da es leitfadengestützte Experteninterviews zulassen, im Interview flexibel auf neue Aspekte zu reagieren. In den Gesprächen konnte somit neben der Groborientierung am Fragebogen eine freie Diskussion und ein bereichernder Gedankenaustausch ermöglicht werden, was sich als hilfreich herausstellte, da aufgrund der Verschiedenartigkeit der befragten Stiftungen jedes Interview einer individuellen Anpassung des Leitfragebogens bedurfte.

% Die Auswahl der Experten für diese Dissertation fiel auf Personen mit unterschiedlichen Positionen in ihrer Stiftung bzw. Institution. Es wurde recherchiert, wer innerhalb der jeweils für den Forschungsschwerpunkt interessanten Stiftung am besten über das Forschungsthema diskutieren kann. 35 Deshalb wurden sowohl Geschäftsführer, als auch Vorstandsmitglieder und Leiter von Projekten oder Gruppen befragt, d.h. Vertreter der Stiftung bzw. Institution, die an verschiedenen Schaltstellen Verantwortung tragen.

% Der Expertenstatus wird hinsichtlich der spezifischen Fragestellung vom Forscher „verliehen“36, da der Experte durch seine besonderen Zuständigkeiten und Aufgaben, seine Erfahrung und Wissen „Insider-Kenntnisse“ zu dem Forschungsgegenstand beisteuern kann. Dadurch erspart die Methode eines Experteninterviews andere aufwendigere Explorationsphasen wie sie z.B. bei einer teilnehmenden Beobachtung oder einer Feldstudie nötig wären. Mit Experteninterviews können Forschungsprojekte kontextspezifisch untersucht werden. Zudem können Experten i.d.R. meist komplikationslos zu einer Interviewteilnahme bewegt werden.

% Gleichzeitig zeigte sich, dass die Interviewpartner bei der Beantwortung der Fragen von den Erfahrungen der Forscherin profitieren konnten, da neue Aspekte und Blickwinkel die Forschungsperspektive erweiterten.

% \begin{attentionForm}
% Laut Meuser und Nagel sind es eben oft nicht Vertreter der „obersten Ebene“ in einer Organisation, die sich als Experten eignen, sondern der Ebenen darunter, weil dort „das detaillierteste Wissen über interne Strukturen und Ereignisse vorhanden ist“. 
% Siehe: Meuser, Michael/Nagel, Ulrike: ExpertInneninterviews – vielfach erprobt, wenig bedacht, in: Bogner, Alexander et al.: Das Experteninterview. Theorie, Methode, Anwendung, Opladen 2009, S. 447. 

% Pfadenhauer, Michaela: Auf gleicher Augenhöhe, in: Bogner, Alexander et al.: Das Experteninterview. Theorie, Methode, Anwendung, Opladen 2009, S. 107.
% \end{attentionForm}

% \textbf{Auswertung durch eine qualitative Inhaltsanalyse nach Philipp Mayring}
\todo{kritik}