\chapter{Durchführung der konzipierten Methode}\label{ch:Durchfuehrung}
Dieses Kapitel behandelt die Durchführung der konzipierten Methode, wofür aufgrund der in Abschnitt \ref{sec:Automatisierung} beschriebenen Vorzüge die BPMN 2.0 verwendet wird.
Im ersten Schritt wird der Geschäftsprozess entworfen, wobei alle erforderlichen Anforderungen bereits berücksichtigt und konkrete Maßnahmen definiert werden. 
Anschließend wird das fachliche Geschäftsprozessmodell in Einheiten aus einer Aktivität oder einem Teilprozess segmentiert, die in einen kausalen Zusammenhang zu Ereignissen gesetzt werden, sogenannten Dynamikeinheiten. 

Der zweite Schritt beinhaltet eine detaillierte Betrachtung der Ereignisverarbeitungsebene, die mittels Echtzeitverarbeitung von zur Laufzeit auftretenden Ereignissen die zugrunde liegende Dynamik für die Geschäftsprozessebene liefert. 
Dazu ist eine Definition des Ereignismodells für die maschinelle Verarbeitung der relevanten Ereignisse zweckmäßig. 
Im Sinne der Softwareentwicklung werden darüber hinaus Richtlinien erarbeitet, anhand derer ein in BPMN 2.0 modelliertes und somit plattformunabhängiges Modell in eine ausführbare und mangels standardisierter Ereignisverarbeitungssprachen plattformspezifische Implementierung überführt werden kann.
Resümierend wird im letzten Schritt der resultierende dynamische Geschäftsprozess modellbasiert dargestellt.

\section{Entwurf auf Geschäftsprozessebene}\label{sec:modellierung}

\missingfigure{Geamt prozess ist nach \cite{Beier.2016}s.34}

\todo{Modellierungs guidelines}

\todo{bpmn elemente}
% \todo{Auf das gewählte Modell in Punkt 4 anpassen}
% Um zu überprüfen, ob die im Rahmen dieser Arbeit angedachte Automatisierung von Testfällen möglich ist wird folgende Hypothese aufstellen: 

% \textit{Vorher manuell ausgeführte Testfälle, welche Blackbox-Verfahren anwenden, lassen sich durch Eagle Drones automatisieren und weisen dabei äquivalente Ergebnisse auf}. 

% Geprüft werden soll diese Hypothese für die im Rahmen dieser Arbeit zur Automatisierung bestimmten Testfälle. Für die Überprüfung der Hypothese wird angenommen, dass ein manueller Testfall automatisierbar ist, wenn er Bestandteil eines automatisierten Tests ist. Gestützt wird dieses Annahme auf das Prinzip der schlüsselwortgetriebenen Automatisierung (siehe Kapitel 2.5.4). Auf dieser Basis soll ein komplexer Testfall entworfen werden, der möglichst viele zu automatisierende Testfälle enthält. Ist die Automatisierung dieses komplexen Testfalls im Rahmen der prototypischen Implementierung möglich, so kann die Hypothese für die inkludierten Testfälle als verifiziert angesehen werden. Gelingt eine Automatisierung nicht, ist die Hypothese zu falsifizieren.

\subsection{Erarbeitung und Vorstellung geeigneter Maßnahmen}

Ein Medienbruch ist ein Wechsel des Mediums, das eine Information trägt. Bei- spielsweise stellt das Abschreiben einer Telefonnummer vom Handy auf einen Notizzettel und das anschließende Übertragen der Nummer von diesem Zettel in eine Datei auf dem PC einen doppelten Medienbruch dar. Da Medienbrüche Zeit kosten, Schnittstellen notwendig machen und Fehler in den übertragenen Informationen verursachen können, sollen Medienbrüche durch eine geeignete Gestaltung der Informationssysteme möglichst vermieden werden.

Die technische Prozessimplementierung geht daher einen Schritt weiter: Die Datenflüsse werden direkt an die Orte der Ausführung geleitet, und Rückmeldungen werden direkt von dort zurück in das führende ERP-System gespeist. Die technische Anbindung kann dabei beliebige Komplexitätsgrade annehmen. Scannerlösungen mit Handhelds, die Wareneingangsbuchungen und Entnahmebuchungen automatisiert per Hallenfunk an der Zentralsystem zurückmelden, sind in den meisten Unternehmen seit vielen Jahren Standard.




Die Sicherstellung der Qualität im Produktionsprozess ist ein weiteres Merk- mal der Lean Production. Fehler müssen bei dieser Philosophie direkt im Fer- tigungsprozess erkannt und auf der Stelle beseitigt werden. Diese Vorgehens- weise wird unter Jidoka zusammengefasst (siehe hierzu auch die Definitionen in Abschnitt 1.3.7).
\todo{medienbruch nach \cite{Schell.2017}s.106 und \cite{Hoppe.2011}}
\todo{Schwachstellenidentifikation nach \cite{Kieviet.2019}{S.53}}
\todo{schwachstellen nach \cite{Gerberich.2011}{S.56}}
\todo{Maßnahmen nach \cite{Schell.2017} s.176}

\todo{Anforderungen in konkrete Maßnahmen umformen in tabellenform}

\subsection{Segmentierung und Bildung von Dynamikeinheiten}

\todo{Bildung der Beziehungen Ereignis <=> Aktivität}

\subsection{Aspekte der Ausführungssemantik}

\missingfigure{ Wesentliche Aspekte der Ausführungssemantik}
\section{Konzeption auf  Ereignisverarbeitungsebene}\label{sec:ereignismodell}
Diesesr Abschnitt behandelt die technische Sichtweise auf die Ereignisverarbeitungsebene, die mittels Echtzeitverarbeitung von zur Laufzeit auftretenden Ereignissen die Dynamik für den Ablauf der Geschäftsprozessebene liefert. Für die digitale Verarbeitung der Ereignisse ist eine explizite Spezifikation des Ereignismodells nötig. Im Sinne einer softwaretechnischen Implementierung wird darüber hinaus der Ereignisverarbeitungsdienst konzipert.

Als Spezifikation einer allgemeinen Beschreibung von Ereignisobjekten dient die \textit{CloudEvents}-Spezifikation in der Version 0.3-wip. 
\textit{CloudEvents} bietet eine standardisierte Definition der Struktur und Metadatenbeschreibung von Ereignissen. Diese Definition definiert, wie die in der \textit{CloudEvents}-Spezifikation definierten Elemente im \ac{JSON}-Format dargestellt werden sollen. \cite{CloudEvents.2019}

Im weiteren Verlauf dieser Bachelorarbeit wird nun auch spezifischer auf eine Umsetzung mit SAP S/4HANA Cloud eingegangen. SAP S/4HANA Cloud bietet für die Behandlung von verschiedenen vordefinierten Geschäftsereignissen eigene Dienste zur Integration an. Der auf der SAP Cloud Platform basierte \textit{SAP Enterprise Messaging-Service} bietet Funktionalitäten für die Integration in SAP S/4HANA Cloud, um Ereignisse zu verbreiten und reaktive Geschäftsprozesse über Unternehmenslandschaften hinweg zu ermöglichen.
\cite{Schneider.2018}

\subsection{Spezifizierung des Ereignismodells}
Ein Ereignisobjekt repräsentiert ein in der Realität eingetretenes Ereignis. Es muss somit alle relevanten Informationen beinhalten, die für eine anschließende Verarbeitung dieses Ereignisses notwendig sind.

Unterschieden wird dabei in verschiedene Ereignistypen. Ein Ereignistyp setzt sich laut der \textit{CloudEvents}-Spezifikation aus dem betroffenen Objekt und der Art der Veränderung zusammen. Einfach ausgedrückt werden durch ein Ereignisobjekt die folgenden vier Fragen beantwortet: \textit{Wann hat es sich ereignet? Was hat sich ereignet? Wer war betroffen?}
Im Kontext dieser Bachelorarbeit wird die Antwort auf die erste Frage als der Zeitpunkt des Ereignissen verstanden. Die zweite Frage zielt auf die Veränderung eines Objektes ab, sie enthält somit die Antwort welcher Objekttyp verändert wurde und wie diese Veränderung aussieht. Die letzte Frage identifiziert schließlich ein eindeutiges Objekt des veränderten Objekttyps. \cite{CloudEvents.2019}

Beim Erstellen eines Ereignisobjektes im Rahmen dieser Bachelorarbeit müssen demnach mindestens die folgenden Daten an den Konstruktor der Ereignisgenerierung übergeben werden:
\begin{itemize}
    \item \textbf{Zeitstempel:} Er beschreibt wann das Ereignis ausgelöst wurde. 
    \item \textbf{Objekttyp:} Er spezifiziert die Klasse des veränderten Objekts. 
    \item \textbf{Ereignistyp:} Er definiert die Art und Weise wie ein Objekt verändert wurde. 
    \item \textbf{Objektschlüssel:} Er identifiziert das betroffene Objekt. 
\end{itemize}

Die SAP-spezifischen Geschäftsobjekte , auch \textit{SAP-Objekttyp} genannt, sind die Basis für Ereignisobjekte innerhalb von SAP S/4HANA Cloud. 
Ein SAP-Objekttyp kann Geschäftsereignisse mit einer eindeutigen Semantik für bestimmte Situationen definieren.
Ein Ereignis kann ausgelöst werden, wenn eine bestimmte Aktion für das Geschäftsobjekt ausgeführt wird oder wenn das Geschäftsobjekt einen bestimmten Status annimmt.
Sobald Ereignisse von Geschäftsobjekten ausgelöst werden, werden diese Ereignisse von SAP S/4HANA Cloud an ein Framework zur Ereignisverarbeitungs weitergegeben. 
\cite{Herzig.2018}
Dieses Framework bildet aus der internen technischen Darstellung des Ereignisses eine Menschen-lesbare Darstellung im \ac{JSON}-Format. Diese Darstellung, wie im Quelltext \ref{code:jsonevent}, wird der \textit{CloudEvents}-Spezifikation gerecht und enthält alle notwendigen Daten.
\cite{Herzig.2018}

\begin{algorithm}[H]
\centering 
\inputminted[linenos]{js}{code/sapevent.json}
\caption{Exemplarisches SAP-spezifisches Ereignisobjekt im JSON-Format \cite{Herzig.2018}}
\label{code:jsonevent}
\end{algorithm}

Um diese Integration zu erreichen, werden zwei Komponenten benötigt: Die allgemeine Integration in die ereignisgesteuerte Architektur von SAP S/4HANA Cloud und eine spezifische Integration mit dem SAP Enterprise Messaging-Service, der auf der SAP Cloud Platform angeboten wird. Beide Komponenten kommunizieren über Standard-Nachrichten- und Verbindungsprotokolle. Anwendungssoftware, die auf der SAP Cloud Platform basiert, kann dabei als Ereigniskonsument fungieren.

\subsection{Architektur der Ereignisverarbeitungsdienstes}
Das Schlüsselelement reaktiver Systeme ist die asynchrone nachrichtengesteuerte Kommunikation zwischen den beteiligten Anwendungssystemen. SAP S/4HANA Cloud setzt hier auf das MQTT-Protokoll. MQTT steht für Message Queuing Telemetry Transport und ist ein standardisiertes Protokoll für die Maschine-to-Maschine-Kommunikation.
Es lassen sich jedoch auch Ereignisverarbeitungsdienste nach den Konzepten des \textit{Publish-Subscribe-Paradigmas} (siehe Abschnitt \ref{sec:Ereignisverarbeitung}) damit umsetzen.
\cite{Herzig.2018}

In SAP S/4HANA Cloud werden zur Integration zwei Dienste angeboten:
\begin{itemize}
    \item Der \textbf{SAP Enterprise Messaging-Service} bietet Ereignisverarbeitungs-Funktionen mit Hilfe der SAP Cloud Platform. Der Diest umfasst die sofortige Integration in SAP S/4HANA Cloud, um Ereignisse zu verbreiten und reaktive Geschäftsprozesse über Anwendungssysteme hinweg zu ermöglichen.
    \item Der \textbf{Business Event Handling-Service} in SAP S/4HANA Cloud, bietet einen standardisierten Weg zur Generierung von Ereignissen auf Basis von Geschäftsobjeken.
\end{itemize}

Geschäftsobjekte sind die Basis für die Ereignisverarbeitung in SAP S/4HANA Cloud. Geschäftsobjekte sind im Kontext von SAP ERP und in der SAP S/4HANA Cloud gut definiert. Ein Geschäftsobjekt ist eine Geschäftsfunktionalität, die eigenständig instanziiert werden kann. Während Geschäftsobjekte fast ständig verwendet werden, werden ihre Typen nur selten explizit berücksichtigt. In Geschäftsanwendungen werden die Typen der Geschäftsobjekte aus dem Kontext, in dem sie angezeigt werden, abgeleitet und sind bekannt, aber im Allgemeinen nicht explizit. Ein gemeinsames Typensystem zwischen dem SAP Enterprise Messaging-Service und SAP S/4HANA Cloud ist daher die Voraussetzung für die generische Interaktion zwischen den Systemen
\cite{Herzig.2018}. 

Dieser kurze Abriss zu den Grundlagen der Ereignisverarbeitung in SAP S/4HANA Cloud  soll an dieser Stelle genügen, während für eine detaillierte Betrachtung der existierenden Dienste auf die genannte Literatur in diesem Umfeld verwiesen sei.\cite{Herzig.2018}

Zur Vorbereitung der Integration des Ereignisverarbeitungsdienstes muss folglich der \textit{SAP Enterprise Messaging-Service} auf der SAP Cloud Platform sowie \textit{Business Event Handling-Service} in SAP S/4HANA Cloud konfiguriert werden, diese Verfahren werden in Abschnitt \ref{sec:integration} noch näher beschrieben. 
\section{Modellbasierte Darstellung des Geschäftsprozesses}\label{sec:bpmnabbildung}
\todo{Grundlangen BPMN 2.0 + neues Modell}