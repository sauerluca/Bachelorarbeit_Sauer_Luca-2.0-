\section{Entwurf auf Geschäftsprozessebene}\label{sec:modellierung}

\missingfigure{Geamt prozess ist nach \cite{Beier.2016}s.34}

\todo{Modellierungs guidelines}

\todo{bpmn elemente}
% \todo{Auf das gewählte Modell in Punkt 4 anpassen}
% Um zu überprüfen, ob die im Rahmen dieser Arbeit angedachte Automatisierung von Testfällen möglich ist wird folgende Hypothese aufstellen: 

% \textit{Vorher manuell ausgeführte Testfälle, welche Blackbox-Verfahren anwenden, lassen sich durch Eagle Drones automatisieren und weisen dabei äquivalente Ergebnisse auf}. 

% Geprüft werden soll diese Hypothese für die im Rahmen dieser Arbeit zur Automatisierung bestimmten Testfälle. Für die Überprüfung der Hypothese wird angenommen, dass ein manueller Testfall automatisierbar ist, wenn er Bestandteil eines automatisierten Tests ist. Gestützt wird dieses Annahme auf das Prinzip der schlüsselwortgetriebenen Automatisierung (siehe Kapitel 2.5.4). Auf dieser Basis soll ein komplexer Testfall entworfen werden, der möglichst viele zu automatisierende Testfälle enthält. Ist die Automatisierung dieses komplexen Testfalls im Rahmen der prototypischen Implementierung möglich, so kann die Hypothese für die inkludierten Testfälle als verifiziert angesehen werden. Gelingt eine Automatisierung nicht, ist die Hypothese zu falsifizieren.

\subsection{Erarbeitung und Vorstellung geeigneter Maßnahmen}

Ein Medienbruch ist ein Wechsel des Mediums, das eine Information trägt. Bei- spielsweise stellt das Abschreiben einer Telefonnummer vom Handy auf einen Notizzettel und das anschließende Übertragen der Nummer von diesem Zettel in eine Datei auf dem PC einen doppelten Medienbruch dar. Da Medienbrüche Zeit kosten, Schnittstellen notwendig machen und Fehler in den übertragenen Informationen verursachen können, sollen Medienbrüche durch eine geeignete Gestaltung der Informationssysteme möglichst vermieden werden.

Die technische Prozessimplementierung geht daher einen Schritt weiter: Die Datenflüsse werden direkt an die Orte der Ausführung geleitet, und Rückmeldungen werden direkt von dort zurück in das führende ERP-System gespeist. Die technische Anbindung kann dabei beliebige Komplexitätsgrade annehmen. Scannerlösungen mit Handhelds, die Wareneingangsbuchungen und Entnahmebuchungen automatisiert per Hallenfunk an der Zentralsystem zurückmelden, sind in den meisten Unternehmen seit vielen Jahren Standard.




Die Sicherstellung der Qualität im Produktionsprozess ist ein weiteres Merk- mal der Lean Production. Fehler müssen bei dieser Philosophie direkt im Fer- tigungsprozess erkannt und auf der Stelle beseitigt werden. Diese Vorgehens- weise wird unter Jidoka zusammengefasst (siehe hierzu auch die Definitionen in Abschnitt 1.3.7).

\todo{Zielsetzung: Nutzenpotential VOB MES nach \cite{Gerberich.2011}{s.77}}
\todo{medienbruch nach \cite{Schell.2017}s.106 und \cite{Hoppe.2011}}
\todo{Schwachstellenidentifikation nach \cite{Kieviet.2019}{S.53}}
\todo{schwachstellen nach \cite{Gerberich.2011}{S.56}}
\todo{Maßnahmen nach \cite{Schell.2017} s.176}

\todo{Anforderungen in konkrete Maßnahmen umformen in tabellenform}

\subsection{Segmentierung und Bildung von Dynamikeinheiten}

\todo{Bildung der Beziehungen Ereignis <=> Aktivität}

\subsection{Aspekte der Ausführungssemantik}

\missingfigure{ Wesentliche Aspekte der Ausführungssemantik}