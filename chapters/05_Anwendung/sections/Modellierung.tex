\section{Modellierung auf Geschäftsprozessebene}\label{sec:modellierung}
% \todo{Auf das gewählte Modell in Punkt 4 anpassen}
% Um zu überprüfen, ob die im Rahmen dieser Arbeit angedachte Automatisierung von Testfällen möglich ist wird folgende Hypothese aufstellen: 

% \textit{Vorher manuell ausgeführte Testfälle, welche Blackbox-Verfahren anwenden, lassen sich durch Eagle Drones automatisieren und weisen dabei äquivalente Ergebnisse auf}. 

% Geprüft werden soll diese Hypothese für die im Rahmen dieser Arbeit zur Automatisierung bestimmten Testfälle. Für die Überprüfung der Hypothese wird angenommen, dass ein manueller Testfall automatisierbar ist, wenn er Bestandteil eines automatisierten Tests ist. Gestützt wird dieses Annahme auf das Prinzip der schlüsselwortgetriebenen Automatisierung (siehe Kapitel 2.5.4). Auf dieser Basis soll ein komplexer Testfall entworfen werden, der möglichst viele zu automatisierende Testfälle enthält. Ist die Automatisierung dieses komplexen Testfalls im Rahmen der prototypischen Implementierung möglich, so kann die Hypothese für die inkludierten Testfälle als verifiziert angesehen werden. Gelingt eine Automatisierung nicht, ist die Hypothese zu falsifizieren.

\subsection{Erarbeitung und Vorstellung geeigneter Maßnahmen}
\todo{Anforderungen in konkrete Maßnahmen umformen}

\subsection{Segmentierung und Bildung von Dynamikeinheiten}

\subsection{Aspekte der Ausführungssemantik}
