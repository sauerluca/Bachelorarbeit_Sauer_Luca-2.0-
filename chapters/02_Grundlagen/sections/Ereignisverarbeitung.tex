\section{Konzept der Ereignisverarbeitung}\label{sec:Ereignisverarbeitung}

\todo{Ereignisse als zentraler Baustein der Softwarearchitektur und der Verarbeitungslogik}

\todo{Ereignisbasierte Systeme definieren}

\todo{Ereignisgesteuerte Architekturen werden durch folgende Besonderheiten}

\todo{Da allerdings kaum reale Geschäftsszenarien existieren, bei denen die Kommunikation ausschließlich mittels Ereignissen und im starren Rahmen der genannten Merkmale erfolgt, wird auch Anwendungen mit teilweiser Ereignissteuerung und partieller Erfüllung dieser Merkmale die Umsetzung einer ereignisgesteuerten Architektur zugeschrieben.}

\todo{Verschmelzung von SOA und EDA wird häufig die Bezeichnung Event-Driven SOA genutzt.}

\todo{Kontext der Ereignisverarbeitung in Unternehmensanwendungen wird häufig der Begriff Echtzeit}

\subsection{Merkmale und Kriterien}

\todo{Architektur Ereignisverarbeitung}

\missingfigure{Logische Strukturierungsschichten von Ereignisverarbeitungssystemen}

\todo{abbildung erkären}

\todo{Formulierung der Ereignisverarbeitungsregeln}

\missingfigure{Kriterien für Funktionalitäten der Ereignisverarbeitung}

\subsection{Bewertung von Funktionalitäten der Ereignisverarbeitung}

\todo{Referenzieren auf \cite{Vidackovic.2010}}

\missingfigure{Tabelle  Bewertung von Funktionalitäten der Ereignisverarbeitung}

