\section{Automatisierung von Geschäftsprozessen}\label{sec:Automatisierung}
Über Geschäftsprozesse lässt sich die Unternehmensstrategie mit den unterstützenden Informations- und Anwendungssystemen verknüpfen.
In einem wirkungsvollen Unternehmensmanagement müssen demnach diese drei Ebenen der Strategie, der Geschäftsprozesse und der IT in ihrer Gesamtheit berücksichtigt werden.
Insbesondere bei der IT-Unterstützung der operativen Geschäftsprozessausführung auf der unteren Ebene ist es vorteilhaft, die Kluft zwischen den realen Geschäftsprozessen und deren informationstechnischen Repräsentation möglichst klein zu halten.

Der Einsatz von Modellierungsverfahren für Geschäftsprozesse ist ein angemessenes Mittel zur Verknüpfung der betriebswirtschaftlichen mit der informationstechnischen Perspektive.
Da diese durch semiformale Notationsweisen sowohl komplexe Sachverhalte der Betriebswirtschaft systematisch unterstützt als auch die notwendige Genauigkeit für den Entwurf von Informationssystemen bietet.
Ein Modell repräsentiert dabei im Allgemeinen eine Repräsentanz eines Ausschnittes der realen Welt, im konkreten Fall die Abbildung eines existenten Geschäftsprozesses, die entweder in Gestalt eines Ist-Modells die derzeitige Situation oder in der Ausprägung eines Soll-Modells eine potenziell angestrebte Möglichkeit darstellt.

Mit der Geschäftsprozessmodellierung werden verschiedene Einsatzzwecke verfolgt.
Dies betrifft zum einen die organisatorische Gestaltung, beispielsweise zur Dokumentation, Reorganisation oder Optimierung von strategischen und operativen Geschäftsprozessen. 
Auf der anderen Seite können, durch das Zusammenfügen der betriebswirtschaftlichen mit der informationstechnischen Perspektive, Anwendungssysteme konstruiert werden, welche für die Automatisierung von Geschäftsprozessen eine entscheidende Rolle spielen.
Um ein Geschäftsprozessmodell letztlich in einem Anwendungssystem abzubilden, existieren verschiedene alternative Ansätze, wie etwa die Ableitung von zugehörigen Anforderungen und anschließende Umsetzung durch prozessorientierte Architekturen, die unternehmensspezifische Anpassung von Standardsoftware oder die direkte Umsetzung in ausführbare Modelle in geeigneten Ausführungsumgebungen.
Im Rahmen dieser Arbeit wird lediglich der erste Ansatz in Betracht gezogen.

\subsection{Merkmale und Kriterien}

Der Einsatz von Modellierungsverfahren bietet die Möglichkeit automatisierbare Geschäftsprozesse in fachlich sowie technisch spezifizierten Geschäftsprozessmodellen formalisiert und detailliert zu erfassen.
Grundsätzlich ist die Umsetzung von Geschäftsprozessen dieser Art durch entsprechende organisatorische aber auch technische Maßnahmen durchzuführen.
Bei der Konzeption derartiger Geschäftsprozessmodelle wird das in Abbildung \ref{fig:Phasenmodell bei der Automatisierung von Geschäftsprozessen} illustrierte Phasenmodell herangezogen.

\begin{figure}[H]
	\centering 
    \begin{tikzpicture}
       \arcarrow{177}{ 96}{Analyse}
       \arcarrow{ 89}{  3}{Konzeption}
       \arcarrow{268}{361}{Implementierung}
       \arcarrow{179}{271}{Ausf{\"u}hrung}
    \end{tikzpicture}
    \caption[Phasenmodell bei der Automatisierung von Geschäftsprozessen]
    {Phasenmodell des Lebenszyklus von Geschäftsprozessen\footnote{in Anlehnung an \author{Scheer.1991} \citeyear{Scheer.1991} \cite{Scheer.1991}} }
    \label{fig:Phasenmodell bei der Automatisierung von Geschäftsprozessen}
\end{figure}

Im ersten Schritt wird eine \ac{IT}-orientierte fachliche Ausgangslösung erstellt.
Diese ergibt sich aus der eingehenden Analyse eines neuen oder existierenden Geschäftsprozesses, in Abbildung \ref{fig:Phasenmodell bei der Automatisierung von Geschäftsprozessen} im linken oberen Bereich veranschaulicht, durch die zunächst die grundsätzlichen Anforderungen des zu untersuchenden Geschäftsprozesses sichtbar gemacht werden. 
Aus diesem Grund werden hier auch noch alle Perspektiven zusammen betrachtet.
In der darauf folgenden Konzeptionsphase wird, auf Basis der zuvor erhobenen Anforderungen an einen Geschäftsprozess, dessen Modellierung auf fachlicher Ebene durchgeführt. Anschließend wird das Fachkonzept, unabhängig von Implementierungsgesichtspunkten, mit technischen Anforderungen an das Anwendungssystem angereichert, sodass es als Ausgangspunkt für eine konsistente softwaretechnische Implementierung dienen kann. 
Dabei wird jedoch noch kein Bezug zu plattformspezifischen Verarbeitungssprachen hergestellt. 
Das technisch spezifizierte Konzept kann geändert werden, ohne dass dies Auswirkungen auf das Fachkonzept hat.
Dies bedeutet jedoch nicht, dass Fachkonzept und technische Spezifikation isoliert voneinander entwickelt werden können. 
Mehr noch soll nach Abschluss der fachkonzeptionellen Darstellung der betriebswirtschaftliche Inhalt so definiert sein, dass ausschließlich \ac{IT}-bezogene Argumente, wie das Leistungsverhalten eines Informationssystems, keine Auswirkungen auf die Fachinhalte nehmen können. 
In der Implementierungsphase wird das Geschäftsprozessmodell in seine finale Ausprägung überführt somit in eine ausführbare Verarbeitungssprache transformiert, getestet und in eine Ausführungsumgebung integriert.
Während der Ausführungsphase erfolgt häufig eine Überwachung des laufenden Geschäftsprozesses, um in einer nachfolgenden Iteration des gesamten Phasenmodells eine Optimierung des Geschäftsprozessmodells basierend auf den hieraus gewonnnen Erkenntnissen durchführen zu können.
Der Fokus dieser Arbeit liegt insbesondere auf den Phasen von der Analyse bis zur softwaretechnischen Implementierung, wohingegen die Ausführung vernachlässigt wird.











Das technisch spezifizierte Geschäftsprozessmodell, das einer formalen Syntax mit eindeutiger Ausführungssemantik folgt, wird häufig auch im deutschen Sprachgebrauch mit dem englischen Begriff Workflow betitelt, der historisch bedingt aus der informationstechnischen Unterstützung menschlicher Arbeitsabläufe stammt. 
Obwohl diese Bezeichnung heutzutage durchaus breiter gefasst wird als in seiner ursprünglichen Bedeutung und insbesondere auch die prozessorientierte Integration unterschiedlicher Anwendungssysteme umfasst, repräsentiert ein Geschäftsprozess als Abgrenzung dazu ein höherwertiges Gut im Unternehmen, mit welchem ein übergeordnetes Geschäftsziel verfolgt wird. 
Aus diesem Grund wird in dieser Arbeit anstatt von einem Workflow weiterhin vom technischen Geschäftsprozessmodell gesprochen, wohl wissend, dass nicht jeder Geschäftsprozess vollständig automatisierbar ist.

Sogenannte Webservices stellen dabei eine auf XML (Extensible Markup Language) ausgerichtete und auf offenen Standards basierende Technologie zur Realisierung einer SOA dar, indem sie die in einer beliebigen Programmiersprache implementierte Anwendungslogik anhand vereinheitlichter Schnittstellen nach außen zur Verfügung stellen. 

Auf diese Weise können Webservices beispielsweise aus einem Geschäftsprozess heraus aufgerufen werden, ohne ihre konkrete Implementierung zu kennen, was insbesondere in einem unternehmensübergreifenden Kontext ein bedeutender Faktor ist. 
Zur Änderung eines Geschäftsprozesses bedarf es dann lediglich einer Anpassung der Kontrollstrukturen mit veränderten Aufrufen der bestehenden oder zusätzlicher Webservices. 
Die Schnittstelle zum Zugriff auf einen Webservice wird dabei mittels der vom W3C (World Wide Web Consortium) standardisierten Web Services Description Language (WSDL) auf XML-Basis spezifiziert.

Ferner können auch von Menschen ausgeführte Geschäftsaktivitäten als Dienste gekapselt und mit IT-Schnittstellen ausgestattet werden, indem die Aufforderung zum Beginn der manuellen Aktivität durch eine geeignete IT-Ausgabe signalisiert wird, woraufhin der erfolgreiche Abschluss der manuellen Aktivität oder ein möglicherweise erforderlicher Abbruch wiederum durch eine IT- Eingabe quittiert wird. 
Auf diese Weise ist auch die Integration manueller Aktivitäten in einen au- tomatisierten Geschäftsprozess möglich. 
Hierfür existiert die Spezifikation Web Services Human Task (WS-HumanTask), die vom unabhängigen Standardisierungsgremium OASIS (Organization for the Advancement of Structured Information Standards) vereinheitlicht wird.

Durch die stetige Verbreitung von dienstorientierten Architekturen, englisch Service-Oriented Architecture (SOA), besteht für Unternehmen die Möglichkeit, Geschäftsprozesse auf Basis modularer elektronischer Dienste in einer lose gekoppelten Weise auch unternehmensübergreifend zu komponieren und zu automatisieren. 
Hierbei wird die wesentliche Geschäftslogik, welcher im Allgemeinen eine lange Lebensdauer beigemessen wird, in elektronischen Diensten gekapselt, während der übergreifende Geschäftsprozess die eher kurzfristig ausgerichteten und demnach veränderlichen Kontrollstrukturen implementiert.

Für die Modellierung und Ausführung von automatisierbaren Geschäftsprozessen ist eine passende Notation erforderlich, die verschiedene Kriterien erfüllen muss.
Bei der Betrachtung dynamischer Geschäftsprozesse im Rahmen dieser Bachelorarbeit spielen dabei auch solche Merkmale eine herausragende Rolle, welche die Ausprägung dynamischer Eigenschaften anhand einer umfassenden Ereignisorientierung und Ereignisverarbeitung begünstigen.
Im Untersuchungsbereich sind verschiedene Kriterien für die Auswahl einer tauglichen Modellierungssprache für das zugrunde liegende Geschäftsprozessmodell von Belang, die in Tabelle XXX jeweils mit einer kurzen Erläuterung ihrer Bedeutung aufgeführt sind.

\todo{Definiton Modellierungsverfahren für automatisierbare Geschäftsprozesse}


\todo{Erklärung des Phasenmodells}

\todo{Service-Oriented Architecture}

\todo{Notwendigkeit einer Notation}

\missingfigure{Kriterien für Modellierungssprachen für dynamische Geschäftsprozesse}

\subsection{Gegenüberstellung geeigneter Modellierungssprachen}
Im Folgenden werden bekannte und weitverbreitete Notationssprachen für die Modellierung und Automatisierung von Geschäftsprozessen herangezogen, die als Basis für dynamische Geschäftsprozesse dienen können. Diese werden zunächst einzeln begreiflich gemacht und anschließend in Bezug auf die in Tabelle \todo{ref} genannten Kriterien in einer tabellarischen Übersicht gegenübergestellt.

\paragraph{EPK}

\paragraph{BPMN 2.0}

\paragraph{UML}

\paragraph{Gegenüberstellung der Modellierungssprachen}

\missingfigure{Bewertung von Modellierungssprachen für dynamische Geschäftsprozesse}