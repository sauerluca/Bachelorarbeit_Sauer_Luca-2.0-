\section{Zusammenfassung der wesentlichen Erkenntnisse und Defizite}\label{sec:grundlagensummary}

Resümierend sind im Wesentlichen folgende Erkenntnisse festzuhalten, die für die betrachtete Problemstellung von Relevanz sind:
\begin{itemize}
	\item 
	Um als Unternehmen die erforderlichen Maßnahmen, damit in Echtzeit agiert werden kann, etablieren zu können, stellt die Umsetzung einer dynamischen Form von Geschäftsprozessmanagement inzwischen einen entscheidenden Faktor dar.
    \item 
    Für die Dynamisierung von Geschäftsprozessen eignet sich insbesondere eine Ausrichtung auf relevante Ereignisse, so dass deren Auftreten zur Laufzeit eine unmittelbare Reaktion im Verlauf eines Geschäftsprozesses auslösen kann.
    \item
    Für das Management von dynamischen Geschäftsprozessen auf Basis von Ereignisverarbeitung sind geeignete Methoden, Vorgehensweisen und Modellierungssprachen erforderlich.
\end{itemize}

Diesen Erkenntnissen kann durch den aktuellen Stand der Wissenschaft und Technik nicht in
reichendem Maße Rechnung getragen werden, da noch maßgebliche Defizite existieren:
\begin{itemize}
	\item
	Geläufige Modellierungssprachen für Geschäftsprozesse unterstützen nicht die Behandlung von Ereignissen zur Laufzeit.
    \item 
    Aktuelle Konzepte und Systeme zur Ereignisverarbeitung weisen keine adäquate Integration mit Geschäftsprozessmodellen auf.
    \item
    Das Gebiet der Ereignisverarbeitung besitzt keine allgemein anerkannten Standards für Ereignismodelle und Ereignisverarbeitungsdienste und lässt ausgereifte Methoden und umfangreiche Praxiserfahrung vermissen.
    \item
    Forschungsansätze mit der Kombination von Geschäftsprozessautomatisierung und Ereignisverarbeitung zur Umsetzung dynamischer Geschäftsprozesse sind zwar vorhanden, allerdings fehlt es insbesondere an einer umfassenden Methode mit fachlich orientierten Modellierungstechniken bis hin zur Integration von Ereignissen.
\end{itemize}

Der Lösungsansatz dieser Bachelorarbeit soll aufgrund der genannten Erkenntnisse eine Methode für dynamische Geschäftsprozesse zur Verfügung stellen, die sich auf die Anreicherung automatisierbarer Geschäftsprozessmodelle mit Konzepten der Ereignisverarbeitung stützt. 
Um die betrachtete Problemstellung bei dynamischen Geschäftsprozessen bewältigen zu können, sind die Defizite im Stand der Wissenschaft und Technik in den relevanten Fachbereichen zu überwinden. Eine Kombination der vorgestellten Forschungsansätze im Rahmen dieser Bachelorarbeit soll dies ermöglichen und somit als Grundlage des Lösungsansatzes dienen.