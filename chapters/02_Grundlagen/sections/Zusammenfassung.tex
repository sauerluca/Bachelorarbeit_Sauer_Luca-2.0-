\section{Zusammenfassung der wesentlichen Erkenntnisse und Defizite}\label{sec:grundlagensummary}

Resümierend sind im Wesentlichen folgende Erkenntnisse festzuhalten, die für die betrachtete Problemstellung von Relevanz sind:
\begin{itemize}
	\item 
	Um die erforderliche Ausgestaltung eines Echtzeitunternehmens in einem turbulenten Geschäftsumfeld etablieren zu können, stellt die Umsetzung einer dynamischen Form von Geschäftsprozessmanagement inzwischen einen entscheidenden Faktor dar.
    \item 
    Zur Dynamisierung von Geschäftsprozessen eignet sich insbesondere eine zunehmende Orientierung an relevanten Ereignissen, sodass deren Auftreten zur Laufzeit eine Anpassung des aktuellen Verlaufs eines Geschäftsprozesses in Echtzeit auslösen kann.
    \item
    Um auch komplexe Beziehungen zwischen Ereignissen im Rahmen von Ereignismustern in Echtzeit berücksichtigen zu können, welche aus Sicht des Geschäftsprozesses in besonderer Weise eine relevante Situation repräsentieren können, stellt die Anreicherung von Geschäftsprozessen mit Konzepten der Ereignisverarbeitung ein zweckmäßiges Mittel dar.
    \item
    Für das Management von dynamischen Geschäftsprozessen auf Basis von Ereignisverarbei- tung sind geeignete Methoden, Modelle und Werkzeuge zur Unterstützung erforderlich.
\end{itemize}

Diesen Erkenntnissen kann durch den aktuellen Stand der Wissenschaft und Technik nicht in
reichendem Maße Rechnung getragen werden, da maßgebliche Defizite vorhanden sind:
\begin{itemize}
	\item
	Geläufige Modellierungssprachen für Geschäftsprozesse unterstützen nicht die Behandlung von Ereignismustern mit komplexen Beziehungen zwischen mehreren Ereignissen.
    \item 
    Aktuelle Konzepte und Systeme zur Ereignisverarbeitung weisen keine adäquate Integration mit Geschäftsprozessmodellen auf.
    \item
    Das noch junge Gebiet der Ereignisverarbeitung besitzt keine allgemein anerkannten Standards für Ereignismodelle und Ereignisverarbeitungssprachen und lässt ausgereifte Methoden und umfangreiche Praxiserfahrung vermissen.
    \item
    Erste Forschungsansätze mit der Kombination von Geschäftsprozessautomatisierung und Ereignisverarbeitung zur Umsetzung dynamischer Geschäftsprozesse sind zwar vorhanden, allerdings fehlt es insbesondere an einer umfassenden Methode mit fachlich orientierten Modellierungstechniken bis hin zur automatischen Transformation in die entsprechende Ausführungssemantik.
    \item
    Bisher ist kein formalisiertes und plattformunabhängiges Ereignisverarbeitungsmodell zur nahtlosen Integration in ein Geschäftsprozessmodell mit gleichzeitiger Behandlung des Ereignismodells verfügbar.
\end{itemize}

\todo{Überarbeiten}