\section{Vorgehen und Aufbau der Bachelorarbeit }\label{sec:Vorgehen}

Das Vorgehen dieser Bachelorarbeit beruht auf der sequenziellen Beachtung der folgenden Interrogativpronomen, die jeweils einen Bereich der Bachelorarbeit charakterisieren: Wozu? Was? Wie? Wohin?

\paragraph{Wozu?}
Das Kapitel \ref{ch:einleitung} umfasst den Zweck, dem die Themenstellung dieser Arbeit entspringt. Nach der Darlegung der Motivation der Bachelorarbeit wird die Problemstellung thematisiert, die sich im Bezug auf die Fertigungsdurhführung mit SAP S/4HANA Cloud bei der Entwicklung dynamischer Geschäftsprozesse auf Basis von Ereignisverarbeitung ergibt. Daraufhin wird die Zielsetzung der Arbeit mit ihrem Betrachtungsfokus formuliert.

\paragraph{Was?}
In diesem Bereich werden die inhaltlichen Bestandteile der Arbeit gekennzeichnet. Dazu wird zunächst im zweiten Kapitel der Stand der Wissenschaft und Technik in den für die Bachelorarbeit relevanten Fachbereichen analysiert und bewertet. Dies betrifft einerseits die Fachgebiete der Geschäftsprozessautomatisierung und der Ereignisverarbeitung sowie andererseits die vorhandenen Forschungsansätze zur Kombination dieser beiden Disziplinen, welche auf ihre Eignung zur Lösung der Problemstellung untersucht werden. Daraufhin werden zentrale Erkenntnisse und Defizite  verzeichnet. Im Anschluss findet im Rahmen des dritten Kapitels die Untersuchung des Anwendungsfalls statt. Es folgt die Identifikation allgemeiner Anforderungen sowie konkreter Anforderungen aus geführten Experteninterviews im Bezug auf die Fertigungsdurchführung, die in dieser Bachelorarbeit zu erfüllen sind.

\paragraph{Wie?} 
Dieser Bereich beinhaltet den Kern der Arbeit und behandelt die Modellierung auf Geschäftsprozessebene, wobei zunächst Maßnahmen in Bezug auf die Anforderungen aus Kapitel \ref{ch:Grundlagen} identifiziert werden, woraus dann Dynamikeinheiten mit Ereigniskonstrukten gebildet werden. Die Betrachtung der Ereignisverarbeitungsebene im fünften Kapitel mit der Konzeption eines Ereignisverarbeitungsmodells stellt einen weiteren Punkt der Bachelorarbeit dar. Das sechste Kapitel behandelt schließlich die softwaretechnische Implementierung im Rahmen dieser Bachelorarbeit. 

\paragraph{Wohin?}
Der abschließende Bereich der Arbeit klärt die Fragestellung, wohin die vorgeschlagene Lösung führen werden kann. Dazu wird der konzipierte  Geschäftsprozess auf seine Eignung bezüglich der gestellten Anforderungen geprüft und bewertet. Anschließend folgt eine kritische Würdigung der Ergebnisse dieser Arbeit sowie ein Ausblick auf mögliche Weiterentwicklungen.

Des Weiteren soll erwähnt sein, dass zur besseren Lesbarkeit in dieser Bachelorarbeit durchgehend das generische Maskulinum genutzt wird. Dies gibt keinerlei Auskunft über das Geschlecht und stellt keine implizierte Geschlechterdiskriminierung des weiblichen Geschlechts dar. Frauen und Männer mögen sich gleichermaßen angesprochen fühlen.