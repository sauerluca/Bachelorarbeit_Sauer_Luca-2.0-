\section{Vorgehen und Aufbau der Bachelorarbeit }\label{sec:Vorgehen}

Das Vorgehen dieser Bachelorarbeit beruht auf der sequenziellen Beachtung der folgenden Interrogativpronomen, die jeweils einen Bereich der Bachelorarbeit charakterisieren: Wozu? Was? Wie? Wohin?

\todo{Fragen beantworten}
\paragraph{Wozu?}

\paragraph{Was?}
tes

\missingfigure{Aufbau der Bachelorarbeit }

\paragraph{Wie?}
Nach Wilde und Hess ( 2007) lässt sich der Forschungsansatz als konzeptionell-deduktive Analyse klassifizieren.


\paragraph{Wohin?}

Des Weiteren soll erwähnt sein, dass zur besseren Lesbarkeit in dieser Bachelorarbeit durchgehend das generische Maskulinum genutzt wird. Dies gibt keinerlei Auskunft über das Geschlecht und stellt keine implizierte Geschlechterdiskriminierung des weiblichen Geschlechts dar. Frauen und Männer mögen sich gleichermaßen angesprochen fühlen.
