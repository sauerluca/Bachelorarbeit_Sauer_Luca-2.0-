\section{Zielsetzung und Fokus der Betrachtungen}\label{sec:Zielsetzung}

% Um der betrachteten Problemstellung bei dynamischen Geschäftsprozessen angemessen begegnen zu können, intendiert die vorliegende Arbeit die Entwicklung einer zumindest teilweisen IT-Unterstützung zur Ereignisverarbeitung und Automatisierung des in erster Linie manuell ausgeführten Geschäftsprozesses zur Fertigungsrückmeldung in \textit{SAP S/4HANA Cloud for Manufacturing} basierend auf Konzepten der Ereignisverarbeitung.

% Der Fokus der Betrachtungen liegt dabei insbesondere auf der Handhabung der Ereignisverarbeitung innerhalb von \textit{SAP S/4HANA Cloud}. Diese soll auf der einen Seite den Entwurf einer Geschäftsprozessmodells umfassen, das sich nahtlos für den gewählten Geschäftsprozess integrieren lässt.
% Auf der anderen Seite soll es die  Transformation des Ereignisverarbeitungsmodells in ausführbaren Code beinhalten, um eine technische Anbindung zu bestehenden Diensten und Ereignisvorrat von \textit{SAP S/4HANA Cloud for Manufacturing} gewährleisten zu können.

\todo{Klare Ziel-Definition}

\todo{Fokus der Betrachtungen}

\todo{Zur Demonstration des praktischen Einsatzes der Methode wird eine softwaretechnische Implementierung in Form eines Prototyps umgesetzt}

\todo{Abgrenzung von MES, FOkus liegt auf ausschließlich auf SAP S/4HANA Cloud}