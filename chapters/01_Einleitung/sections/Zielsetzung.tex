\section{Zielsetzung und Fokus der Betrachtungen}\label{sec:Zielsetzung}
Um der betrachteten Problemstellung der dynamischen Geschäftsprozessen angemessen begegnen zu können, intendiert die vorliegende Arbeit die Entwicklung eines dynamischen Geschäftsprozesses auf Basis von Ereignisverarbeitung zur Dynamisierung der Fertigungsdurchführung mit \textit{SAP S/4HANA Cloud}. Neben der Modellierung eines dynamischen Geschäftsprozesses zur Fertigungsrückmeldung behandelt die Arbeit vornehmlich dessen softwaretechnische Implementierung als Erweiterung von SAP S/4HANA Cloud.

% Aus diesen Überlegungen heraus ergeben sich für die Arbeit die zwei folgenden zentralen Fragestellungen:
 
% \begin{itemize}
%   \item Wie können die Erkenntnisse aus Wissenschaft und Technik zur Dynamisierung der Fertigungsdurchführung mit \textit{SAP S/4HANA Cloud} beitragen?
%   \item Welche technischen Möglichkeiten zur Integration von ereignisverabreitenden Konzepten stehen zum Stand der Arbeit im Kontext von S/4HANA Cloud zur Verfügung und wie lassen sich diese anwenden?
% \end{itemize}

Daraus leitet sich das Ziel dieser Bachelorarbeit ab, einen dynamischen Geschäftsprozess zu entwerfen, welches den Werker und Fertigungssteuerer bei der Verrichtung ihrer Tätigkeiten durch die Bereitstellung der Betriebsdaten in Echtzeit innerhalb einer diskreten Fertigung unterstützt. Als Ergebnis wird eine Verminderung nicht wertschöpfender Aktivitäten in der Fertigungsdurchführung angestrebt, die im Zusammehang mit der Betriebsdatenerfassung auftreten. Neben der eigentlichen Suche sind dies Laufwege, die Benachrichtigung und die Abstimmung mit anderen Personen.

Es wird erwartet, dass die Ergebnisse der Bachelorarbeit zeigen, dass der Einsatz von dynamischen Geschäftsprozessen auf Basis von Ereignisverarbeitung, im Gegensatz zu klassischen, ablauforientierten Geschäftsprozessen, einen nutzenbringenden Beitrag im Bezug auf das Management von Betriebsdaten in der Fertigungsdurchführung leisten kann. Zur Erprobung des Geschäftsprozesses ist eine softwaretechnische Implementierung, welche die Durchführbarkeit einer solchen Dynamisierung belegen soll, ebenfalls Bestandteil dieser Arbeit.
