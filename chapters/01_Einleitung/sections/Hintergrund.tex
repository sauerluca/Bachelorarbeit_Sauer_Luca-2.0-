\section{Hintergrund und Motivation}\label{sec:Hintergrund und Motivation}
Durch den steigenden Marktdruck im Zuge der Globalisierung der Industrie wächst der Bedarf der Unternehmen, auf Ereignisse in  in Echtzeit zu reagieren, um adäquate Entscheidungen treffen zu können. 
In einem derart turbulenten Geschäftsumfeld hängt der Erfolg eines Unternehmens entscheidend von der Fähigkeit ab, Veränderungen zu berücksichtigen und mit sich schnell verändernden Situationen angemessen umzugehen.
\cite{Kuhlin.2005}
Neben Qualität und Kosten werden Zeit und Innovationsfähigkeit zunehmend zu entscheidenden Erfolgsfaktoren im globalen Wettbewerb. Um den aktuellen Herausforderungen gerecht zu werden, tendieren viele Unternehmen zur Konzentration auf ihre Kernkompetenzen.
\cite{Grauer.2019}

Im Jahre 2002 hat das Marktforschungsunternehmen Gartner den Begriff des \textit{Echtzeitunternehmens} geprägt und damit die Fähigkeit zu sofortigem Handeln auf relevante Ereignisse verbunden. 
Ein \textit{Echtzeitunternehmen} kann zur Sicherung und Steigerung seiner Wettbewerbsfähigkeit zu einem beliebigen Zeitpunkt aktuelle Informationen verarbeiten und transparent über Geschäftsprozesse hinweg verfügbar machen. Entscheidungen werden unmittelbar getroffen, um Latenzen im Ablauf der Geschäftsprozesse zu vermeiden.  
\cite{Bruns.2015}
Im Vordergrund stehen dabei die Aspekte: \textit{Automatisierung von Abläufen ohne menschlichen Eingriff, situative Entscheidungsunterstützung durch transparente und aktuelle Informationen sowie direkte Vernetzung, indem wertschöpfende Geschäftsprozesse und nötige begleitende Prozesse miteinander verschmelzen.}
\cite{Grauer.2019}

Ein dynamischer Geschäftsprozess wird in dieser Arbeit als automatisierbarer Geschäftsprozess verstanden, dessen Verlauf vom Eintritt relevanter Ereignisse abhängt und der unmittelbar danach angemessen reagiert.
\cite{Vidackovic.2014}
In diesem Verständnis ist ein dynamischer Geschäftsprozess also ein ereignisgetriebener Geschäftsprozess im Gegensatz zu prozessorientierten und damit eher statischen Geschäftsprozessen, die immer wieder auf die gleiche Weise ausgeführt werden. Ein Ereignis bezieht sich in diesem Zusammenhang auf alles, was als relevantes Ereignis geschieht oder betrachtet wird, wobei ein Ereignis im Sinne der Informationstechnologie als Objekt im Sinne der maschinellen Verarbeitung verstanden wird.
\cite{Krumeich.}

Eine Dynamisierung von Geschäftsprozessen mittels Konzepten der Ereignisverarbeitung ist daher ein geeigneter Beitrag zum notwendigen Echtzeit-Management bei der Erfassung von Betriebsdaten. Die Konzepte der Ereignisverarbeitung wurden bisher nur in allgemeiner Form im Rahmen verschiedener Forschungsansätze angewendet, ohne einem konkreten Anwendungsfall zu dienen.
\cite{Krumeich.}

Im Kontext produzierender Industriebetriebe geht es dabei letztlich um die Steigerung der Effizienz, also die Verbesserung von Durchlaufzeiten, Qualität und Auslastung. 
Jedoch mangelt es im produzierenden Gewerbe häufig an der Transparenz der wertschöpfenden Geschäftsprozesse. 
Eine zeitnahe Rückkopplung zwischen der Produktionsplanung uns -steuerung und der tatsächlich erbrachten Leistung bei der Ausführung von wertschöpfenden Geschäftsprozessen war bisher unzureichend.
\cite{Grauer.2010}




