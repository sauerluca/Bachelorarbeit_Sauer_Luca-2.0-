\section{Hintergrund und Motivation}\label{sec:Hintergrund und Motivation}

% Bereits der deutsche Schriftsteller und Experimentalphysiker Georg Christoph
% Lichtenberg (1742 -1799) sagte:
% "Ich weiß nicht, ob es besser wird, wenn es anders wird. Aber es muss anders
% werden, wenn es besser werden soll. "
% Auf die Wahrheit dieser Aussage bauen auch die Unternehmen, die seit 2008 mit der
% Wirtschaftskrise zu kämpfen haben. Angefangen als Immobilienkrise 

% Exzellenz in der Produktion basiert auf der Fähigkeit, Mensch, Technik und Organisation optimal miteinander zu verbinden.

\todo{Durch hohe Wettbewerbsdynamik => Konzentration auf Kernkompetenzen, bedeutet für diskrete Fertigung konzentration auf Fertigung}

\todo{Integration, die Automatisierung und die Individualisierung}

\todo{geschäftsprozessmanagement}

\todo{abgrenzung informationssystem anwendungssystem}

\missingfigure{Serviceorientierung bcuh sap Abbiludung}

% Unternehmen stehen unter einem hohen Änderungsdruck. Die Technologie und die Techniken zur Implementierung von Anwendungssystemen besitzen eine hohe Änderungsrate und Innovationsgeschwindigkeit. Moderne Anwendungssystem-Architekturen basieren häufig auf Microservices mit der Laufzeitumgebung in der Cloud. Die (Formal-) Ziele von Anwendungssystemen betreffen die Unterstützung von Nutzern mit mobilen Endgeräten sowie die durchgängige Digitalisierung von Geschäftsprozessen. Dabei wird unter Digitalisierung häufig die (Teil-) Automatisierung von Geschäftsprozessen und die Integration von zugehörigen Anwendungssystemen verstanden. \cite{Popp.2016}

\todo{Notwendigkeit zur Dynamisierung von Geschäftsprozessen}
