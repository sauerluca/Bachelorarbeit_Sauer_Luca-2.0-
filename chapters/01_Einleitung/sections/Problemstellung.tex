\section{Problemstellung}\label{sec:Problemstellung}
Einen exemplarischen Anwendungsfall für eine derartige Problemstellung mit dynamischen Geschäftsprozessen verkörpert die Fertigungsdurchführung in der Produktionsplanung und -steuerung mit SAP S/4HANA Cloud. Ein Geschäftsprozess der Produktionsplanung und -steuerung muss zusätzlich den hybriden Ansatz der Vernetzung von Mensch und Maschine berücksichtigen.
\cite{Schell.2017}
Ausgehend von der Ausgangssituation muss sich die Fertigungsdurchführung in der diskreten Fertigung heute schnell an Veränderungen anpassen können. Die Verfügbarkeit von aktueller Information bei zu treffenden Entscheidungen, die einen Bezug auf den Ablauf umgebender Geschäftsprozesse besitzen, ist ein wesentlicher Faktor, um Veränderungen schneller ausführen zu können und Verschwendungen infolge fehlender Information zu eliminieren.
\cite{Westkamper.2006}

Der Mehrwert in der Integration von Ereignisverarbeitung und Automation in die Fertigungsdurchführung besteht also darin, die Erfassung der aktuellen Betriebsdaten in Echtzeit zu ermöglichen, um durch zielgerichtete, vorausschauende Entscheidungen Latenzen oder sonstige Arten von Verschwendungen zu vermeiden.  


