\section{Problemstellung}

Unter einem dynamischen Geschäftsprozess wird im Rahmen dieser Arbeit ein automatisierbarer Geschäftsprozess verstanden, dessen Ablauf vom Auftreten relevanter Ereignisse abhängig ist und der in unmittelbarer Folge darauf in geeigneter Weise reagiert. 

Ein dynamischer Geschäftsprozess ist in diesem Verständnis demnach ein ereignisgesteuerter Geschäftsprozess im Gegensatz zu ablauforientierten und somit eher statischen Geschäftsprozessen, welche wiederholt in derselben Weise ausgeführt werden. Ein Ereignis bezeichnet in diesem Zusammenhang alles, was geschieht oder was als relevantes Geschehnis angesehen wird, wobei ein Ereignis im informationstechnischen Sinn als Objekt zum Zwecke der maschinellen Verarbeitung verstanden wird.

Eine Dynamisierung von Geschäftsprozessen mittels Konzepten der Ereignisverarbeitung ist demnach ein angemessener Beitrag für das erforderliche Echtzeitmanagement bei der Erfassung von Betriebsdaten.

Einen exemplarischen Anwendungsfall für eine derartige Problemstellung mit dynamischen Geschäftsprozessen verkörpert die Fertigungsrückmeldung im Rahmen der Betriebsdatenerfassung in \textit{SAP S/4HANA Cloud for Manufacturing}. Die Ausführung dieses Produktionsbereich-übergreifenenden Geschäftsprozesses findet in einem dynamischen Geschäftsumfeld mit (zum Teil unvorhersehbaren) Ereignissen statt, die im weiteren Verlauf des Geschäftsprozesses unmittelbar berücksichtigt werden müssen. Dazu gehören beispielsweise Verspätungen in der Produktion, Änderungen am Fertigungsauftrags oder Materialengpässe.  

Bisher ist keine Funktionalität zur
ereignisgesteuerten Betriebsdatenerfassung im  Geschäftsprozess der Fertigungsrückmeldung in \textit{SAP S/4HANA Cloud for Manufacturing} verfügbar.