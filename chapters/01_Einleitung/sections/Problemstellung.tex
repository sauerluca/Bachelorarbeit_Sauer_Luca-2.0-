\section{Problemstellung}\label{sec:Problemstellung}


\todo{Verständins von dynamischen Geschäftsprozessen in der Bachelorarbeit }



\todo{Definition Ereignis}

\todo{Dynamisierung von Geschäftsprozessen mittels Konzepten der Ereignisverarbeitung}

% omplexe Geschäftsprozesse sind nicht vollständig vor ihrer Durchführung planbar. Unvorhersehbare Ereignisse erfordern, dass von der in einem
% Prozessmodell festgehaltenen Planung während der Prozessdurchführung abgewichen wird. Diese Eigenschaft von Geschäfts- und Entwicklungsprozessen
% wird als Dynamik bezeichnet.

\todo{exemplarischen Anwendungsfall für eine derartige Problemstellung mit dynamischen Geschäftsprozessen verkörpert die Fertigungsdurchführung in der Produktionsplanung und -steuerung}

\todo{Verschwendung in der Fertigung Erklären}

\missingfigure{Analyse der Anwendbarkeit von 
Internetanwendungen in  
betriebsübergreifenden Produktionsprozessen  s.2}

% Die in dieser Arbeit zentrale Prozesseigenschaft ist die Dynamik von Prozessen. Dynamik wird dabei im engeren Sinne aufgefasst. Sie bezeichnet nicht
% die in jedem Prozess trivialerweise vorhandene Dynamik, die daher rührt,
% dass ein Prozess immer einen Veränderungsvorgang bezeichnet. Stattdessen bezieht sich Dynamik hier auf die Planbarkeit von Prozessen. Ein dynamischer
% Prozess ist nur schlecht planbar. Antonym zur Dynamik wird in Bezug auf
% Prozesse der Begriff Statik verwendet.
% Die Planbarkeit von Prozessen wird dadurch bestimmt, inwieweit die für
% den Prozess relevanten Eingaben bekannt sind. Zu den Eingaben gehören
% Vorgaben, die besagenen Ergebnisse EIngabegrößen 

% http://www.se-rwth.de/phdtheses/Diss-Woerzberger-Management-dynamischer-Geschaeftsprozesse-auf-Basis-statischer-Prozessmanagementsysteme.pdf



% Unter einem dynamischen Geschäftsprozess wird im Rahmen dieser Arbeit ein automatisierbarer Geschäftsprozess verstanden, dessen Ablauf vom Auftreten relevanter Ereignisse abhängig ist und der in unmittelbarer Folge darauf in geeigneter Weise reagiert. 

% Ein dynamischer Geschäftsprozess ist in diesem Verständnis demnach ein ereignisgesteuerter Geschäftsprozess im Gegensatz zu ablauforientierten und somit eher statischen Geschäftsprozessen, welche wiederholt in derselben Weise ausgeführt werden. Ein Ereignis bezeichnet in diesem Zusammenhang alles, was geschieht oder was als relevantes Geschehnis angesehen wird, wobei ein Ereignis im informationstechnischen Sinn als Objekt zum Zwecke der maschinellen Verarbeitung verstanden wird.

% Eine Dynamisierung von Geschäftsprozessen mittels Konzepten der Ereignisverarbeitung ist demnach ein angemessener Beitrag für das erforderliche Echtzeitmanagement bei der Erfassung von Betriebsdaten.

% Einen exemplarischen Anwendungsfall für eine derartige Problemstellung mit dynamischen Geschäftsprozessen verkörpert die Fertigungsrückmeldung im Rahmen der Betriebsdatenerfassung in \textit{SAP S/4HANA Cloud for Manufacturing}. Die Ausführung dieses Produktionsbereich-übergreifenden Geschäftsprozesses findet in einem dynamischen Geschäftsumfeld mit (zum Teil unvorhersehbaren) Ereignissen statt, die im weiteren Verlauf des Geschäftsprozesses unmittelbar berücksichtigt werden müssen. Dazu gehören beispielsweise Verspätungen in der Produktion, Änderungen am Fertigungsauftrags oder Materialengpässe.  

% Bisher ist keine Funktionalität zur
% ereignisgesteuerten Betriebsdatenerfassung im  Geschäftsprozess der Fertigungsrückmeldung in \textit{SAP S/4HANA Cloud for Manufacturing} verfügbar.


% \cite{Eckert.2011}
% Große Unternehmen werden durch monolithische, technologisch veraltete Anwendungssysteme mit oft mehrfach redundant implementierter Funktionalität vor zunehmende Herausforderungen gestellt. Gesteigerte geschäftliche Anforderungen an die Flexibilität der Anwendungen können nicht mehr erfüllt werden. Eine Transparenz hinsichtlich der IT-Unterstützung der Geschäftsprozesse in den Fachbereichen ist häufig nicht gegeben. 

% Der Ansatz der Serviceorientierung verspricht hier Abhilfe durch die Bereitstellung standardisierter, wiederverwendbarer Services mit wohldefinierten Schnittstellen, die oftmals traditionellen Integrationsansätzen überlegen sind. Auf der anderen Seite ist für mittlere und große Unternehmen die Komplexität einer umfassenden SOA-Transformation nicht ohne eine übergreifende Aufteilung von Verantwortlichkeiten, die Zuordnung von Services zu Systemen, die Beschreibung der zu erwartenden Steigerung der Geschäftsfähigkeit sowie die Planung und Umsetzungsverfolgung der dazugehörigen Maßnahmen zu stemmen. Dies unterstützt das Unternehmensarchitekturmanagement (UAM).