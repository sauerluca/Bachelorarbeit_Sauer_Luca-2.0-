\chapter{Bezugsrahmen und Anforderungen}
Der Bezugsrahmen dieser Arbeit soll einen exemplarischen Anwendungsfall zur Verfügung stellen, der sich für eine Dynamisierung von Geschäftsprozessen mittels Konzepten der Ereignisverarbeitung eignet. Hierbei handelt es sich um die ablauforientierte und somit eher statische Fertigungsdurchführung in der diskreten Fertigung, welche im Zuge ihrer \ac{IT}-gestützten Geschäftsprozessautomatisierung auf die Echtzeitverarbeitung verschiedenster Ereignisse angewiesen ist. Um die betrachtete Problemstellung bei dynamischen Geschäftsprozessen bewältigen zu können, werden im Rahmen der Arbeit mithilfe von Experteninterviews Merkmale und Defizite im Geschäftsprozess der Fertigungsdurchführung empirisch erhoben, sodass aus den korrespondierenden Informationen Kriterien für die Dynamisierung des Geschäftsprozesses mittels Konzepten der Ereignisverarbeitung aufgestellt werden können. Anschließend werden Anforderungen sowohl auf Basis der im vorherigen Kapitel gewonnenen Erkenntnisse als auch aus den durchgeführten Experteninterviews sowie der Problemstellung heraus sondiert, um eine objektive Bewertung des konzipierten Geschäftsprozesses zu ermöglichen.
\section{Management von Betriebsdaten im Industriebetrieb}

\todo{Aufgaben im Industriebetrieb}

\missingfigure{Wertschöpfungskette}

\todo{}

\subsection{Geschäftsprozesse in der Produktionsplanung und -steuerung}

\subsection{Vorstellung der Fertigungsdurchführung in der diskreten Fertigung}

\section{Anforderungen}

\subsection{Methodisches Vorgehen zum Experteninterview}

\subsection{Charakteristika und Defizite der Fertigungsdurchführung}

\subsection{Ableitung der Anforderungen an die Fertigungsdurchführung}
\section{Anforderungen}\label{sec:anforderungen}

\todo{Welche Aufgaben müssen erledigt werden? tabellarisch lösen mit input outpu und zeitpunkt der ausführung (start, ende, jederzeit, nach xy)}

\todo{Freigabe mit Papieren digital aber auch analog}

\todo{}





\todo{Weclhe Anforderungen liegen in der Fertigungsdurchführung vor}

\todo{Anforderungen mit Bezug zur Software}

\todo{Anforderungen mit Bezug zur Problemstellung}