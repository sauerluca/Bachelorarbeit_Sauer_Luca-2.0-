\section{Untersuchung der Fertigungsdurchführung}\label{sec:untersuchung}

\todo{Aufgaben nach \cite{Seidlmeier.2015} fallsstudie}

\todo{Einleitung der Methodik der Experteninterviews}
Aus der Einleitung ist bekannt, dass in dieser Bachelorarbeit speziell im Geschäftsprozess der Fertigungsrückmeldung nach automatisierbaren Aktivitäten gesucht werden soll. 
Nachdem in Kapitel \ref{ch:Grundlagen} die Automatisierung von Geschäftsprozessen mittels Ereignisverarbeitung allgemein betrachtet wurden, wird die Fertigungsdurchführung als exemplarischer Untersuchungsgegenstand dieser Bachelorarbeit nun im Kontext der Geschäftsprozessanalyse betrachtet. 

\subsection{Methodisches Vorgehen zum Experteninterview}

\todo{Methodisches Vorgehen zum Experteninterview}
Mithilfe von Experteninterviews sollen alle als Standard geltenden Aufgaben im entsprechenden Geschäftsprozess empirisch erhoben werden, sodass aus den korrespondierenden Informationen Kriterien für die zu findenden Maßnahmen zur Automatisierung aufgestellt werden können. 
Da Experteninterviews in dieser Bachelorarbeit keineswegs die einzige Methode darstellen, sondern lediglich zusätzliches Wissen liefern, ist eine ausgewählte Stichprobe der für das Thema relevanten Perspektiven im vorliegenden Fall ausreichend.

Das Wissen darüber, welche Aufgaben in der Fertigungsdurchführung allgemein sowie mit SAP S/4HANA Cloud als unabkömmlich gelten und welche Merkmale und Defizite diese aufweisen, ist extrem unternehmensspezifisch und kann daher ausschließlich durch ausreichend Erfahrung generalisiert werden. 
Zugleich ist es in einschlägiger Literatur nicht auffindbar.
Die ausgewählten Experten sollten daher Repräsentanten verschiedener Perspektiven auf den Geschäftsprozess darstellen, sodass die Ergebnisse möglichst generalisierbar sind. 
Bei der Auswahl der Experteninterviews wurden folgende Perspektiven beachtet:

\begin{itemize}
    \item 
    \textbf{Branchenexperte}: Dieser vertritt die Rolle des externen Stakeholders. Er hat ein eigenes Interesse am Ablauf des Geschäftsprozesses und war selbst schon in der Planung und Durchführung des Geschäftsprozesses beteiligt. 
    \item
    \textbf{Prozessexperte}: Als Prozessexperte wird im Kontext dieser Bachelorarbeit eine Person verstanden, zu deren Kerngebieten der ausgewählte Geschäftsprozess gezählt wird. Er arbeitet zusammen mit Kunden im SAP-Umfeld an der Einführung und Optimierung des Geschäftsprozesses
    \item
    \textbf{Technologieexperte}:  Ein Technologieexperte beschäftigt sich eingehend mit den technologischen Hilfsmitteln des Geschäftsprozesses, sowie mit deren Verwendung und Implementierung in SAP S/4HANA Cloud.
\end{itemize}

Bei empirischen Untersuchungen ist des Weitere  zwischen qualitativen und quantitativen Untersuchungen zu unterscheiden. Während qualitative Verfahren die Gemeinsamkeiten von mehreren Gegebenheiten untersuchen indem existierende Unterschiede überwunden werden, erfasst die quantitative Sozialforschung Unterschiede auf Vergleichsbasis von Gemeinsamkeiten. Für die empirische Untersuchung der Fertigungsdurchführung auf Möglichkeiten zur Integration von Betriebswirtschaft und Informationstechnologie wird die qualitative Methode gewählt. 

Damit alle Informationen ordnungsgemäß und nachvollziehbar erhoben werden und alle Erkenntnisse wissenschaftlich fundiert sind, werden die Experteninterviews nach dem untenstehenden Vorgehen strukturiert durchgeführt.

\begin{enumerate}
    \item 
    \textbf{Identifikation eines Experten}: Nur eine Person, die ausreichend Erfahrung in dem Gebiet der Fertigungsdurchführung gesammelt hat, ist in der Lage, Merkmale und Defizite in den Aktivitäten dieses Geschäftsprozesses zu erkennen und zu benennen. Es werden also Personen benötigt, die spezifisches Rollenwissen besitzen und somit als kompetent und erfahren angesehen wird.
    \item
    \textbf{Entwicklung eines Fragebogens}: Die Güte und Verwendbarkeit eines Experteninterviews hängt maßgeblich von dem Fragebogen ab. Die Fragen für diesen werden deswegen mit großer Sorgfalt gesammelt, woraufhin sie von dem Autor dieser Bachelorarbeit präzise geprüft und sortiert werden. Die Fragen werden so formuliert, dass sie bewusst sehr präzise Antworten provozieren.
    \item
    \textbf{Formung und Vorstrukturierung des Interviewablaufs}: Zu Beginn der Experteninterviews werden formale Informationen akquiriert und Wünsche der Experten entgegengenommen. 
    Für die Experteninterviews wird jeweils eine Zeitspanne von 30 bis 60 Minuten angesetzt. Als Interviewer fungiert der Autor dieser Bachelorarbeit.
    \item
    \textbf{Gestaltung der an dem Interview beteiligten Rollen}: Zwischen den Experten und dem Interviewer liegt augrund der Erfahrung und der Vorkenntnisse eine wissensbezogene Disparität zugunsten der Experten vor. Um also möglichst viele Informationen von der Expertin zu erhalten, gewährt der Interviewer den Experten einen maßgeblich größeren Teil der gesamten Redezeit.
    \item
   \textbf{ Durchführung des Experteninterviews}: Die Experteninterviews wird werden online via Skype durchgeführt. Nachdem eine Frage gestellt wird, erhalten die Experten eine variable Antwortzeit. Gemäß des Fragenkatalogs geben Experten die gewünschten Informationen an den Interviewer weiter.
    \item
    \textbf{Nachbereitung und Transkription des Experteninterviews}: Die Notizen aus dem Experteninterview werden inhaltlich, sprachlich und strukturell aufbereitet. Die Transkription erfolgt summarisch. Die summarischen Transkriptionen sind im Anhang \ref{ah:protokolle} abgelegt.
    
\end{enumerate}
\todo{Dokumentation des konkreten Vorgehens }

\subsection{Interpretation der Ergebnisse aus den Experteninterviews}
Für die Betriebsdatenerfassung werden spezielle Hardwaresysteme wie störungsunanfällige Terminals oder automatische Signalgeber an Produktionsanlagen eingesetzt, die zum Teil auch einen Prozeßrechner erfordern. Aus diesem Grunde führen BDE-Systeme in der Regel zu einer Vernetzung von mehreren Rechnersystemen.
\todo{WOEZBERGER s.24}

\cite{Benker.2016}s.147

\todo{Beschreibung des Ablaufs}

\todo{Paragraphen für jeden Aspekt}

\todo{Rückgriff auf das Experteninterview}


\subsection{Merkmale und Defizite der Fertigungsdurchführung}

\todo{Relevante Daten für Rückmeldung}

\todo{Daten und Dokumente während der Rückmeldung}

\missingfigure{ER-Diagramm}

\missingfigure{medienbruch nach \cite{Hoppe.2011}}

\todo{2 der Experteninterviews in papierform/ einmal mit Terminal}

\todo{Datenübertragungstechniken}

\todo{Verschwendungen im Prozess}

\missingfigure{IST-Diagramm}

\todo{Was läuft schon automatisch?}



\missingfigure{ darstellung prozess \cite{Beier.2016} s.330}

\todo{Wo Kann automatisiert werden?}

