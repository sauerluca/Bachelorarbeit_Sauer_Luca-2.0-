\section{Anforderungen}\label{sec:anforderungen}
Die \ac{ANF} an die zu entwickelnde Lösung entspringen im Allgemeinen bewährten Abläufen der Fertigungsdurchführung, wobei zur Konkretisierung der Anforderungen die in Abschnitt \ref{sec:untersuchung} dargelegten Erkenntnisse herangezogen werden. Die identifizierten Anforderungen an die Fertigungsdurchführung werden im Folgenden hergeleitet und tabellarisch beschrieben.

\paragraph{ANF-1: Freigabe des Fertigungsauftrags}
Die Freigabe des Fertigungsauftrags stellt eine zentrale Rolle der Fertigungsdurchführung dar. 
Sie wird vom Fertigungssteuerer durchgeführt und endet beim Empfang des freigegebenen Fertigungsauftrags auf Seiten der Werker.  Während der Übergabe der Fertigungsaufträge an den zuständigen Werker ist es notwendig einen digitalen als auch einen papierbasierten Weg anzubieten, um alle Bedürfnisse der Industriebetriebe erfüllen zu können (siehe Abschnitt \ref{sec:untersuchung}).

\begin{table}[H]
	\centering
	\begin{tabularx}{\textwidth}{|l|X|} 
		\hline
		Auslöser                                     &   
		Termin der Freigabe erreicht \\ 
		\hline\hline
		Zeitpunkt                                     &   
		Startpunkt der Fertigungsdurchführung \\ 
		\hline\hline
		Aktivitäten &   
		\begin{minipage}{4.8in}
    		\begin{compactenum} 
        		\renewcommand{\labelenumi}{(\arabic{enumi})}
        		\item Durchführung der Freigabe
        		\item Übergabe des Fertigungsauftrags an den zuständigen Werker
            		\begin{compactenum} 
            		\renewcommand{\labelenumi}{(\arabic{enumi})}
            		\item Druck und Abholung der Papiere
            		\item Digitale Anzeige einer Vorgangsliste
            		\end{compactenum}
    		\end{compactenum}
    		\vspace{1pt}		
		\end{minipage} \\
		\hline\hline
		Herleitung                                        &   
		Der Werker muss möglichst in Echtzeit über neue Fertigungsaufträge benachrichtigt werden. 
		Ist das nicht der Fall können Wartezeit zu Auslastungslücken und Verzögerungen in der Produktion auftreten.\\
		\hline\hline
		Input                                         &   
		Planauftrag  \\ 
		\hline\hline
		Output                                        &   
		Fertigungsauftrag \\
		\hline
	\end{tabularx}
	\caption{\label{tab:anf1}\textbf{ANF-1:} Freigabe des Fertigungsauftrags}
\end{table}



% -----------------------------------------------------------------------
\paragraph{ANF-2: Rückmeldung des Fertigungsauftrags}
Die Rückmeldung des Fertigungsauftrags ist verantwortlich für die Erfassung der Betriebsdaten und spiegelt den Herstellungsfortschritt eines Fertigungsauftrags. 
Während der Rückmeldung eines Fertigungsauftrags sollte versucht werden, den Erfassungsaufwand minimal zu halten. Es ist es notwendig einen digitalen als auch einen papierbasierten Weg anzubieten, um alle Bedürfnisse der Industriebetriebe erfüllen zu können (siehe Abschnitt \ref{sec:untersuchung}). 

\begin{table}[H]
	\centering
	\begin{tabularx}{\textwidth}{|l|X|} 
		\hline
		Auslöser                                     &   
		Fortschritt in der Fertigung \\ 
		\hline\hline
		Zeitpunkt                                     &   
		Jederzeit in der Fertigungsdurchführung \\ 
		\hline\hline
		Aktivitäten &   
		\begin{minipage}{4.8in}
    		\begin{compactenum}
        		\renewcommand{\labelenumi}{(\arabic{enumi})}
        		\item Erfassung der Betriebsdaten
            		\begin{compactenum}
            		\renewcommand{\labelenumi}{(\arabic{enumi})}
            		\item Erfassung in Papierform und anschließende Übertragung in SAP S/4HANA Cloud
            		\item Digitale Erfassung in SAP S/4HANA Cloud
            		\end{compactenum}
    		\end{compactenum}
    		\vspace{1pt}		
		\end{minipage} \\
		\hline\hline
		Herleitung                                        &   
		Der Werker muss möglichst in Echtzeit über die Fortschritte der Fertigungsaufträge Auskunft geben. 
		Ist das nicht der Fall kann nicht auf veränderte Gegebenheiten in der Fertigung reagiert werden.\\
		\hline\hline
		Input                                         &   
		Fertigungsauftrag  \\ 
		\hline\hline
		Output                                        &   
		Rückmeldung \\
		\hline
	\end{tabularx}
	\caption{\label{tab:anf1}\textbf{ANF-2:} Rückmeldung des Fertigungsauftrags}
\end{table}

% -----------------------------------------------------------------------
\paragraph{ANF-3: Erfassung eines Defekts}
Die Erfassung von Defekten ist verantwortlich für anschließende Qualitätsmaßnahemn. 
Treten hierbei Defekte in Form von Qualitätsmängeln oder nicht funktionsfähigen Hilfsmitteln auf, kann das schwerwiegende Folgen haben. Deshalb ist eine unverzügliche Erfassung von Defekten in der Praxis unabkömmlich, auch wenn die Literatur diesen Schritt im Zuge von \ac{PPS} nicht explizit erwähnt (siehe Abschnitt \ref{sec:usecase}). Eine Benachrichtigung an den Fertigungssteuerer ist ein sinnvolles Mittel um Entscheidungen in Echtzeit zu ermöglichen.

\begin{table}[H]
	\centering
	\begin{tabularx}{\textwidth}{|l|X|} 
		\hline
		Auslöser                                     &   
		Defekt aufgetreten \\ 
		\hline\hline
		Zeitpunkt                                     &   
		Jederzeit in der Fertigungsdurchführung \\ 
		\hline\hline
		Aktivitäten &   
		\begin{minipage}{4.8in}
    		\begin{compactenum}
        		\renewcommand{\labelenumi}{(\arabic{enumi})}
        		\item Erfassung eines Defekts
            		\begin{compactenum}
            		\renewcommand{\labelenumi}{(\arabic{enumi})}
            		\item Erfassung in Papierform und anschließende Übertragung in SAP S/4HANA Cloud
            		\item Digitale Erfassung in SAP S/4HANA Cloud
            		\end{compactenum}
    		\end{compactenum}
    		\vspace{1pt}		
		\end{minipage} \\
		\hline\hline
		Herleitung                                        &   
		Der Fertigungssteuerer muss möglichst in Echtzeit über Einblicke in die Gegebenheiten der Fertigung verfügen. 
		Ist das nicht der Fall kann nicht auf kritische Situationen in der Fertigung reagiert werden.\\
		\hline\hline
		Input                                         &   
		-  \\ 
		\hline\hline
		Output                                        &   
		Defekt \\
		\hline
	\end{tabularx}
	\caption{\label{tab:anf1}\textbf{ANF-3:} Erfassung eines Defekts}
\end{table}

% -----------------------------------------------------------------------
\paragraph{ANF-4: Einsatz von mobilen Geräte}
Der Einsatz von mobilen Geräten in der Fertigungsdurchführung verspricht eine Steigerung der Effizienz und Effektivität der Fertigungsdurchführung. Durch die Vermeidung von zusätzlichen Laufwegen zu Terminals und anderen Übergabestellen papierbasierter Betriebsdaten, werden Betriebsdaten direkt am Arbeitsplatz mithilfe von mobilen Geräten erfasst.

\begin{table}[H]
	\centering
	\begin{tabularx}{\textwidth}{|l|X|} 
        \hline
		Aufgaben &   
		\begin{minipage}{4.8in}
    		\begin{compactenum}
        		\renewcommand{\labelenumi}{(\arabic{enumi})}
        		\item Übersicht des Arbeitsvorrats einzelner Arbeitsplätze 
        		\item Erfassung von Rückmeldungen über das mobile Gerät
        		\item Erfassung von Defekten durch das mobile Gerät
            % 		\begin{compactenum}
            % 		\renewcommand{\labelenumi}{(\arabic{enumi})}
            % 		\item Erfassung in Papierform und anschließende Übertragung in SAP S/4HANA Cloud
            % 		\item Digitale Erfassung in SAP S/4HANA Cloud
            % 		\end{compactenum}
    		\end{compactenum}
    		\vspace{1pt}		
		\end{minipage} \\
		\hline\hline
		Herleitung                                        &   
		Der Einsatz von mobilen Geräten kann maßgeblich zur Steigerung der Arbeitseffizienz und zur Erhöhung der Datenqualität beitragen, indem Laufwege eingespart werden, Plausibiltätsprüfungen digital ablaufen und die Transparenz durch Echtzeitdaten erhöht wird.\\
		\hline
	\end{tabularx}
	\caption{\label{tab:anf1}\textbf{ANF-4:} Einsatz von mobilen Geräte}
\end{table}