\section{Verfahrensweisen und wesentliche Gestaltungsprinzipien}\label{sec:methodenGrundlage}
Zur Konkretisierung des Verständnisses werden eingangs die Merkmale dynamischer Geschäftsprozesse beschrieben, um den inhaltlichen Fokus der Arbeit darzulegen.
Da sich die Entwicklung der Methode grundsätzlich an den Ansätzen des sogenannten Software-Engineering orientiert, werden nachfolgend dessen Grundlagen skizziert, um die wesentlichen Gestaltungselemente der Methode in Erfahrung bringen, bevor die Konzepte von Lean Management charakterisiert werden, auf die sich die Methode insbesondere auf der Geschäftsprozessebene stützt.


% Mit den diskutierten informationstechnischen Konzepten, 
% Methoden und Technologien ist es möglich, geschlossene Regel-
% kreise zwischen der Unternehmensleitebene und der Fertigungs-
% ebene aufzubauen. Grundgedanke des RTE ist es, Verzögerungen 
% in Management und Ablauf kritischer Geschäftsprozesse durch 
% Nutzung aktueller Informationen abzubauen [1]. Dieser Heraus-
% forderung sollte nicht alleine mit dem Einsatz potenter IT-Tech-
% nologien begegnet werden. Vielmehr sollte ein ganzheitlicher 
% Ansatz gewählt werden, der bewährte Vorgehensmodelle und 
% Methoden der Betriebswirtschaftslehre einschließt.
% So können die Methoden des Business Process Managements 
% (BPM) und des Business Process Reengineerings (BPR) durch-
% aus auch auf wertschöpfende Unternehmensprozesse auf Ferti-
% gungsebene angewandt werden. Vor der Implementierung kom-
% plexer IT-Systeme muss das Verständnis der zu steuernden 
% Unternehmensprozesse stehen. Des Weiteren kann es notwendig 
% sein, die Unternehmensprozesse auf Basis der Prozessanalyse 
% einer Neugestaltung zu unterziehen. Erst auf der Grundlage einer 
% transparenten Prozessstruktur wird die Entwicklung komplexer 
% informationstechnologischer Lösungen in einem (produ-
% zierenden) Unternehmen erfolgreich verlaufen. Darüber hinaus 
% sollte in einem ganzheitlichen Ansatz neben der Berücksichti-
% gung der Unternehmensprozesse und deren Abbildung und 
% Unterstützung mit (IT-)Systemen auch eine klare Ausarbeitung 
% der Unternehmensstrategie stehen [6]. 
\cite{Grauer.2010}

\subsection{Charakterisierung dynamischer Geschäftsprozesse}
\todo{defintion dynamischer Geschäftsprozess nochmals auf Basis von \ref{ch:Grundlagen}}

\subsection{Grundlagen des Software-Engineering}

\todo{Ingenieurmäßiges Vorgehen hervorheben  nach \cite{Scheer.1991}}

\todo{Softwareentwicklungsprozesses}

\todo{Prototyping}

\cite{JorgBecker.2001}

 \missingfigure{Wesentliche Gestaltungselemente der konzipierten Methode}

\subsection{Lean Management}

\todo{Historie von lean Mangement}

\todo{Kundenbedürfnisse nach \cite{Kieviet.2019}{S.23}}

\todo{Kategorien und Verschwendung nach \cite{Kieviet.2019}{S.37}}

\todo{Leanmanagement konzepte}
% Forschungsseitiger Schwerpunkt seiner Tätigkeit am Institut liegt in den anwendungsorientierten Methoden der Schlanken Produktion, welche in der Prozesslernfabrik CiP erforscht und einem breiten Teilnehmerkreis aus Universität und Industrie vermittelt werden. Darüber hinaus werden Lösungsansätze zur nahtlosen Integration von Produktion, Logistik und Verkehr erarbeitet.

% \hyperlink{https://www.ptw.tu-darmstadt.de/landingpage_ptw/das_ptw_1/geschichte_ptw_1/index.de.jsp}{Link zur Aussage}

\todo{KVP}
% Ausgangspunkt ist ein ausgeprägtes Verständnis der Anwender bezüglich der durch die Software unterstützten Prozesse. Ohne diese grundlegende Kenntnis ist es schwer mög- lieh, Verbesserungspotenziale zu erkennen und zu realisieren. Mit den folgenden Leitfra- gen kann geprüft werden, inwieweit die bestehende Prozessdokumentation ausreicht oder ob ein Nachholbedarf besteht:

\missingfigure{Verschwendungsarten \cite{Muller.2011b}{S.140}}

\paragraph{Probleme mit der Anwendung}

\paragraph{Probleme mit der Datenqualität}

\paragraph{Probleme mit papiergestützten Prozessen}

\paragraph{Probleme im Prozess}

\todo{Zielsetzung: Nutzenpotential VOB MES nach \cite{Gerberich.2011}{s.77}}