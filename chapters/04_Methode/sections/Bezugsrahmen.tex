\section{Verfahrensweisen und wesentliche Gestaltungsprinzipien}\label{sec:methodenGrundlage}
Zur Konkretisierung des Verständnisses werden eingangs die Merkmale dynamischer Geschäftsprozesse beschrieben, um den inhaltlichen Fokus der Arbeit darzulegen.
Da sich die Entwicklung der Methode grundsätzlich an den Ansätzen des sogenannten Software-Engineering orientiert, werden nachfolgend dessen Grundlagen skizziert, um die wesentlichen Gestaltungselemente der Methode in Erfahrung bringen, bevor die Konzepte von Lean Management charakterisiert werden, auf die sich die Methode insbesondere auf der Geschäftsprozessebene stützt.

\subsection{Charakterisierung dynamischer Geschäftsprozesse}
\todo{defintion dynamischer Geschäftsprozess nochmals auf Basis von \ref{ch:Grundlagen}}

\subsection{Grundlagen des Software-Engineering}

\todo{Ingenieurmäßiges Vorgehen hervorheben}

\todo{Softwareentwicklungsprozesses}

\todo{Prototyping}

\cite{Becker.2001}

 \missingfigure{Wesentliche Gestaltungselemente der konzipierten Methode}

\subsection{Lean Management}

\todo{Historie von lean Mangement}


% Forschungsseitiger Schwerpunkt seiner Tätigkeit am Institut liegt in den anwendungsorientierten Methoden der Schlanken Produktion, welche in der Prozesslernfabrik CiP erforscht und einem breiten Teilnehmerkreis aus Universität und Industrie vermittelt werden. Darüber hinaus werden Lösungsansätze zur nahtlosen Integration von Produktion, Logistik und Verkehr erarbeitet.

% \hyperlink{https://www.ptw.tu-darmstadt.de/landingpage_ptw/das_ptw_1/geschichte_ptw_1/index.de.jsp}{Link zur Aussage}



\cite{Muller.2011}{S.140}