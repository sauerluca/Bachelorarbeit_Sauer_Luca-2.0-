% !TEX root =  master.tex

%		LANGUAGE SETTINGS AND FONT ENCODING 
%
\usepackage[ngerman]{babel} 	% German language
\usepackage[utf8]{inputenc}
\usepackage[german=quotes]{csquotes} 	% correct quotes using \enquote{}
\usepackage[T1]{fontenc}
\usepackage{epigraph}
\usepackage{booktabs}
\usepackage{tabularx}
\usepackage{diagbox}
\usepackage[dvipsnames]{xcolor}
\usepackage{chngcntr}
\usepackage{enumitem}
\usepackage{minted}
\usepackage{wasysym}
\setminted{fontsize=\small,baselinestretch=1}

\usepackage{todonotes}
% \presetkeys%
%     {todonotes}%
%     {inline,backgroundcolor=yellow}{}
    

\usepackage{amsmath}
\usepackage{amssymb}
\counterwithout{table}{chapter}
\counterwithout{figure}{chapter}

\usepackage[hidelinks]{hyperref}

\usepackage{pgfplots}
\pgfplotsset{compat=newest}
\usetikzlibrary{patterns}
\makeatletter
\pgfdeclarepatternformonly[\LineSpace]{my north east lines}{\pgfqpoint{-1pt}{-1pt}}{\pgfqpoint{\LineSpace}{\LineSpace}}{\pgfqpoint{\LineSpace}{\LineSpace}}%
{
    \pgfsetcolor{\tikz@pattern@color}
    \pgfsetlinewidth{0.4pt}
    \pgfpathmoveto{\pgfqpoint{0pt}{0pt}}
    \pgfpathlineto{\pgfqpoint{\LineSpace + 0.1pt}{\LineSpace + 0.1pt}}
    \pgfusepath{stroke}
}
\makeatother
\newdimen\LineSpace
\tikzset{
    line space/.code={\LineSpace=#1},
    line space=10pt
}

% Code Highlighting
\usepackage{customizing/listingsstyles}
% Für Definitionsboxen 
\usepackage{amsthm}                     % Liefert die Grundlagen für Theoreme
\usepackage[framemethod=tikz]{mdframed} % Boxen für die Umrandung
% ---- Definition für Highlight Boxen

% ---- Grundsätzliche Definition zum Style
\newtheoremstyle{defi}
  {\topsep}         % Abstand oben
  {\topsep}         % Abstand unten
  {\normalfont}     % Schrift des Bodys
  {0pt}             % Einschub der ersten Zeile
  {\bfseries}       % Darstellung von der Schrift in der Überschrift
  {:}               % Trennzeichen zwischen Überschrift und Body
  {.5em}            % Abstand nach dem Trennzeichen zum Body Text
  {\thmname{#3}}    % Name in eckigen Klammern
\theoremstyle{defi}

% ------ Definition zum Strich vor eines Texts
\newmdtheoremenv[
  hidealllines = true,       % Rahmen komplett ausblenden
  leftline = true,           % Linie links einschalten
  innertopmargin = 0pt,      % Abstand oben
  innerbottommargin = 4pt,   % Abstand unten
  innerrightmargin = 0pt,    % Abstand rechts
  linewidth = 3pt,           % Linienbreite
  linecolor = gray!40,       % Linienfarbe
]{definitionForm}{Definition}     % Name der des formats 

% ------ Definition zum Eck-Kasten um einen Text
\newmdtheoremenv[
  hidealllines = true,
  innertopmargin = 6pt,
  linecolor = gray!40,
  singleextra={              % Eck-Markierungen für die Definition
    \draw[line width=3pt,gray!50,line cap=rect] (O|-P) -- +(1cm,0pt);
    \draw[line width=3pt,gray!50,line cap=rect] (O|-P) -- +(0pt,-1cm);
    \draw[line width=3pt,gray!50,line cap=rect] (O-|P) -- +(-1cm,0pt);
    \draw[line width=3pt,gray!50,line cap=rect] (O-|P) -- +(0pt,1cm);
  }
]{attentionForm}{Definition}  % Name der des formats

\usepackage{tikz}
\usetikzlibrary{shapes.misc,arrows,calc,positioning,positioning,fit,calc}

\tikzstyle{question} = [ fill=gray!10, minimum width=2cm, text centered, draw, rounded rectangle]
\tikzstyle{goal} = [ fill=gray!10, minimum width=6.8cm, text centered, draw,text centered,text width=6,7cm]
\tikzstyle{main} = [ fill=gray!10, minimum width=7.2cm, text centered]
\tikzstyle{submain} = [node distance=0.3cm,  fill=white!100,minimum width=6.8cm,minimum height=0.9cm,text width=6.7cm, text centered, draw, rectangle]

% Zwei eigene Befehle zum Setzen von Autor und Titel. Ausserdem werden die PDF-Informationen richtig gesetzt.
\newcommand{\TitelDerArbeit}[1]{\def\DerTitelDerArbeit{#1}\hypersetup{pdftitle={#1}}}
\newcommand{\AutorDerArbeit}[1]{\def\DerAutorDerArbeit{#1}\hypersetup{pdfauthor={#1}}}
\newcommand{\Firma}[1]{\def\DerNameDerFirma{#1}}
\newcommand{\Kurs}[1]{\def\DieKursbezeichnung{#1}}
\usepackage{nameref}
% Correct superscripts 
\usepackage{fnpct}

%		CALCULATIONS
\usepackage{calc} % Used for extra space below footsepline

%		BIBLIOGRAPHY SETTINGS
%
\usepackage[url=true, isbn=false, doi=false, series=false, backend=biber, style=ieee]{biblatex}
\DeclareLanguageMapping{german}{german-apa}
\AdaptNoteOpt\footcite\multfootcite 
\AdaptNoteOpt\autocite\multautocite
\DefineBibliographyStrings{ngerman}{  %Change u.a. to et al. (german only!)
	andothers = {{et\,al\adddot}},
}

%%% Uncomment the following lines to support hard URL breaks in bibliography 
\apptocmd{\UrlBreaks}{\do\f\do\m}{}{}
\setcounter{biburllcpenalty}{9000}% Kleinbuchstaben
\setcounter{biburlucpenalty}{9000}% Großbuchstaben

\setlength{\bibparsep}{\parskip}		%add some space between biblatex entries in the bibliography
\addbibresource{bibliography.bib}	%Add file bibliography.bib as biblatex resource

%		FOOTNOTES 
%
% Count footnotes over chapters
\counterwithout{footnote}{chapter}

%	ACRONYMS
%%%
%%% WICHTIG: Installieren Sie das neueste Acronyms-Paket!!!
%%%
\makeatletter
\usepackage[printonlyused]{acronym}
\@ifpackagelater{acronym}{2015/03/20}
  {%
    \renewcommand*{\aclabelfont}[1]{\textbf{\textsf{\acsfont{#1}}}}
  }%
  {%
  }%
\makeatother

%		LISTINGS
\usepackage{listings}	%Format Listings properly
\renewcommand{\lstlistingname}{Quelltext} 
 \renewcommand\listingscaption{Quelltext}
\renewcommand{\lstlistlistingname}{Quelltextverzeichnis}
\counterwithout{listing}{chapter}
% \lstset{numbers=left,
% 	numberstyle=\tiny,
% 	captionpos=b,
% 	basicstyle=\ttfamily\small}
	


%		EXTRA PACKAGES
\usepackage{lipsum}    %Blindtext
\usepackage{graphicx} % use various graphics formats
\usepackage[german]{varioref} 	% nicer references \vref
\usepackage{caption}	%better Captions
\usepackage{booktabs} %nicer Tabs
\usepackage{array}
%\newcolumntype{P}[1]{>{\raggedright\arraybackslash}p{#1}}


%		ALGORITHMS
\usepackage{algorithm}
\usepackage{algpseudocode}
\renewcommand{\listalgorithmname}{Quelltextverzeichnis}
\floatname{algorithm}{Quelltext}
\addtocontents{loa}{\def\string\figurename{Quelltext}}

\counterwithout{algorithm}{chapter}


%		FONT SELECTION: Entweder Latin Modern oder Times / Helvetica
\usepackage{lmodern} %Latin modern font
% \usepackage{mathptmx}  %Helvetica / Times New Roman fonts (2 lines)
% \usepackage[scaled=.92]{helvet} %Helvetica / Times New Roman fonts (2 lines)

%		PAGE HEADER / FOOTER
%	    Warning: There are some redefinitions throughout the master.tex-file!  DON'T CHANGE THESE REDEFINITIONS!
\RequirePackage[automark,headsepline,footsepline]{scrpage2}
\pagestyle{scrheadings}
\renewcommand*{\pnumfont}{\upshape\sffamily}
\renewcommand*{\headfont}{\upshape\sffamily}
\renewcommand*{\footfont}{\upshape\sffamily}
\renewcommand{\chaptermarkformat}{}
\RedeclareSectionCommand[beforeskip=0pt]{chapter}
\clearscrheadfoot

\ifoot[\rule{0pt}{\ht\strutbox+\dp\strutbox}DHBW Mannheim]{\rule{0pt}{\ht\strutbox+\dp\strutbox}DHBW Mannheim}
\ofoot[\rule{0pt}{\ht\strutbox+\dp\strutbox}\pagemark]{\rule{0pt}{\ht\strutbox+\dp\strutbox}\pagemark}
\ohead{\headmark}

\newcommand{\spaceparagraph}[1]{\paragraph{#1}~\\~\\}
\newcommand{\nocontentsline}[3]{}
\newcommand{\tocless}[2]{\bgroup\let\addcontentsline=\nocontentsline#1{#2}\egroup}


% Configure Code Highlighting

% \renewcommand{\baselinestretch}{1.2}
\usepackage{textgreek}
\usetikzlibrary{decorations.text,arrows.meta,bending}

\definecolor{mygray}{RGB}{208,208,208}
\newcommand*{\mytextstyle}{\sffamily\footnotesize}
\newcommand{\arcarrow}[3]{%
   % inner radius, middle radius, outer radius, start angle,
   % end angle, tip protusion angle, options, text
   \pgfmathsetmacro{\rin}{1.7}
   \pgfmathsetmacro{\rmid}{2.2}
   \pgfmathsetmacro{\rout}{2.7}
   \pgfmathsetmacro{\astart}{#1}
   \pgfmathsetmacro{\aend}{#2}
   \pgfmathsetmacro{\atip}{5}
   \fill[mygray, very thick] (\astart+\atip:\rin)
                         arc (\astart+\atip:\aend:\rin)
      -- (\aend-\atip:\rmid)
      -- (\aend:\rout)   arc (\aend:\astart+\atip:\rout)
      -- (\astart:\rmid) -- cycle;
   \path[
      decoration = {
         text along path,
         text = {|\mytextstyle|#3},
         text align = {align = center},
         raise = -1.0ex
      },
      decorate
   ](\astart+\atip:\rmid) arc (\astart+\atip:\aend+\atip:\rmid);
}
