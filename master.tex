\documentclass[
	12pt,
	BCOR=5mm,
	DIV=12,
	headinclude=on,
	footinclude=off,
	parskip=half,
	bibliography=totoc,
	listof=entryprefix,
	toc=listof,
	pointlessnumbers,
	plainfootsepline]{scrreprt}

%	Konfigurationsdatei einziehen
% !TEX root =  master.tex

%		LANGUAGE SETTINGS AND FONT ENCODING 
%
\usepackage[ngerman]{babel} 	% German language
\usepackage[utf8]{inputenc}
\usepackage[german=quotes]{csquotes} 	% correct quotes using \enquote{}
\usepackage[T1]{fontenc}
\usepackage{epigraph}
\usepackage{booktabs}
\usepackage{tabularx}
\usepackage{diagbox}
\usepackage[dvipsnames]{xcolor}
\usepackage{chngcntr}
\usepackage{enumitem}

\usepackage{todonotes}
\presetkeys%
    {todonotes}%
    {inline,backgroundcolor=yellow}{}
    

\usepackage{amsmath}
\usepackage{amssymb}
\counterwithout{table}{chapter}
\counterwithout{figure}{chapter}

\usepackage{hyperref}

\usepackage{pgfplots}
\pgfplotsset{compat=newest}
\usetikzlibrary{patterns}
\makeatletter
\pgfdeclarepatternformonly[\LineSpace]{my north east lines}{\pgfqpoint{-1pt}{-1pt}}{\pgfqpoint{\LineSpace}{\LineSpace}}{\pgfqpoint{\LineSpace}{\LineSpace}}%
{
    \pgfsetcolor{\tikz@pattern@color}
    \pgfsetlinewidth{0.4pt}
    \pgfpathmoveto{\pgfqpoint{0pt}{0pt}}
    \pgfpathlineto{\pgfqpoint{\LineSpace + 0.1pt}{\LineSpace + 0.1pt}}
    \pgfusepath{stroke}
}
\makeatother
\newdimen\LineSpace
\tikzset{
    line space/.code={\LineSpace=#1},
    line space=10pt
}

% Code Highlighting
\usepackage{customizing/listingsstyles}
% Für Definitionsboxen 
\usepackage{amsthm}                     % Liefert die Grundlagen für Theoreme
\usepackage[framemethod=tikz]{mdframed} % Boxen für die Umrandung
\input{customizing/boxstyles.tex}

% Zwei eigene Befehle zum Setzen von Autor und Titel. Ausserdem werden die PDF-Informationen richtig gesetzt.
\newcommand{\TitelDerArbeit}[1]{\def\DerTitelDerArbeit{#1}\hypersetup{pdftitle={#1}}}
\newcommand{\AutorDerArbeit}[1]{\def\DerAutorDerArbeit{#1}\hypersetup{pdfauthor={#1}}}
\newcommand{\Firma}[1]{\def\DerNameDerFirma{#1}}
\newcommand{\Kurs}[1]{\def\DieKursbezeichnung{#1}}

% Correct superscripts 
\usepackage{fnpct}

%		CALCULATIONS
\usepackage{calc} % Used for extra space below footsepline

%		BIBLIOGRAPHY SETTINGS
%
\usepackage[url=true, isbn=false, doi=false, series=false, backend=biber, style=ieee]{biblatex}
\DeclareLanguageMapping{german}{german-apa}
\AdaptNoteOpt\footcite\multfootcite 
\AdaptNoteOpt\autocite\multautocite
\DefineBibliographyStrings{ngerman}{  %Change u.a. to et al. (german only!)
	andothers = {{et\,al\adddot}},
}

%%% Uncomment the following lines to support hard URL breaks in bibliography 
%\apptocmd{\UrlBreaks}{\do\f\do\m}{}{}
%\setcounter{biburllcpenalty}{9000}% Kleinbuchstaben
%\setcounter{biburlucpenalty}{9000}% Großbuchstaben

\setlength{\bibparsep}{\parskip}		%add some space between biblatex entries in the bibliography
\addbibresource{bibliography.bib}	%Add file bibliography.bib as biblatex resource

%		FOOTNOTES 
%
% Count footnotes over chapters
\counterwithout{footnote}{chapter}

%	ACRONYMS
%%%
%%% WICHTIG: Installieren Sie das neueste Acronyms-Paket!!!
%%%
\makeatletter
\usepackage[printonlyused]{acronym}
\@ifpackagelater{acronym}{2015/03/20}
  {%
    \renewcommand*{\aclabelfont}[1]{\textbf{\textsf{\acsfont{#1}}}}
  }%
  {%
  }%
\makeatother

%		LISTINGS
\usepackage{listings}	%Format Listings properly
\renewcommand{\lstlistingname}{Quelltext} 
\renewcommand{\lstlistlistingname}{Quelltextverzeichnis}
\lstset{numbers=left,
	numberstyle=\tiny,
	captionpos=b,
	basicstyle=\ttfamily\small}
	


%		EXTRA PACKAGES
\usepackage{lipsum}    %Blindtext
\usepackage{graphicx} % use various graphics formats
\usepackage[german]{varioref} 	% nicer references \vref
\usepackage{caption}	%better Captions
\usepackage{booktabs} %nicer Tabs
\usepackage{array}
%\newcolumntype{P}[1]{>{\raggedright\arraybackslash}p{#1}}


%		ALGORITHMS
\usepackage{algorithm}
\usepackage{algpseudocode}
\renewcommand{\listalgorithmname}{Algorithmenverzeichnis}
\floatname{algorithm}{Algorithmus}
\counterwithout{algorithm}{chapter}


%		FONT SELECTION: Entweder Latin Modern oder Times / Helvetica
\usepackage{lmodern} %Latin modern font
% \usepackage{mathptmx}  %Helvetica / Times New Roman fonts (2 lines)
% \usepackage[scaled=.92]{helvet} %Helvetica / Times New Roman fonts (2 lines)

%		PAGE HEADER / FOOTER
%	    Warning: There are some redefinitions throughout the master.tex-file!  DON'T CHANGE THESE REDEFINITIONS!
\RequirePackage[automark,headsepline,footsepline]{scrpage2}
\pagestyle{scrheadings}
\renewcommand*{\pnumfont}{\upshape\sffamily}
\renewcommand*{\headfont}{\upshape\sffamily}
\renewcommand*{\footfont}{\upshape\sffamily}
\renewcommand{\chaptermarkformat}{}
\RedeclareSectionCommand[beforeskip=0pt]{chapter}
\clearscrheadfoot

\ifoot[\rule{0pt}{\ht\strutbox+\dp\strutbox}DHBW Mannheim]{\rule{0pt}{\ht\strutbox+\dp\strutbox}DHBW Mannheim}
\ofoot[\rule{0pt}{\ht\strutbox+\dp\strutbox}\pagemark]{\rule{0pt}{\ht\strutbox+\dp\strutbox}\pagemark}

\ohead{\headmark}

% Configure Code Highlighting

% \renewcommand{\baselinestretch}{1.2}


% \usepackage{setspace}
% %\singlespacing
% % \onehalfspacing
% %\doublespacing
% % or
% \setstretch{2}   %% change this number as you wish. 1.667 is double spacing.

\begin{document}
%% BITTE GEBEN SIE HIER DEN TITEL UND DIE AUTORIN / DEN AUTOR DER ARBEIT AN!
%% DIESE INFORMATIONEN _MÜSSEN_ GESETZT SEIN, UM TITELBLATT, ABSTRACT UND
%% EIGENSTÄNDIGKEITSERKLÄRUNG AUTOMATISCH ANZUPASSEN!
\TitelDerArbeit{Entwicklung eines dynamischen Geschäftsprozesses zur ereignisgesteuerten Betriebsdatenerfassung mit SAP S/4HANA Cloud}
\AutorDerArbeit{Luca Sauer}
\Firma{SAP SE}
\Kurs{WWI-16-SC-B}

\begin{titlepage}
\begin{minipage}{\textwidth}
		\vspace{-2cm}
		\noindent \includegraphics[scale=0.275]{img/SAP_R_grad.jpg} \hfill   \includegraphics{img/logo.jpg}
\end{minipage}
\vspace{1em}
\sffamily
\begin{center}
	\textsf{\large{}Duale Hochschule Baden-W\"urttemberg\\[1.5mm] Mannheim}\\[2em]
	\textsf{\textbf{\Large{}Bachelorarbeit}}\\[3mm]
	\textsf{\textbf{\DerTitelDerArbeit}} \\[1.5cm]
	\textsf{\textbf{\Large{}Studiengang Wirtschaftsinformatik}\\[3mm] \textsf{Studienrichtung Sales \& Consulting}}
	
	\vspace{3em}
% 	\textsf{\Large{Sperrvermerk}}
\vfill

\begin{minipage}{\textwidth}

\begin{tabbing}
	Wissenschaftlicher Betreuer: \hspace{0.85cm}\=\kill
	Verfasser/in: \> \DerAutorDerArbeit \\[1.5mm]
	Matrikelnummer: \> 8386622 \\[1.5mm]
	Firma: \> \DerNameDerFirma  \\[1.5mm]
	Abteilung: \> IEG S/4HANA - Produce \\[1.5mm]
	Kurs: \> \DieKursbezeichnung \\[1.5mm]
	Studiengangsleiter: \> Prof. Dr. Clemens Martin  \\[1.5mm]
	Wissenschaftlicher Betreuer: \> Dipl.-Math. Heinz Grögler \\
	\> ithzg@groegler.com \\
	\> +49 731 / 267814 \\[1.5mm]
	Firmenbetreuer: \> Dr. Stefan Feickert \\
	\> stefan.feickert@sap.com \\
	\> +49 62277 / 56722 \\[1.5mm]
	Bearbeitungszeitraum: \> 18. Februar 2019 - 13. Mai 2019
\end{tabbing}
\end{minipage}

\end{center}

\end{titlepage} 


\normalfont
\setlength\epigraphwidth{0.8\textwidth}
\renewcommand*{\textflush}{flushright}
\renewcommand*{\epigraphsize}{\normalsize}
\setlength\epigraphrule{0pt}
\renewcommand{\epigraphflush}{flushright}
\def\quotepage{%
      \clearpage%
      \thispagestyle{empty}%
      \addtocounter{page}{-1}%
      \pagebreak
      \hspace{0pt}
      \vfill
      \epigraph{\itshape \enquote{In der digitalen Wirtschaft braucht die Fertigung zusammenhängende und intelligente Lösungen, die es Produktionsmitarbeitern ermöglichen, Fertigungsprozesse transparenter und effizienter zu gestalten und aus Erkenntnissen sofortige Maßnahmen abzuleiten. In der Industrie 4.0 sind Software und Technologie der Schlüssel für Innovationen.}}{\textbf{Bernd Leukert, 2017} \cite{Schell.2017} \\ \textit{Ehemaliges Vorstandsmitglied der SAP SE}}
      \vfill
    \hspace{0pt}
    \pagebreak
      \clearpage
      }

\counterwithin{figure}{chapter}
\counterwithin{table}{chapter}
\counterwithin{figure}{chapter}
\counterwithin{algorithm}{chapter}
%--------------------------------
% Verzeichnisse - nicht benötige Verzeichnisse bitte auskommentieren / löschen.
%--------------------------------
\pagenumbering{gobble}
%   Sperrvermerk
% \input{sperrvermerk/nondisclosurenotice}

\quotepage
\newpage

%	Kurzfassung
\chapter*{Kurzfassung}
 \thispagestyle{empty}
 \begingroup
\begin{table}[h!]
\setlength\tabcolsep{0pt}
\begin{tabular}{p{3.7cm}p{11.7cm}}
Titel & \DerTitelDerArbeit \\
Verfasser/in: & \DerAutorDerArbeit \\
Kurs: & \DieKursbezeichnung \\
Ausbildungsstätte: & \DerNameDerFirma\\
\end{tabular}
\end{table}
\endgroup
Die Fertigungsdurchführung der diskreteten Fertigung muss sich heute schnell an veränderte Situationen anpassen können, um im Wettbewerb weiter zu bestehen. Die Verfügbarkeit von aktuellen, situativen Informationen bei Entscheidungen im laufenden Betrieb ist ein wesentlicher Faktor, um Veränderungen schneller ausführen zu können und Verschwendungen infolge fehlender Informationen zu reduzieren. Zur Unterstützung kommt heute eine Vielzahl aufgabenspezifischer IT-Systeme zum Einsatz, auf welche sich die Informationen verteilen. Diese verursachen während der Betriebsdatenerfassung nicht wertschöpfende Zeiten. Hieraus ergibt sich die Motivation neue, übergreifende Ansätze auf Basis von Ereignisverarbeitung zu entwickeln, welche innerhalb einer diskreten Fertigung das operative Personal in der Erfassung sowie Beschaffung der Informationen unmittelbar und effizient unterstützen.

Den Schwerpunkt bildet das Gesamtkonzept des fertigungsnahen Anwendungssystems, das aus dem Geschäftsprozessmodell und der Ereignisverarbeitungsebene besteht. 
Das Geschäftsprozessmodell beschreibt und vernetzt enthaltene Ereignisse, Aktivitäten und Zusammenhänge der Fertigungsdurchführung in der diskreten Fertigung. Die Ereignisverarbeitungsebene ist als Webservice aufgebaut. Der Webservice verarbeitet eingehende Ereignisse und unterstützt durch unmittelbare Reaktionen auf diese Ereignisse die automatische und manuelle Betriebsdatenerfassung.
Im Rahmen dieser Arbeit wird ein Ansatz zur ereignisgesteuerten Betriebsdatenerfassung von Werkern und Fertigungssteuerern vorgestellt, der es ermöglicht, aus der auftretenden Situation heraus, auf aktuelle notwendige Informationen und die Zusammenhänge in der Fertigungsdurchführung in der diskreten Fertigung zuzugreifen.

Zur Erprobung des Geschäftsprozesses wurde eine softwaretechnische Implementierung in Form eines Prototyps durchgeführt und anhand zuvor definierter Anforderungen evaluiert und validiert.

%	Inhaltsverzeichnis
\tableofcontents

% TO-DOs
\listoftodos


% 	Abkürzungsverzeichnis (siehe Datei acronyms.tex!)
\clearpage
\chapter*{Abkürzungsverzeichnis}	
\addcontentsline{toc}{chapter}{Abkürzungsverzeichnis}

\begin{acronym}
    \acro{A-FD}{Anforderungen mit Bezug zur Fertigungsdurchführung}
    \acro{A-MQ}{Anforderungen mit Bezug zur Modellierungsqualit}
    \acro{A-SQ}{Anforderungen mit Bezug zur Softwarequalität}
    \acro{A-PS}{Anforderungen mit Bezug zur Problemstellung}
    \acro{ABAP}{Advanced Business Application Programming}
    \acro{API}{Application Programming Interface}
    \acro{ARIS}{ Architektur integrierter Informationssysteme}
    \acro{BMWi}{Bundesministerium für Wirtschaft und Technologie}
    \acro{BPMN}{Business Process Model and Notation}
    \acro{CEP}{Complex Event Processing}
    \acro{CIM}{Computer-Integrated Manufacturing}
    \acro{CRM}{Customer-Relationship-Management}
    \acro{CSS3}{Cascading Style Sheets 3}
    \acro{EDA}{Event-Driven Architecture}
    \acro{EPK}{Ereignisgesteuerte Prozesskette}
    \acro{ERM}{Entity-Relationship-Modell}
    \acro{ERP}{Enterprise-Resource-Planning}
    \acro{FA}{Funktionale Anforderungen}
    \acro{GoM}{Grundsätze ordnungsmäßiger Modellierung}
    \acro{HANA}{SAP HANA Datenbank} 
    \acro{HTML5}{Hypertext Markup Language 5}
    \acro{HTTP}{Hypertext Transfer Protocol}
    \acro{IaaS}{Infrastructure as a Service}
    \acro{ISO}{International Organization for Standardization}
    \acro{IT}{Informationstechnik}
    \acro{Java EE}{Java Platform, Enterprise Edition}
    \acro{JMS}{Java Message Service}
    \acro{JSON}{JavaScript Object Notation}
    \acro{MDM}{Master Data Management}
    \acro{MOM}{Message-orientierte Middleware}
    \acro{MRP}{Material Ressource Planing}
    \acro{MVC}{Model View Controller}
    \acro{NFA}{Nicht-funktionale Anforderungen}
    \acro{NIST}{U.S. National Institute of Standards and Technology}
    \acro{OAuth}{Open Authorization Framework}
    \acro{OData}{Open Data Protocol}
    \acro{PaaS}{Platform as a Service} 
    \acro{REST}{Representational State Transfer}
    \acro{SaaS}{Software as a Service}
    \acro{SAML}{Security Assertion Markup Language}
    \acro{SAPUI5}{SAP User Interface for HTML 5}
    \acro{SDK}{Software Development Kit}
    \acro{SOA}{Serviceorientierte Architektur}
    \acro{SOAP}{Simple Object Access Protocol}
	\acro{UCD}{User Centered Design}
	\acro{XML}{Extensible Markup Language}
\end{acronym}

\ohead{Acronyms} % Neue Header-Definition

\pagenumbering{Roman} % Römische Seitennummerierung

%	Abbildungsverzeichnis
\listoffigures

%	Tabellenverzeichnis
\listoftables

%	Listingsverzeichnis
% \lstlistoflistings

% 	Algorithmenverzeichnis
\listofalgorithms



%--------------------------------
% Start des Textteils der Arbeit
%--------------------------------
\clearpage
\ihead{\chaptername~\thechapter} % Neue Header-Definition (inner header)
\ohead{\headmark} % Neue Header-Definition (outer header)
\pagenumbering{arabic}  % Arabische Seitenzahlen

% Auf diese Weise sollten Sie versuchen, für jedes einzelne Kapitel eine eigene Datei anzulegen und mittels input-Kommando einzuziehen.
\setcounter{secnumdepth}{3}
% \input{kapitel/1_Ausgangssituation.tex}
\chapter{Einführung}

\section{Hintergrund und Motivation}
% Bereits der deutsche Schriftsteller und Experimentalphysiker Georg Christoph
% Lichtenberg (1742 -1799) sagte:
% "Ich weiß nicht, ob es besser wird, wenn es anders wird. Aber es muss anders
% werden, wenn es besser werden soll. "
% Auf die Wahrheit dieser Aussage bauen auch die Unternehmen, die seit 2008 mit der
% Wirtschaftskrise zu kämpfen haben. Angefangen als Immobilienkrise 

% Exzellenz in der Produktion basiert auf der Fähigkeit, Mensch, Technik und Organisation optimal miteinander zu verbinden.

\todo{Durch hohe Wettbewerbsdynamik => Konzentration auf Kernkompetenzen, bedeutet für diskrete Fertigung konzentration auf Fertigung}

\todo{Integration, die Automatisierung und die Individualisierung}

\todo{Notwendigkeit zur Dynamisierung von Geschäftsprozessen}

\section{Problemstellung}\label{sec:Problemstellung}
Einen exemplarischen Anwendungsfall für eine derartige Problemstellung mit dynamischen Geschäftsprozessen verkörpert die Fertigungsdurchführung in der Produktionsplanung und -steuerung mit SAP S/4HANA Cloud. Ein Geschäftsprozess der Produktionsplanung und -steuerung muss zusätzlich den hybriden Ansatz der Vernetzung von Mensch und Maschine berücksichtigen.
\cite{Schell.2017}
Ausgehend von der Ausgangssituation muss sich die Fertigungsdurchführung in der diskreten Fertigung heute schnell an Veränderungen anpassen können. Die Verfügbarkeit von aktueller Information bei zu treffenden Entscheidungen, die einen Bezug auf den Ablauf umgebender Geschäftsprozesse besitzen, ist ein wesentlicher Faktor, um Veränderungen schneller ausführen zu können und Verschwendungen infolge fehlender Information zu eliminieren.
\cite{Westkamper.2006}

Der Mehrwert in der Integration von Ereignisverarbeitung und Automation in die Fertigungsdurchführung besteht also darin, die Erfassung der aktuellen Betriebsdaten in Echtzeit zu ermöglichen, um durch zielgerichtete, vorausschauende Entscheidungen Latenzen oder sonstige Arten von Verschwendungen zu vermeiden.  



\section{Zielsetzung und Fokus der Betrachtungen}

Um der betrachteten Problemstellung bei dynamischen Geschäftsprozessen angemessen begegnen zu können, intendiert die vorliegende Arbeit die Entwicklung einer zumindest teilweisen IT-Unterstützung zur Ereignisverarbeitung und Automatisierung des in erster Linie manuell ausgeführten Geschäftsprozesses zur Fertigungsrückmeldung in \textit{SAP S/4HANA Cloud for Manufacturing} basierend auf Konzepten der Ereignisverarbeitung.

Der Fokus der Betrachtungen liegt dabei insbesondere auf der Handhabung der Ereignisverarbeitung innerhalb von \textit{SAP S/4HANA Cloud}. Diese soll auf der einen Seite den Entwurf einer Geschäftsprozessmodells umfassen, das sich nahtlos für den gewählten Geschäftsprozess integrieren lässt.
Auf der anderen Seite soll es die  Transformation des Ereignisverarbeitungsmodells in ausführbaren Code beinhalten, um eine technische Anbindung zu bestehenden Diensten und Ereignisvorrat von \textit{SAP S/4HANA Cloud for Manufacturing} gewährleisten zu können.


% \section{Fachliche Anforderungen an die Bachelorarbeit}

\todo{Der Punkt \enquote{Fachliche Anforderungen an die Bachelorarbeit} Sollte in einen anderen Einleitungspunkt integriert werden}

\cite{Staud2006Geschaftsprozessanalyse:Standardsoftware}
\section{Vorgehen und Aufbau der Bachelorarbeit }\label{sec:Vorgehen}

Das Vorgehen dieser Bachelorarbeit beruht auf der sequenziellen Beachtung der folgenden Interrogativpronomen, die jeweils einen Bereich der Bachelorarbeit charakterisieren: Wozu? Was? Wie? Wohin?

\todo{Fragen beantworten}
\paragraph{Wozu?}

\paragraph{Was?}
tes

\missingfigure{Aufbau der Bachelorarbeit }

\paragraph{Wie?}

\paragraph{Wohin?}

\chapter{Stand der Wissenschaft und Technik}

\section{Automatisierung von Geschäftsprozessen}\label{sec:Automatisierung}
Über Geschäftsprozesse lässt sich die Unternehmensstrategie mit den unterstützenden Informations- und Anwendungssystemen verknüpfen.
In einem wirkungsvollen Unternehmensmanagement müssen demnach diese drei Ebenen der Strategie, der Geschäftsprozesse und der \ac{IT} in ihrer Gesamtheit berücksichtigt werden.
\cite{Scheer.1991}
Insbesondere bei der \ac{IT}-Unterstützung der operativen Geschäftsprozessausführung auf der unteren Ebene ist es vorteilhaft, die Diskrepanzen zwischen den tatsächlichen Geschäftsprozessen und deren informationstechnischen Repräsentation möglichst gering zu halten.
\cite{Gadatsch.2013}

Der Einsatz von Verfahren zur geschäftsprozessorientierten Systemgestaltung wird als adäquates Mittel zur Verknüpfung der betriebswirtschaftlichen mit der informationstechnischen Perspektive angesehen, da derart Verfahren durch semiformale Notationsweisen sowohl komplexe Sachverhalte der Betriebswirtschaft systematisch unterstützten als auch die notwendige Genauigkeit für den Entwurf von Informationssystemen bieten.
\cite{Staud.2006}
Ein Geschäftsprozessmodell repräsentiert dabei im Allgemeinen eine Repräsentanz eines Ausschnittes der realen Welt, im konkreten Fall die Abbildung eines existenten Geschäftsprozesses, die entweder in Gestalt eines Ist-Modells die derzeitige Situation oder in der Ausprägung eines Soll-Modells eine potenziell angestrebte Möglichkeit darstellt.
\cite{Becker.2012}

Die Modellierung von Geschäftsprozessen verfolgt verschiedene Einsatzzwecke.
Dies betrifft zum einen die organisatorische Gestaltung, wie etwa die Dokumentation, Reorganisation oder Optimierung von strategischen und operativen Geschäftsprozessen. 
Auf der anderen Seite können durch das Zusammenfügen der betriebswirtschaftlichen mit der informationstechnischen Perspektive Kriterien für die Gestaltung von Informationssystemen bestimmt werden, welche für die Automatisierung von Geschäftsprozessen eine herausragende Rolle spielen.
\cite{Scheer.2017}
Unter der Automatisierung von Geschäftsprozessen wird in der vorliegenden Bachelorarbeit in Anlehnung an \citeauthor{Abolhassan.2016} \cite{Abolhassan.2016} sowie \citeauthor{Jobst.2010} \cite{Jobst.2010} die digitale Unterstützung und vollständige oder teilweise digitale Ausführung manueller Geschäftsprozesse verstanden.

Um ein Geschäftsprozessmodell letztlich in Gestalt eines Informationssystems abzubilden, existieren alternative Ansätze, wie etwa die Verwendung des Geschäftsprozessmodells zur Beschreibung von zugehörigen Anforderungen, um eine softwaretechnische Implementierung im Rahmen eines Anwendungssystems zu realisieren, die unternehmensspezifische Anpassung von Standardsoftware oder die Umsetzung in ausführbaren Modellen in geeigneten Ausführungsumgebungen.
\cite{Lehmann.2008} 
Im Rahmen dieser Arbeit wird lediglich der erste Ansatz in Betracht gezogen, demnach werden Anwendungssysteme als Aufgabenträger zur Automatisierung von Geschäftsprozessen verstanden, wohl wissend, dass nicht jeder Geschäftsprozess vollständig automatisierbar ist.

\subsection{Merkmale und Kriterien}
Der Einsatz von Verfahren zur geschäftsprozessorientierten Systemgestaltung bietet die Möglichkeit automatisierbare Geschäftsprozesse in fachlich sowie technisch spezifizierten Geschäftsprozessmodellen formalisiert und detailliert zu erfassen.
Grundsätzlich lässt sich die Gestaltung von Geschäftsprozessen dieser Art in einem kreisförmigen Lebenszyklusmodell zum Ausdruck bringen, das in die organisatorische Unternehmensgestaltung integriert werden muss. 
\cite{Scheer.1991}
Ein solches Modell beschreibt den sich wiederholenden, idealtypischen Ablauf der verschiedenen Aufgaben der Geschäftsprozessentwicklung und betont dadurch dessen kontinuierlichen Charakter gegenüber einmaligen und isolierten Prozessverbesserungsinitiativen.
\cite{Leiting.2012}

In der Literatur findet sich eine Vielzahl unterschiedlicher Lebenszyklusmodelle.
\cite{MacedodeMorais.2014} 
Diese unterscheiden sich zwar hintsichtlich der Anzahl, Benennung sowie Partitionierung der einzelnen Aufgaben in Lebenzyklusphasen, in ihrer Essenz weichen sie jedoch nicht fundamental voneinander ab.
\cite{Houy.2010}
In der vorliegenden Bachelorarbeit wird das etablierte Lebenszyklusmodell von \citeauthor{Scheer.1991}
\cite{Scheer.1991} zugrunde gelegt.
Der Ablauf dieses aus vier Phasen bestehenden Modells ist in Abbildung \ref{fig:Phasenmodell bei der Automatisierung von Geschäftsprozessen} illustriert und wird im Folgenden näher beleuchtet.

Im ersten Schritt wird eine \ac{IT}-orientierte fachliche Ausgangslösung erstellt.
Diese ergibt sich aus der eingehenden Analyse eines neuen oder existierenden Geschäftsprozesses, in Abbildung \ref{fig:Phasenmodell bei der Automatisierung von Geschäftsprozessen} im linken oberen Bereich veranschaulicht, durch die zunächst die grundsätzlichen Anforderungen des zu untersuchenden Geschäftsprozesses sichtbar gemacht werden.
\cite{Schwegmann.2002}
Aus diesem Grund werden hier auch noch alle Perspektiven zusammen betrachtet.

Die darauf folgende Konzeptionsphase behandelt, auf Basis der zuvor erhobenen Anforderungen an einen Geschäftsprozess, dessen Beschreibung auf fachlicher Ebene. 
\cite{Schwegmann.2002}
Anschließend wird das Fachkonzept, unabhängig von Implementierungsgesichtspunkten, mit technischen Anforderungen an das Anwendungssystem angereichert, sodass es als Ausgangspunkt für eine konsistente softwaretechnische Implementierung dienen kann.
\cite{Scheer.1991}
Dabei wird jedoch noch kein Bezug zu plattformspezifischen Programmiersprachen hergestellt. 
Das technisch spezifizierte Konzept kann geändert werden, ohne dass dies Auswirkungen auf das Fachkonzept hat.
\cite{Speck.2002}
Dies bedeutet jedoch nicht, dass Fachkonzept und technische Spezifikation isoliert voneinander entwickelt werden können. 
Mehr noch soll nach Abschluss der fachkonzeptionellen Darstellung der betriebswirtschaftliche Inhalt so definiert sein, dass ausschließlich \ac{IT}-bezogene Argumente, wie das Leistungsverhalten eines Informationssystems, keine Auswirkungen auf die Fachinhalte nehmen können.  

\begin{figure}[H]
	\centering 
    \begin{tikzpicture}
       \arcarrow{177}{ 96}{Analyse}
       \arcarrow{ 89}{  3}{Konzeption}
       \arcarrow{268}{361}{Implementierung}
       \arcarrow{179}{271}{Ausf{\"u}hrung}
    \end{tikzpicture}
    \caption[Phasenmodell bei der Automatisierung von Geschäftsprozessen]
    {Phasenmodell des Lebenszyklus von Geschäftsprozessen \protect\footnotemark}
    \label{fig:Phasenmodell bei der Automatisierung von Geschäftsprozessen}
\end{figure}
\footnotetext{in Anlehnung an \citeauthor{Scheer.1991} \citeyear{Scheer.1991} \cite{Scheer.1991} }

In der Implementierungsphase wird das Geschäftsprozessmodell in eine ausführbare Programmiersprache überführt somit in eine Anwendungssoftware transformiert, getestet und in ein Anwendungssystem integriert.
\cite{Scheer.1991}
Während der Ausführungsphase erfolgt häufig eine Überwachung des laufenden Geschäftsprozesses, um in einer nachfolgenden Iteration des gesamten Phasenmodells eine Optimierung des Geschäftsprozessmodells basierend auf den angeeigneten Erkenntnissen durchführen zu können.
\cite{Scheer.2005}
Der Fokus dieser Arbeit liegt insbesondere auf der Phase der Analyse bis zur softwaretechnischen Implementierung, wohingegen die Ausführung vernachlässigt wird.

In der geschäftsprozessorientierten Systementwicklung steht während der Modellierung die ablauforientierte Sicht im Mittelpunkt.
Geschäftsprozessorientierte Modellierungsansätze betrachten das darzustellende Informationssystem ausgehend von der der informationstechnischen Unterstützung menschlicher Arbeitsabläufe, die sie erfüllen sollen, und dem dazu notwendigen Vorgehen, das sich aus der Folge der durchzuführenden Aktivitäten ergibt.
\cite{Wolf.2016}
Dabei werden die zu erfüllenden Aktivitäten systematisch analysiert und anschließend entsprechend ihrer Reihenfolge in den Geschäftsprozess integriert werden.
Eine Aktivität \footnote{Aktivitäten werden in anderer Literatur oft auch als Geschäftsprozessschritte bezeichnet.} ist ein diskreter Schritt innerhalb eines Prozesses, welcher entweder manuell durch einen Menschen oder automatisch durch einen Dienst durchgeführt wird.
\cite{Benker.2016}
Das daraus resultierende Geschäftsprozessmodell repräsentiert den zur Erfüllung der Aufgabe notwendigen Ablauf von Aktivitäten, mit welchem ein übergeordneter Geschäftszweck verfolgt wird.

Durch die stetige Verbreitung von serviceorientierten Architekturen besteht für Unternehmen die Möglichkeit, vollständige Geschäftsprozesse oder einzelne Aktivitäten auf Basis modularer digitaler Dienste in einer lose gekoppelten Weise aufeinander abzustimmen und zu automatisieren. 
\cite{Masak.2007}
Eine \ac{SOA} zielt auf eine optimale Unterstützung der fachlichen Geschäftsprozesse durch die Dynamisierung der Informationssysteme ab; Unternehmensstrategie und Informations- und Anwendungssysteme sollen besser integriert werden.
\cite{Teusch.2016}

% \interfootnotelinepenalty=10000
\begin{figure}[H]
	\centering 
    \includegraphics[width=\textwidth]{img/Serviceaufbau.png}	
    \caption[Nutzung eines Dienstes]
    {Nutzung eines Dienstes  \protect\footnotemark}
    \label{fig:Nutzung eines Dienstes}
\end{figure}
\footnotetext{in Anlehnung an \citeauthor{Masak.2007} \citeyear{Masak.2007} \cite{Masak.2007} }

Hierbei wird die wesentliche Geschäftslogik der Aktivitäten in digitalen Diensten gekapselt, während der übergreifende Geschäftsprozess in Form von Anwendungssoftware die eher reaktive und demnach dynamische Ausführungssemantik implementiert. 
\cite{Teusch.2016}
Unter dem Aspekt der Wiederverwendbarkeit eines Dienstes in einer \ac{SOA} sollte des Weiteren der in Abbildung \ref{fig:Nutzung eines Dienstes} illustrierte Umstand, dass digitale Dienste zur Ausführung einer Aktivität auf Basis weiterer Dienste aufgebaut sein können oder im Umkehrschluss ein Dienst von unterschiedlichen Aktivitäten genutzt werden kann, einer kritischen Betrachtung unterzogen werden.
\cite{Masak.2007}

Sogenannte Webservices stellen dabei eine auf \ac{XML} ausgerichtete und auf offenen Standards basierende Technologie zur Realisierung einer \ac{SOA} dar, indem sie die in einer beliebigen Programmiersprache implementierte Geschäftslogik in Form von Diensten anhand vereinheitlichter Schnittstellen nach außen zur Verfügung stellen.
\cite{Masak.2005}
An dieser Stelle sei jedoch der Hinweis erlaubt, dass das Konzept der Serviceorientierung allgemeiner ist und schon früher existierte als Webservices. Webservices sollten daher nur als eine, wenn auch nur zum Verfassungszeitpunkt dieses Buches als wahrscheinlich am besten geeignete Möglichkeit zur Realisierung serviceorientierter Architekturen betrachtet werden.

Ein Webservice ist als Maschine-zu-Machine-Kommunikation zu verstehen. Diese Maschinen sprechen über offenen Standards miteinander. 
\cite{Finger.2009b}
Nachdem ein Webservice in ein Anwendungssystem integriert wurde, erfolgt die Kommunikation in der Regel automatisch. 
Ein Endanwender auf der Seite der Benutzerschnittstelle wird nicht zur Kenntnis nehmen, dass die Anwendungssoftware, die er bedient, mit einem Webservice kommuniziert. Dieses Prinzip entspricht ebenfalls dem Grundgedanken einer \ac{SOA}.
\cite{Teusch.2016}
Auf diese Weise können Webservices beispielsweise aus einem Geschäftsprozess heraus aufgerufen werden, ohne ihre konkrete Implementierung zu kennen, was insbesondere in einem geschäftsprozess- oder unternehmensübergreifenden Kontext ein bedeutender Faktor ist. Zur Änderung eines Geschäftsprozesses bedarf es lediglich einer Anpassung der Ausführungssemantik der Anwendungssoftware durch veränderte Aufrufe von bestehenden oder zusätzlichen Webservices.

% \todo{SOA Texte durch SAP Book Quelle ersetzen \cite{Pohl.2009} \cite{Hoppe.2011}}

Ferner können auch von Menschen ausgeführte Aktivitäten als Dienste gekapselt und mit Benutzerschnittstellen ausgestattet werden, indem die Aufforderung zum Beginn der manuellen Aktivität durch eine geeignete Ausgabe signalisiert wird, woraufhin der erfolgreiche Abschluss der manuellen Aktivität oder ein erforderlicher Abbruch wiederum durch eine Eingabe quittiert wird. 
Auf diese Weise ist auch die Integration manueller Aktivitäten in einen automatisierten Geschäftsprozess möglich.
\cite{Weske.2007}

Für die Konzeption und Implementierung von automatisierbaren Geschäftsprozessen ist eine angemessene Spezifizierung erforderlich, die verschiedene Kriterien erfüllen muss.
Bei der Betrachtung dynamischer Geschäftsprozesse im Rahmen dieser Bachelorarbeit spielen dabei auch solche Merkmale eine herausragende Rolle, welche die Ausprägung dynamischer Eigenschaften anhand einer zielgerichteten Ereignisorientierung und Ereignisverarbeitung unterstützen. 

Im Fokus der Betrachtungen sind verschiedene Kriterien zur Auswahl einer geeigneten Modellierungssprache für das zugrunde liegende Geschäftsprozessmodell von Belang, die in Tabelle \ref{tab:Kriterien für Modellierungssprachen} jeweils mit einer Beschreibung ihrer Bedeutung aufgeführt sind.

\begin{table}[H]
	\centering
	\begin{tabularx}{\textwidth}{l X} 
		\toprule
		\textbf{Kriterium}  &   
		\textbf{Beschreibung}  \\ 
		\toprule
		Funktionalität &   
		Unterstützung verschiedener Aktivitätsarten, die Darstellung sequenzieller, paralleler, alternativer und iterativer Prozessabläufe sowie die Verbindung von Aktivitäten mit Objekten, Relationen und Rollen. \cite{Funk.2010b} \\  \cmidrule(r){1-1} \cmidrule(r){2-2}
		
		Verständlichkeit &   
		Auf fachlicher Modellebene ist eine modellbasierte Darstellung der Geschäftsprozesse erwünscht, die von Fachleuten mit betriebswirtschaftlichem und technischem Hintergrund mit vertretbarem Aufwand verstanden werden kann. \\ \cmidrule(r){1-1} \cmidrule(r){2-2}
		
		Formalisierbarkeit &   
		Die Notation soll formale oder semiformale Modellierungsregeln aufweisen, damit die Konzeption korrekter Geschäftsprozessmodelle unterstützt wird und eine nahtlose Implementierung des technischen Geschäftsprozessmodells erfolgen kann. \cite{Becker.2012}  \\ \cmidrule(r){1-1} \cmidrule(r){2-2}
		
		Unterstützung von \ac{SOA} &   
		Digitale Dienste, insbesondere Webservices, und von Menschen durchgeführte Aktivitäten mit geeigneten Benutzerschnittstellen sollen integriert werden können.  \\ \cmidrule(r){1-1} \cmidrule(r){2-2}
		
		Ereignisverarbeitung &   
		Die Modellierungssprache soll ein Konzept für die Integration von Ereignissen in das Geschäftsprozessmodell enthalten und insgesamt eine hohe Ereignisorientierung aufweisen.   \\ \cmidrule(r){1-1} \cmidrule(r){2-2}
		
		Standardisierung &   
		Die Modellierungssprache soll durch eine zentrale Institution standardisiert worden sein oder zumindest einen anerkannten Industriestandard darstellen. \\ \cmidrule(r){1-1} \cmidrule(r){2-2}
		
		Werkzeugunterstützung &   
		Umfangreiche Softwarewerkzeuge für die Modellierung sollen verfügbar sein.  \\
	    \bottomrule
	\end{tabularx}
	\caption[Kriterien für Modellierungssprachen]
    {Kriterien für Modellierungssprachen für dynamische Geschäftsprozesse}
    \label{tab:Kriterien für Modellierungssprachen}
\end{table}

\subsection{Gegenüberstellung geeigneter Modellierungssprachen}
Im Folgenden werden bekannte und etablierte Modellierungssprachen für die Modellierung von Geschäftsprozessen herangezogen, die als Basis für dynamische Geschäftsprozesse dienen können. 
Diese werden zunächst einzeln begreiflich gemacht und anschließend in Bezug auf die in Tabelle \ref{tab:Kriterien für Modellierungssprachen} genannten Kriterien in einer tabellarischen Übersicht gegenübergestellt. 
% Es ist nicht Ziel des folgenden Abschnitts die \acl{EPK}, \acl{BPMN} 2.0 und  Aktivitätsdiagramme der \acl{UML} ausführlich zu diskutieren und vollumfänglich darzustellen.
Es sollen wesentliche Eigenschaften unds Elemente daraus erklärt werden, die für das Verständnis dieser Bachelorarbeit notwendig sind.

\paragraph{\acl{EPK}}
Die \textit{\acf{EPK}} stellt die zentrale Modellierungssprache in der \acf{ARIS} dar.
Entwickelt wurde die \ac{ARIS}-Architektur bereits \citeyear{Scheer.1991} am Institut für Wirtschaftsinformatik in Zusammenarbeit mit dem \ac{CIM}-Technologie-Transfer-Zentrum an der Universität des Saarlandes in Kooperation mit der SAP AG.\footnote{Am 07. Juli 2014 erfolgte eine Umwandlung der SAP AG von einer Aktiengesellschaft in eine Europäische Aktiengesellschaft (SE).}
\cite{Scheer.1991}
Der Modellierungsansatz der \ac{EPK} hat sich insbesondere im deutschsprachigen Raum als eine der meistverbreitetste semiformale Methode zur Modellierung von Geschäftsprozessen durchgesetzt. 
\cite{Gadatsch.2013}
Diese Verbreitung lässt sich unter anderem darauf zurückzuführen, dass die SAP SE die \ac{EPK} als Standard in ihr Produktportfolio integriert hat. 
\cite{Staud.2006}
In \ac{EPK}s, die nicht sequentiell ablaufen, bilden Konnektoren die Verknüpfungsknoten. Gemeint sind hierbei die aus der Aussagenlogik stammenden konjunktiven, adjunktiven und disjunktiven Verknüfungen. Dadurch wird gewährleistet, dass auch alternative, parallele und iterative Geschäftsprozessabläufe durch logische Ereignisverknüpfungen erstellt werden können.
\cite{Lehmann.2008}

In einer erweiterten \ac{EPK} wird die einfache \ac{EPK}, die sich auf Ereignisse, Funktionen, Prozessschnittstellen, Konnektoren und Kanten beschränkt, um zusätzlich verschiedene Elemente wie die organisatorische Einheit, Informationsobjekte, Anwendungssysteme oder Prozesswegweiser ergänzt. 
\cite{Seidlmeier.2015}
Dadurch ist es möglich umfassende Geschäftsprozessmodelle mit all ihren Beziehungen zum Umfeld des Geschäftsprozesses in semiformaler Form darzustellen, um durch die Modellierungssprache alle Perspektiven auf den Geschäftsprozess miteinander zu verbinden.
\cite{Gadatsch.2013}

\paragraph{\acl{BPMN} 2.0}
Die \textit{\acf{BPMN} 2.0} ist eine weltweit verbreiteter Standard der \ac{OMG} zur Modellierung von Geschäftsprozessen.
Da der ersten Version von \ac{BPMN} kein Metamodell zugrunde liegt, handelt es sich bei \ac{BPMN} in seiner ursprünglichen Gestalt streng genommen nicht um eine Modellierungssprache, dies wurde  mit Version 2.0 jedoch behoben. 
\cite{OMG.2014}

Die \ac{BPMN} 2.0 beinhaltet eine Reihe an Elementen für verschiedene Aktivitäten in einem Geschäftsprozess, wie etwa Sendeaufgaben, Empfangsaufgaben, Benutzeraufgaben, Dienstaufgaben, oder Geschäftsregelaufgaben, und eine Vielzahl an unterschiedlichen Ereignistypen, wie etwa Nachrichten-, Zeit-, Signal- oder Fehlerereignisse. 
Darüber hinaus kann der Ablauf eines Prozesses mithilfe sogenannter Gateways gesteuert werden und es kann zudem auf Datenobjekte zugegriffen werden. 
\cite{Weidlich.2010}
Obwohl die Geschäftsprozessmodellierung in \ac{BPMN} 2.0 bezüglich der konkreten Realisierungstechnologien allgemein gehalten ist, werden \ac{SOA} und Webservices durch die reichhaltige Auswahl an Elementen in besonderem Maße unterstützt.
\cite{Jobst.2010}

\paragraph{Aktivitätsdiagramme der \acl{UML}}
Das Aktivitätsdiagramm ist ein Diagrammtyp, der von der \ac{OMG} sowie der \ac{ISO} zusammen mit anderen Diagrammtypen zur Modellierung von Informationen in der \acf{UML} vereinheitlicht wurde.
\cite{OMG.2014}\cite{ISO.2012}
\ac{UML} ist heute eine der dominiernden Modellierungssprachen zur Darstellung informationstechnischer und anderer Systeme, die sich bereits in vielfältigen Einsatzgebieten in der Praxis bewährt hat.
Neben einer grafischen Notation spezifiziert die \ac{UML} auch die Semantik objektorientierter Komponenten und deren Beziehungen mit definierten Regeln für Strukturen und Abläufe.
\cite{Rumpe.2011}

Obwohl die \ac{UML} ursprünglich nicht für die Modellierung von Geschäftsprozessen entworfen wurde, ist eine Geschäftsprozessmodellierung dennoch durch den Entwurf von Aktivitätsdiagrammen eingeschränkt möglich, da hiermit funktionale Abläufe mit Hilfe von Aktivitäten und Verknüpfungen darstellbar sind. 
\cite{vanRanden.2016}
Wohl aber existieren verschiedene Defizite, etwa bezüglich der Verarbeitung von Ereignissen oder durch eine umständliche Darstellung komplexer Verknüpfungen, da bei der \ac{UML} die Systemorientierung gegenüber der Geschäftsprozessorientierung deutlich prädominiert.
\cite{Staud.2006}

\paragraph{Gegenüberstellung der Modellierungssprachen}
In Tabelle \ref{tab:Bewertung von Modellierungssprachen} wird die Eignung der vorgestellten Modellierungssprachen in Bezug auf die relevanten Kriterien aus dem vorherigen Abschnitt in tabellarischer Form gegenübergestellt.

\newcolumntype{Y}{>{\centering\arraybackslash}X}
\begin{table}[H]
	\centering
	\begin{tabularx}{\textwidth}{l Y Y Y} 
		\toprule
		\textbf{Kriterium}  &   
	    \multicolumn{3}{c}{\textbf{Modellierungssprache}}	  \\   \cmidrule(r){2-4}
	    
	                  &   
		\textbf{\acs{EPK}}\cite{Scheer.1991}\cite{Lehmann.2008}      &
		\textbf{\acs{BPMN}}\cite{OMG.2014}\cite{Weidlich.2010}       &
		\textbf{\acs{UML}}\cite{ISO.2012}\cite{Rumpe.2011}	    \\  \cmidrule(r){1-1} \cmidrule(r){2-2} \cmidrule(r){3-3} \cmidrule(r){4-4}
	    
		Funktionalität &   
		\CIRCLE  &
		\CIRCLE   &
		\CIRCLE		\\ \cmidrule(r){1-1} \cmidrule(r){2-2} \cmidrule(r){3-3} \cmidrule(r){4-4}
		
		Verständlichkeit &   
		\CIRCLE &
		\CIRCLE   &
		\LEFTcircle			\\ \cmidrule(r){1-1} \cmidrule(r){2-2} \cmidrule(r){3-3} \cmidrule(r){4-4}
		
		Formalisierbarkeit &   
		\LEFTcircle	 &
		\CIRCLE   &
		\CIRCLE		\\ \cmidrule(r){1-1} \cmidrule(r){2-2} \cmidrule(r){3-3} \cmidrule(r){4-4}
		
		Unterstützung von \ac{SOA} &   
		\Circle &
		\CIRCLE   &
		\Circle		\\ \cmidrule(r){1-1} \cmidrule(r){2-2} \cmidrule(r){3-3} \cmidrule(r){4-4}
		
		Ereignisverarbeitung &   
		\CIRCLE &
		\CIRCLE   &
		\LEFTcircle	\\ \cmidrule(r){1-1} \cmidrule(r){2-2} \cmidrule(r){3-3} \cmidrule(r){4-4}
		
		Standardisierung &   
		\LEFTcircle	 &
		\CIRCLE   &
		\CIRCLE		\\ \cmidrule(r){1-1} \cmidrule(r){2-2} \cmidrule(r){3-3} \cmidrule(r){4-4}
		
		Werkzeugunterstützung  &
		\CIRCLE &
		\CIRCLE   &
		\CIRCLE		\\ 
	    \bottomrule
	    \multicolumn{4}{c}{
	    Legende:
	    \CIRCLE
	    erfüllt
	    \LEFTcircle
	    bedingt erfüllt
	    \Circle wenig bis nicht erfüllt}
	\end{tabularx}
	\caption[Bewertung von Modellierungssprachen]
    {Bewertung von Modellierungssprachen für dynamische Geschäftsprozesse}
    \label{tab:Bewertung von Modellierungssprachen}
\end{table}

Es wird ersichtlich, dass für die Spezifizierung dynamischer Geschäftsprozessmodelle zwar alle Modellierungssprachen durch Werkzeugunterstützung praktisch anwendbar sind, diese sich allerdings deutlich anhand anderer Kriterien unterscheiden. Auffallend ist des Weiteren, dass die Unterstützung der benötigten Funktionalitäten in allen betrachteten Modellierungssprachen gegeben ist, sodass grundsätzlich die Schlussfolgerung gezogen werden kann, dass alle Modellierungssprachen prinzipiell für die Modellierung dynamischer Geschäftsprozesse einsetzbar sind. 

Aufgrund der zwingenden Ereignisbehandlung des Geschäftsprozessmodells, um für die notwendige Dynamik des Geschäftsprozesses zu sorgen, werden lediglich \ac{EPK} und \ac{BPMN} 2.0 als Modellierungssprachen in Betracht gezogen. 
\ac{BPMN} 2.0 zeichnet sich hierbei durch die erforderliche Untersützung von SOA mit Webservices und eine bessere Ausgestaltung der Ereignisverarbeitung mit umfassenderen Ereigniskonzepten sowie einer unabhängig regulierten Standardisierung aus. Letzteres drückt sich insbesondere in der Möglichkeit der grafischen Modellierung von Geschäftsprozessen und somit einer besseren Modellkonsistenz auf der fachlichen Ebene aus. Aus diesen Gründen fällt die Wahl schließlich auf den Einsatz von \ac{BPMN} 2.0 zur formalen Beschreibung des Geschäftsprozessmodells in Kapitel \ref{ch:Durchfuehrung}. 

\section{Konzept der Ereignisverarbeitung}\label{sec:Ereignisverarbeitung}
In der realen Welt ist eine Wechselwirkung aller Geschehnisse mit einer Vielzahl von divergenten, meist nicht-deterministisch eintretenden Ereignissen zu beobachten, als Konsequenz folgt eine signifikante Beeinflussung auf den Fortgang der Geschehnisse.
\cite{Grauer.2010}
Dieser Tatbestand findet sich auch in den Abläufen und Geschäftsprozessen in Unternehmen wieder.
Ein Unternehmen ist demzufolge gezwungen, auf diese Ereignisse angemessen und möglichst zeitnah zu reagieren – die Unternehmen operieren demnach ereignisgesteuert. 
\cite{Schaaf.2015}
Mit dem Konzept der Ereignisverarbeitung rücken Ereignisse als zentraler Leitgedanke einer Orientierung an Ereignissen in den Fokus der Gestaltung von Geschäftsprozessen, indem Ereignisse als Baustein der Softwarearchitektur und der Geschäftslogik in den Mittelpunkt der Betrachtung rücken. 
\cite{Bruns.2010}

Die resultierenden ereignisgesteuerten Unternehmensanwendungen ermöglichen eine praxisnahe Darstellung der dynamischen Geschäftsprozesse eines Unternehmens. 
Das Ziel ist es dadurch die Agilität, Reaktionsfähigkeit und Echtzeitfähigkeit der Geschäftsprozesse eines Unternehmens zu erhöhen. 
In der Praxis hat eine derartige Ereignisorientierung in Unternehmensanwendungen bereits breiten Zuspruch gefunden. 
\cite{Bruns.2015}
In diesem Kontext beruhen ereignisgesteuerte Architekturen, englisch \ac{EDA}, auf einem ereignisgesteuerten Prinzip, bei dem eine lose Kopplung zwischen den beteiligten Komponenten eines Informationssystems vorgesehen ist.  
In ihrer reinen Ausprägung kommunizieren diese Komponenten ausschließlich mittels sogenannter Ereignisbenachrichtigungen, englisch \textit{Event Messages}, miteinander.
Abbildung \ref{fig:Grundschritte von ereignisgesteuerten Architekturen} illustriert den Ablauf der Grundschritte ereignisgesteuerter Architekturen, deren Charakteristika nachfolgend erläutert werden.
\cite{Schaaf.2015}

\begin{figure}[H]
	\centering 
    \begin{tikzpicture}
        \fill[even odd rule, white] circle (1.5);
    
       \node at (0,0) [
          font  = \sffamily\Large\bfseries\color{black!85},
          align = center
       ]{
          \acs{EDA}
       };
       \arcarrow{197}{ 96}{Erkennen}
       \arcarrow{ 89}{-17}{Verarbeiten}
       \arcarrow{199}{341}{Reagieren}
    \end{tikzpicture}
    \caption[Grundschritte von ereignisgesteuerten Architekturen]
    {Grundschritte von ereignisgesteuerten Architekturen \protect\footnotemark}
    \label{fig:Grundschritte von ereignisgesteuerten Architekturen}
\end{figure}
\footnotetext{in Anlehnung an \citeauthor{Bruns.2010} \citeyear{Bruns.2010} \cite{Bruns.2010} }

Demnach besteht eine zentrale Charakteristik von ereignisgesteuerten Architekturen aus der Komposition von drei Grundschritten: \textit{Erkennen, Verarbeiten und Reagieren von oder auf Ereignisse.}

\paragraph{Erkennen} 
Das Auftreten von relevanten Informationen und Sachverhalten in  den Geschäftsprozessen ist der Ausgangspunkt für die Erkennung von Ereignissen.
Diese Informationen und Sachverhalte werden analysiert, als Ereignisse klassifiziert und spiegeln einen spezifizierten Ausschnitt des Zustands eines Geschäftsprozesses wieder. 
Für registrierte Ereignisse wird ein entsprechendes Ereignisobjekt generiert.
Entscheidend für ereignisgesteuerte Informationssysteme ist, dass die Ereignisse unmittelbar zum Zeitpunkt ihres Auftretens erkannt werden und nicht zeitverzögert.
\cite{Bruns.2010}

\paragraph{Verarbeiten}
Im Verarbeitungsschritt werden die erkannten Ereignisse, die aus unterschiedlichen Ereignisquellen stammen können, analysiert. Bei der Analyse werden Ereignisse zu Einheiten aggregiert, mit anderen Ereignissen verknüft, generalisiert, aufgeteilt oder aber auch als irrelevant eingestuft. 
Gesucht werden Paradigmen in den gesammelten Ereignissen, die bestimmte Beziehungen und Abhängigkeiten zwischen den Ereignissen ausdrücken.
\cite{Hedtstuck.2017}

\paragraph{Reagieren}
Aufgrund von analysierten Mustern, die im Fluss der eingetretenen Ereignisse erkannt wurden, können vielfältige Arten von Reaktionen zeitnah veranlasst werden. 
Die Reaktionen, die angestrebt werden, charkterisieren sich durch Aktualität und Individualität, wie etwa die unmittelbare Übergabe der Ereignisse an Anwendungssysteme, das Senden von Benachrichtigungen, der Aufruf von Funktionen in Form von digitalen Diensten, das Auslösen eines Geschäftsprozesses oder die Initiierung von manuellen Aktivitäten durch menschliche Benutzer, aber auch die Generierung neuer Ereignisse ist eine legitime Reaktion.
\cite{Bruns.2010}

In der Realität existieren allerdings kaum Geschäftsprozesse, bei denen eine ausschließliche Kommunikation mithilfe von Ereignissen und alleinig im Rahmen der beschriebenen Grundschritte einer \ac{EDA} erfolgen, weshalb auch Anwendungen mit teilweiser Ereignisverarbeitung und partieller Anwendung dieser Grundschritte die Umsetzung einer ereignisgesteuerten Architektur zugeschrieben.
\cite{Etzion.2011}
Die \ac{SOA}-Architektur beinhaltet, wie in Abschnitt \ref{sec:Automatisierung} beschrieben, die Konzepte, die notwendig sind, um Echtzeit-Anwendungssysteme zu ermöglichen und die Unterstützung der Dynamik ereignisgesteuerter Geschäftsprozesse zur selben Zeit zu gewährleisten, nicht vollständig.
Eine \ac{SOA} ohne Ereignisverarbeitung ist folglich nicht in der Lage, den aktuellen Zustand von Aktivitäten zu erkennen, da hierzu in Echtzeit temporale und kausale Verbindungen zwischen Geschäftsprozessen und Aktivitäten identifiziert und analysiert werden müssen. 
Eine \ac{EDA} kann als Maßnahme auch eine \ac{SOA} komplementieren, zumal beide Konzepte grundsätzlich modulare und verteilte Informationssysteme mit loser Kopplung unterstützen, um die Vorteile beider Architekturen im Kollektiv zu nutzen. 
Für eine derartige Kombination von \ac{SOA} und \ac{EDA} wird häufig die Bezeichnung Event-Driven SOA genutzt.
In dieser Ausprägung können Ereignisse den Aufruf von digitalen Diensten auslösen, woraufhin im Verlauf, deren Ausführung weitere Ereignisse generiert werden, die parallel von einem Ereignisverarbeitungsdienst analysiert und verarbeitet werden. 
\cite{Bruns.2010}

Im Zusammenhang mit der Ereignisverarbeitung in Unternehmensanwendungen wird der Begriff Echtzeit, englisch real-time, häufig verwendet, um die Aktualität von verarbeiteten Ereignissen zu hervorzuheben.
\cite{Bruns.2015}
Echtzeit wird jedoch nicht im Sinne der Informatik verstanden, wonach für jede Funktion die exakte Einhaltung einer restriktiv vorgeschriebenen Antwortzeit erforderlich ist, um die Gegenwartsbezogenheit und Vorhersagbarkeit von Informationssystemen sicherzustellen.
Vielmehr dominiert ein betriebswirtschaftliches Verständnis, das darauf Bedacht ist, in allen relevanten Informationssystemen aktuelle Informationen bereitzustellen, ohne spezifizierte Leistungskennzahlen zur Erfüllung von Echtzeit zu formulieren.
\cite{Worn.2005}
Unter Echtzeit versteht man aus dieser Sicht daher die möglichst zeitnahe Verfügbarkeit, Transparenz und Nutzbarkeit von Informationen ohne unnötige Zeitverzögerung.
Eingehende Ereignisse werden somit im Rahmen der Ereignisverarbeitung analysiert, sobald sie auftreten, um geeignete Reaktionen in Echtzeit auslösen zu können.
\cite{Grauer.2010}

\subsection{Merkmale und Kriterien}
Die \ac{EDA} wird, wie schon erwähnt, im Wesentlichen durch die drei Grundschritte Erkennen, Verarbeiten und Reagieren charakterisiert, woraus sich drei logische architektonische Schichten für eine Ereignisverarbeitung ergeben: Ereignisquellen, Ereignisverarbeitung und Ereignisbehandlung. Das
Zusammenspiel dieser drei Schichten wird  in Abbildung \ref{fig:Grundlegender architektonischer Entwurf einer EDA} veranschaulicht.

\begin{figure}[H]
	\centering 
    \includegraphics[width=\textwidth]{img/Ereignisverarbitungablauf.png}	
    \caption[Grundlegender architektonischer Entwurf einer EDA]
    {Grundlegender architektonischer Entwurf einer EDA}
    \label{fig:Grundlegender architektonischer Entwurf einer EDA}
\end{figure}
\footnotetext{in Anlehnung an \citeauthor{Metz.2014} \citeyear{Metz.2014} \cite{Metz.2014} }

\todo{abbildung erkären}

\todo{Formulierung der Ereignisverarbeitungsregeln}

\missingfigure{Kriterien für Funktionalitäten der Ereignisverarbeitung}

\subsection{Bewertung von Funktionalitäten der Ereignisverarbeitung}

\todo{Referenzieren auf \cite{Vidackovic.2010}}

\missingfigure{Tabelle  Bewertung von Funktionalitäten der Ereignisverarbeitung}


\section{Dynamische Geschäftsprozesse auf Basis von Ereignisverarbeitung}\label{sec:Kombi}
Bei der Automatisierung von dynamischen Geschäftsprozessen wächst die Notwendigkeit, auf kritische Ereignisse in Echtzeit und ohne Latenzzeiten zu reagieren. 
Die Integration von Ereignisverarbeitungskonzepten in die Konzeption dynamischer Geschäftsprozesse ist ein geeignetes Mittel, um den steigenden Anforderungen an das Echtzeit-Management von Geschäftsprozessen unter Berücksichtigung relevanter Ereignisse zur Laufzeit gerecht zu werden. 
\cite{Abolhassan.2016}
Die Vorteile für ein Unternehmen bei einer solchen Vervollständigung der Geschäftsprozessautomatisierung mit Konzepten der Ereignisverarbeitung gründen sich im Wesentlichen auf die folgenden Merkmale:

\begin{itemize}
    \item 
    Identifizierung relevanter oder kritischer Situationen für den Geschäftsprozess unter Berücksichtigung externer Ereignisse aus dem Geschäftsumfeld und interner Ereignisse aus dem Geschäftsprozess.
    \item 
    Fähigkeit, sofort auf sich ändernde Situationen zu reagieren, indem Ereignisse in Echtzeit verarbeitet werden.
    \item
    Möglichkeit der Trennung der Ereignisverarbeitungslogik von der Geschäftsprozesslogik durch lose Kopplung und Kommunikation über Ereignisobjekte.
    \item
    Gute Unterstützung von verteilten Umgebungen, die insbesondere in unternehmensübergreifenden Geschäftsprozessnetzwerken und \ac{SOA}-Umgebungen eine wichtige Rolle spielen.
\end{itemize}

Es existieren bereits erste Forschungsansätze mit dem Ziel der Anreicherung von Geschäftsprozessmodellen mit Konzepten der Ereignisverarbeitung, die unter der englischen Bezeichnung Event-Driven Business Process Management zusammengefasst werden. Ausgewählte Konzepte aus diesem Bereich werden im Folgenden vorgestellt und anschließend anhand wesentlicher Kriterien gegenübergestellt.

\subsection{Merkmale und Kriterien}

Dynamische und somit ereignisorientierte Geschäftsprozesse operieren prinzipiell auf zwei Ebenen, der Geschäftsprozessebene und der Ereignisverarbeitungsebene. 
Diese erfüllen ihre Aufgaben in erster Linie separat und in paralleler Weise, kommunizieren allerdings mittels des Austauschs von Ereignissen miteinander, was in diesem Kontext den maßgeblichen Aspekt darstellt. 
Abbildung \ref{fig:Ebenen dynamischer Geschäftsprozesse} illustriert in schematischer Darstellung die Zusammenhänge.

Auf der Ereignisverarbeitungsebene werden eingehende Ereignisse aus diversen externen und internen Ereignisquellen kontinuierlich analysiert, wobei zudem Ereignisse generiert werden können, welche wiederum dieselbe Analyse durchlaufen. 
Da zwischen der Ereignisverarbeitungsebene und der Geschäftsprozessebene mit Ereignissen kommuniziert werden kann, können diese generierten Ereignisse als unmittelbare Reaktion eine dynamische Wirkung auf den Ablauf des laufenden Geschäftsprozess ausüben.
Der Geschäftsprozess fungiert demnach einerseits als eine der internen Ereignisquellen für die Ereignisverarbeitungsebene und andererseits als wesentlicher Ereigniskonsument der generierten Ereignisse zur Adaption des Prozessablaufs.
\cite{Benker.2016}

Auf der Geschäftsprozessebene erfolgt der operative und automatisierte Ablauf von Aktivitäten des Geschäftsprozesses, wobei Letztere in einer \ac{SOA} überwiegend auf elektronischen Diensten basieren, die als Webservices über standardisierte Schnittstellen aufgerufen werden und auch über Unternehmensgrenzen hinweg verteilt sein können.
\cite{Finger.2009}
Bei den Aktivitäten kann es sich darüber hinaus auch um manuelle Benutzeraufgaben handeln, die zwar von Menschen ausgeführt werden, aber dennoch mittels informationstechnischen Schnittstellen in einen automatisierten Prozessablauf integriert werden können.
\cite{Bruns.2010}

\begin{figure}[H]
	\centering 
    \includegraphics[width=\textwidth]{img/dynamicbp.png}	
    \caption[Ebenen dynamischer Geschäftsprozesse]
    {Ebenen dynamischer Geschäftsprozesse \protect\footnotemark}
    \label{fig:Ebenen dynamischer Geschäftsprozesse}
\end{figure}
\footnotetext{in Anlehnung an \citeauthor{Vidackovic.2014} \citeyear{Vidackovic.2014} \cite{Vidackovic.2014} }

In ereignisorientierten im Gegensatz zu ablauforientierten Geschäftsprozessen üben Ereignisse zur Laufzeit einen signifikanten Einfluss auf den Prozessablauf aus, sodass die Geschäftsprozesse anhand geeigneter Ereignisverarbeitungsregeln mit dynamischen Eigenschaften ausgestattet werden können. 

\subsection{Gegenüberstellung vorhandener Forschungsansätze}
Die wesentlichen Aspekte der existierenden Forschungsansätze werden zunächst einzeln betrachtet und anschließend gegenübergestellt.

\paragraph{Entwicklung agiler Geschäftsprozesse mit Ereignisverarbeitung}
Dieser Forschungsansatz von \citeauthor{Alexopoulou.2008} aus dem Jahr \citeyear{Alexopoulou.2008} liefert eine Vorgehensweise zur Entwicklung dynamischer und ereignisgesteuerter Geschäftsprozesse, die nicht in Form eines abgeschlossenen Systems modelliert werden, sondern lediglich aus einzelnen Ereignis-Aktivität-Einheiten, sogenannten Dynamikeinheiten, bestehen. 
\cite{Alexopoulou.2008} 
Diese werden in Echtzeit auf der Ereignisverabreitungsebene ausgeführt und fügen den Geschäftsprozess dynamisch zusammen, wobei neue Ereignis-Aktivität-Einheiten bedarfsgerecht hinzugefügt werden können. 

Ein Defizit dieses Forschungsansatzes ist die fehlende Betrachtung des gesamten Ablaufs. Das Konzept dieser Dynamikeinheiten wird auch in dieser Bachelorarbeiten aufgegriffen, allerdings ausgehend von einem in seiner Gesamtheit modellierten Geschäftsprozess, der in diese Einheiten zerlegt wird.

\paragraph{RESTful SOA zur Automatisierung von Geschäftsprozessen}
Die Realisierung von \ac{SOA} in diesem Forschungsansatz von \citeauthor{Wolf.2016} ist in Übereinstimmung mit den Prinzipien von \ac{REST}. 
\footnote{
Da der technische Fokus in dieser Arbeit auf der Geschäftsprozessebene liegt, werden die einzelnen technischen Konzepte an dieser Stelle nicht weiter vertieft.
Für eine detaillierte Betrachtung der Prinzipien von \ac{REST} sei lediglich auf weiterführende Literatur verwiesen.
\cite{Wolf.2016}\cite{Masak.2007}\cite{Finger.2009}}
Die \ac{REST}ful \ac{SOA} entsteht durch den Entwurf serviceorientierter Architekturen gemäß den Bedingungen und Technologien von \ac{REST}. \ac{REST} ist ein Architekturstil für die Gestaltung verteilter Systeme, insbesondere bei der Umsetzung von Webservices. Durch dieses Vorgehen wird auf eine modellbasierte Spezifikation von \ac{REST}ful \ac{SOA} für die Automatisierung von Geschäftsprozessen abgezielt. \cite{Wolf.2016}

Insbesondere die Überwindung der semantischen Lücke zwischen Geschäftsprozessmodell und Anwendungssystem durch ein systematisches Vorgehen und die Sicherstellung einer konsistenten Koordination von Anwendungssystemen mit den Geschäftsprozessmodellen wird in diesem Forschungsansatz behandelt. Die identifizierten Probleme werden durch einen Ansatz zur Konzeption eines Anwendungssystems auf Basis von auf fachlicher Ebene modellierten Geschäftsprozessen angegangen. 

Es fehlen allerdings Möglichkeiten zur Modellierung wesentlicher Konzepte der Echtzeitverarbeitung von Ereignissen.
Da der Forschungsansatz wohl aber auf die automatisierte Ausführung von Geschäftsprozessen mithilfe von digitalen Diensten ausgerichtet ist, bleibt der Ansatz für die softwaretechnische Implementierung in Kapitel von Relevanz. 

\paragraph{SOEDA-Methode}
SOEDA beschreibt ein Verfahren zur Entwicklung von Anwendungen durch die Zusammenführung von serviceorientierten (SOA) und ereignisgesteuerten Architekturen (EDA). Geschäftsprozesse werden hier mit EPK modelliert und in WS-BPEL umgewandelt. Mit Hilfe einer rudimentären Ereigniskopplungstechnik werden komplexe Ereignisse mittels grafischer und nicht-formaler Modellierung vom Geschäftsprozessmodell auf die zugrunde liegenden singulären Ereignisse heruntergebrochen, so dass ein Entwickler die Regeln der Ereignisverarbeitung anschließend manuell in einer spezifischen Ereignisverarbeitungssprache implementieren kann. Schließlich wird das Geschäftsprozessmodell durch Hinzufügen technischer Details zum WS-BPEL-Modell ausführbar.

Schwächen der SOEDA-Methode sind die unzureichende Modellierung von Ereignisverarbeitungsregeln und das Fehlen einer automatischen Transformation in ein ausführbares Ereignisverarbeitungsmodell. Auch fehlen dynamische Anpassungen des Geschäftsprozesses zur Laufzeit, da Ereignisse nur als Teil des normalen Geschäftsprozessablaufs betrachtet werden. Mit der Version 2.0 verfügt die BPMN nun nicht nur über eine grafische Notation, sondern auch über die Semantik der technischen Ausführung und kann somit direkt in einer Engine ausgeführt werden. Damit scheint der Umweg einer Transformation vom EPC zum WS-PEL nicht mehr notwendig zu sein, zumal eine Konsistenz des Geschäftsprozessmodells auf betriebswirtschaftlicher und technischer Ebene wünschenswert ist.
\cite{MatthiasWieland.2009} 
\cite{Bruns.2010}S.37
\cite{RobraBissantz.2009}
% SOEDA bezeichnet eine Methode zur Entwicklung von Anwendungen durch die Verschmelzung aus dienstorientierten (SOA) und ereignisgesteuerten Architekturen (EDA). Geschäftsprozesse werden hier mit EPK modelliert und in WS-BPEL transformiert. Mithilfe einer rudimentären Technik zur Ereignisverknüpfung werden die komplexen Ereignisse aus dem Geschäftsprozessmodell mittels grafischer und nicht-formaler Modellierung auf die zugrunde liegenden singulären Ereignisse her- untergebrochen, sodass anschließend ein Entwickler die Ereignisverarbeitungsregeln von Hand in einer Engine-spezifischen Ereignisverarbeitungssprache implementieren kann. Durch Hinzufügen technischer Details zum WS-BPEL-Modell wird das Geschäftsprozessmodell schließlich ausführbar.

% Schwächen der SOEDA-Methode liegen in der unzulänglichen Modellierung von Ereignisverarbeitungsregeln und dem Fehlen einer automatischen Transformation in ein ausführbares Ereignis- verarbeitungsmodell. Auch mangelt es an dynamischen Anpassungen des Geschäftsprozesses zur Laufzeit, da Ereignisse lediglich als Bestandteil des regulären Geschäftsprozessablaufs betrachtet werden. Ab Version 2.0 verfügt die BPMN neben einer grafischen Notation nun auch über die tech- nische Ausführungssemantik und ist somit direkt in einer Engine ausführbar. Somit erscheint der Umweg einer Transformation von EPK zu WS-BPEL als nicht mehr erforderlich, zumal eine Konsistenz im Geschäftsprozessmodell auf Geschäfts- und technischer Ebene erstrebenswert ist.



\paragraph{Gegenüberstellung der Ansätze}

\todo{}
\section{Zusammenfassung der wesentlichen Erkenntnisse und Defizite}


\todo{Anpassen}

Resümierend sind im Wesentlichen folgende Erkenntnisse festzuhalten, die für die betrachtete Pro-
blemstellung von Relevanz sind:
\begin{itemize}
	\item This is a bullet point.
    \item This is another one.
\end{itemize}

Diesen Erkenntnissen kann durch den aktuellen Stand der Wissenschaft und Technik nicht in
reichendem Maße Rechnung getragen werden, da maßgebliche Defizite vorhanden sind:
\begin{itemize}
	\item This is a bullet point.
    \item This is another one.
\end{itemize}
\chapter{Bezugsrahmen und Anforderungen}

\section{Management von Betriebsdaten im Industriebetrieb}\label{sec:usecase}

\todo{Aufgaben im Industriebetrieb}

\todo{ERP System und SAP S/4HANA erklären}

\missingfigure{Wertschöpfungskette}

\subsection{Geschäftsprozesse in der Produktionsplanung und -steuerung}

\missingfigure{Prozessüberblick}

\subsection{Vorstellung der Fertigungsdurchführung in der diskreten Fertigung}

\todo{Definition Fertigungsdurchführung}

\todo{Diskrete Fertigung}

\missingfigure{Prozessüberblick}


\section{Untersuchung der Fertigungsdurchführung}\label{sec:untersuchung}

\todo{Aufgaben nach \cite{Seidlmeier.2015} fallsstudie}

\todo{Einleitung der Methodik der Experteninterviews}
Aus dem der Einleitung ist bekannt, dass in dieser Bachelorarbeit speziell im Geschäftsprozess der Fertigungsrückmeldung nach automatisierbaren Aktivitäten gesucht werden soll. 
Nachdem in Kapitel \ref{ch:Grundlagen} die Automatisierung von Geschäftsprozessen mittels Ereignisverarbeitung allgemein betrachtet wurden, wird die Fertigungsdurchführung als exemplarischer Untersuchungsgegenstand dieser Bachelorarbeit nun im Kontext der Geschäftsprozessanalyse betrachtet. 

\subsection{Methodisches Vorgehen zum Experteninterview}

\todo{Methodisches Vorgehen zum Experteninterview}
Mithilfe von Experteninterviews sollen alle als Standard geltenden Aufgaben im entsprechenden Geschäftsprozess empirisch erhoben werden, sodass aus den korrespondierenden Informationen Kriterien für die zu findenden Maßnahmen zur Automatisierung aufgestellt werden können. 
Da Experteninterviews in dieser Bachelorarbeit keineswegs die einzige Methode darstellen, sondern lediglich zusätzliches Wissen liefern, ist eine ausgewählte Stichprobe der für das Thema relevanten Perspektiven im vorliegenden Fall ausreichend.

Das Wissen darüber, welche Aufgaben in der Fertigungsdurchführung allgemein sowie mit SAP S/4HANA Cloud als unabkömmlich gelten und welche Merkmale und Defizite diese aufweisen, ist extrem unternehmensspezifisch und kann daher ausschließlich durch ausreichend Erfahrung generalisiert werden. 
Zugleich ist es in einschlägiger Literatur nicht auffindbar.
Die ausgewählten Experten sollten daher Repräsentanten verschiedener Perspektiven auf den Geschäftsprozess darstellen, sodass die Ergebnisse möglichst generalisierbar sind. 
Bei der Auswahl der Experteninterviews wurden folgende Perspektiven beachtet:

\begin{itemize}
    \item 
    \textbf{Branchenexperte}: Dieser vertritt die Rolle des externen Stakeholders. Er hat ein eigenes Interesse am Ablauf des Geschäftsprozesses und war selbst schon in der Planung und Durchführung des Geschäftsprozesses beteiligt. 
    \item
    \textbf{Prozessexperte}: Als Prozessexperte wird im Kontext dieser Bachelorarbeit eine Person verstanden, zu deren Kerngebieten der ausgewählte Geschäftsprozess gezählt wird. Er arbeitet zusammen mit Kunden im SAP-Umfeld an der Einführung und Optimierung des Geschäftsprozesses
    \item
    \textbf{Technologieexperte}:  Ein Technologieexperte beschäftigt sich eingehend mit den technologischen Hilfsmitteln des Geschäftsprozesses, sowie mit deren Verwendung und Implementierung in SAP S/4HANA Cloud.
\end{itemize}

Bei empirischen Untersuchungen ist des Weitere  zwischen qualitativen und quantitativen Untersuchungen zu unterscheiden. Während qualitative Verfahren die Gemeinsamkeiten von mehreren Gegebenheiten untersuchen indem existierende Unterschiede überwunden werden, erfasst die quantitative Sozialforschung Unterschiede auf Vergleichsbasis von Gemeinsamkeiten. Für die empirische Untersuchung der Fertigungsdurchführung auf Möglichkeiten zur Integration von Betriebswirtschaft und Informationstechnologie wird die qualitative Methode gewählt. 

Damit alle Informationen ordnungsgemäß und nachvollziehbar erhoben werden und alle Erkenntnisse wissenschaftlich fundiert sind, werden die Experteninterviews nach dem untenstehenden Vorgehen strukturiert durchgeführt.

\begin{enumerate}
    \item 
    \textbf{Identifikation eines Experten}: Nur eine Person, die ausreichend Erfahrung in dem Gebiet der Fertigungsdurchführung gesammelt hat, ist in der Lage, Merkmale und Defizite in den Aktivitäten dieses Geschäftsprozesses zu erkennen und zu benennen. Es werden also Personen benötigt, die spezifisches Rollenwissen besitzen und somit als kompetent und erfahren angesehen wird.
    \item
    \textbf{Entwicklung eines Fragebogens}: Die Güte und Verwendbarkeit eines Experteninterviews hängt maßgeblich von dem Fragebogen ab. Die Fragen für diesen werden deswegen mit großer Sorgfalt gesammelt, woraufhin sie von dem Autor dieser Bachelorarbeit präzise geprüft und sortiert werden. Die Fragen werden so formuliert, dass sie bewusst sehr präzise Antworten provozieren.
    \item
    \textbf{Formung und Vorstrukturierung des Interviewablaufs}: Zu Beginn der Experteninterviews werden formale Informationen akquiriert und Wünsche der Experten entgegengenommen. 
    Für die Experteninterviews wird jeweils eine Zeitspanne von 30 bis 60 Minuten angesetzt. Als Interviewer fungiert der Autor dieser Bachelorarbeit.
    \item
    \textbf{Gestaltung der an dem Interview beteiligten Rollen}: Zwischen den Experten und dem Interviewer liegt augrund der Erfahrung und der Vorkenntnisse eine wissensbezogene Disparität zugunsten der Experten vor. Um also möglichst viele Informationen von der Expertin zu erhalten, gewährt der Interviewer den Experten einen maßgeblich größeren Teil der gesamten Redezeit.
    \item
   \textbf{ Durchführung des Experteninterviews}: Die Experteninterviews wird werden online via Skype durchgeführt. Nachdem eine Frage gestellt wird, erhalten die Experten eine variable Antwortzeit. Gemäß des Fragenkatalogs geben Experten die gewünschten Informationen an den Interviewer weiter.
    \item
    \textbf{Nachbereitung und Transkription des Experteninterviews}: Die Notizen aus dem Experteninterview werden inhaltlich, sprachlich und strukturell aufbereitet. Die Transkription erfolgt summarisch. Die summarischen Transkriptionen sind im Anhang \ref{ah:protokolle} abgelegt.
    
\end{enumerate}
\todo{Dokumentation des konkreten Vorgehens }

\subsection{Interpretation der Ergebnisse aus den Experteninterviews}

\todo{WOEZBERGER s.24}

\cite{Benker.2016}s.147

\todo{Beschreibung des Ablaufs}

\todo{Paragraphen für jeden Aspekt}

\todo{Rückgriff auf das Experteninterview}


\subsection{Merkmale und Defizite der Fertigungsdurchführung}

\todo{Relevante Daten für Rückmeldung}

\todo{Daten und Dokumente während der Rückmeldung}

\missingfigure{ER-Diagramm}

\missingfigure{medienbruch nach \cite{Hoppe.2011}}

\todo{2 der Experteninterviews in papierform/ einmal mit Terminal}

\todo{Datenübertragungstechniken}

\todo{Verschwendungen im Prozess}

\missingfigure{IST-Diagramm}

\todo{Was läuft schon automatisch?}



\missingfigure{ darstellung prozess \cite{Beier.2016} s.330}

\todo{Wo Kann automatisiert werden?}




\chapter{Konzeption der Methode}

\section{Verfahrensweisen und wesentliche Gestaltungsprinzipien}\label{sec:methodenGrundlage}
Um eine einheitliche Auffassung der Charakteristika dynamischer Geschäftsprozesse zu beschrieben, wird der inhaltliche Rahmen der Bachelorarbeit zunächst dargelegt.
Da sich die Entwicklung der Methode an fundamentalen Ansätzen der Softwareentwicklung orientiert, werden nachfolgend dessen Grundlagen dargestellt, um die wesentlichen Gestaltungselemente der Methode in Erfahrung bringen, bevor die Konzepte von Lean Management charakterisiert werden, auf die sich die Methode insbesondere bei der Modellierung auf der Geschäftsprozessebene stützt.

\subsection{Charakterisierung dynamischer Geschäftsprozesse}
Die Auffassung dynamischer Geschäftsprozesse im Rahmen dieser Bachelorarbeit versteht diese so, dass sich deren Ablauf, als unmittelbare und automatisierte Reaktion auf Ereignisse, in Echtzeit anpassen kann. Gemäß Abbildung \ref{fig:Grundlegender architektonischer Entwurf einer EDA} werden die zugehörigen Ereignisverarbeitungsregeln zur Echtzeitverarbeitung der relevanten Ereignisse und deren Beziehungen untereinander ausschließlich auf der Ereignisverarbeitungsebene implementiert, während die Kommunikation mit der Geschäftsprozessebene einzig mittels Ereignissen durch einen Ereignisverarbeitungsdienst abläuft.
\cite{Vidackovic.2014}

Die unmittelbare Reaktion auf ein solches Ereignis auf Geschäftsprozessebene soll sich im Rahmen der Modellierungsmöglichkeiten von \ac{BPMN} 2.0 bewegen, sodass der Fokus hier auf der Modellierung automatisierbarer Geschäftsprozesse in der Entwurfsphase liegt mit der wesentlichen Eigenschaft, auf eintretende Ereignisse möglichst dynamisch reagieren zu können. Die Dynamik solcher ereignisgesteuerter Geschäftsprozesse basiert demnach auf der Ereignisverarbeitung auf der Ereignisverarbeitungsebene, während der Ablauf der Geschäftsprozesse mit ausreichender Flexibilität auf der Geschäftsprozessebene ausgestattet sein muss, um die von der Ereignisverarbeitung angestoßenen Aktivitäten als Reaktion in Echtzeit aufgetretenen Ereignisse realisieren zu können.

\subsection{Grundlagen der Softwareentwicklung}
Ein wesentliches Ziel dieser Bachelorarbeit ist die Umsetzung des konzipierten Geschäftsprozesses in Form von Software. Daher wird an dieser Stelle zunächst der Begriff Software selbst spezifiziert.

Laut \citeauthor{Balzert.2009} umfasst der Begriff \textit{Software} sowohl die Programme und Daten, als auch entsprechende Dokumentationen, die zur Anwendung der Software benötigt werden. Des Weiteren beschreibt er ein \textit{Softwaresystem}, das ein wesentlicher Bestandteil eines jeden modernen Informationssystems ist, als eine Ansammlung von Komponenten und Elementen, die wiederum aus Software bestehen. Softwaresysteme werden noch weiter unterteilt in Anwendungssoftware und Systemsoftware, wobei Systemsoftware grundsätzlich das Betriebssystem, in der Praxis aber auch den Compiler, Datenbanken, Kommunikationsprogramme und Dienstprogramme umfasst.
\cite{Balzert.2009}
Anwendungssoftware ist Software, die Tätigkeiten eines Anwenders durch die Unterstützung von \ac{IT} unterstützen sollen. 
Anwendungssoftware nutzt in der Regel wesentliche Funktionalitäten der zugrunde liegenden Systemsoftware. Der Fokus der Softwareentwicklung in der vorliegenden Bachelorareit liegt auf der Entwicklung von Anwendungssoftware. 
\cite{Balzert.2009}
Die Softwareentwicklung als Ganzes betrachtet orientiert sich an der Zielsetzung, Softwaresysteme durch den systematischen Einsatz von Prinzipien, Vorgehensweisen und Werkzeugen für die Herstellung und Anwendung von Software.
\cite{Balzert.2009}
Bei der Softwareentwicklung, englisch \textit{Software-Engineering}, handelt es sich folglich, um eine ingenieurmäßige Vorgehensweise, da es sie eine die Schaffung von möglichst erffektiven sowie effizienten Lösungen für konkrete und praktische Problemstellungen.
\cite{Balzert.2009}

Im Laufe der Zeit sind die Softwaresysteme weit vorangeschritten, wodurch auch die Komplexität im den Softwareentwicklungsprojekten stetig wächst. Heute sind iterative und agile Eigenschaften in einem Vorgehensmodell für komplexe Softwaresysteme kaum wegzudenken. Beispiele hierzu sind das \textit{Wasserfallmodell, V-Modell, Extreme Programming oder Scrum}, da diese Vorgehensmodelle jedoch zu komplex für eine softwaretechnische Implementierung im Rahmen dieser Bachelorarbeit sind, wird an dieser Stelle von einer expliziten Betrachtung der einzelnen Modelle abgesehen und der Ansatz verfolgt die notwendigen Gemeinsamkeiten darzustellen.
\cite{Krypczyk.2018}

Im Wesentlichen umfasst ein jedes Vorgehensmodell zur Entwicklung von Software die drei, in Abbildung \ref{fig:Phasen der Softwareentwicklung} dargestellten, Phasen: \textit{Analyse, Entwurf und Implementierung}

\begin{figure}[H]
	\centering 
    \includegraphics[width=\textwidth]{img/entwicklung.png}	
    \caption[Phasen der Softwareentwicklung]
    {Phasen der Softwareentwicklung\protect\footnotemark}
    \label{fig:Phasen der Softwareentwicklung}
\end{figure}
\footnotetext{Eigene Darstellung}
\footnotetext{Die Abbildung dient lediglich der Visualisierung und ist nicht \ac{BPMN} 2.0 konform.}

Die Analysephase beschäftigt sich eingehend damit welche erfolgskritischen Anforderungen an die zu erstellende Software existieren. Dabei erfolgt eine explizite Betrachtung der Ausgangslage, um die Probleme, die es mit Hilfe der Software zu lösen oder zu unterstützen gilt, zu identifizieren. Sind die wesentlichen Anforderungen zusammengetragen, werden Maßnahmen zur Problembehandlung erörtert und ausgewählt. Anschließend erfolgt der grundlegende Architekturentwurf der Software. In der letzten Phase, der Implementierung, wird die entworfene Software letzendlich in eine ausführbare Version transformiert. Die Aufgaben der Implementierung umfassen die Erstellung von Benutzerschnittstellen, die Erarbeitung der notwendigen Geschäftslogik und die Umsetzung der Datenhaltung in Form von Datenbanken oder Schnitttstellen zu weiterführender Software. Diese Phasen können bei Bedarf beliebig oft wiederholt werden.
\cite{Krypczyk.2018}

Als bewährte Methode für diesen Minimalansatz der Softwareentwicklung wird im Rahmen dieser Bachelorarbeit das \textit{Design-Thinking} herangezogen. 
\cite{Elsner.2018}
Design-Thinking ist eine Methode, um zielorientiert Lösungen zu finden, die aus Anwendersicht gewünscht sind. Die Phasen des Design-Thinking-Prozesses heißen \textit{Discover, Design und Deliver}, die als Synonym zu den Phasen \textit{Analyse, Entwurf und Implementierung} angesehen werden dürfen. Ziel dieser Form der Softwareentwicklung ist es möglichst zeitnahe an einen funktionsfähiges Produkts zu gelangen.

Im Zusammenhang mit dieserart Vorgehensmodellen hört man daher oft den Begriff \ac{MVP}. Ein \ac{MVP}, wörtlich ein minimales, überlebensfähiges Produkt, ist die erste Version eines Produkts, die entwickelt werden muss, um mit minimalem Aufwand die Mindestanforderungen eines Bedarfs zu decken. Das \ac{MVP} muss bereits insofern funktionsfähig sein, dass es vom Anwendern getestet werden kann.
\cite{Elsner.2018}
Die in dieser Bachelorarbeit konzipierte Methode orientiert sich an den grundlegenden Phasen der Softwareentwicklung, sowie den Prinzipien von Design-Thinking. 

\subsection{Lean Management}

Lean Management beschäftigt sich mit der Effizienz und Effektivität von Geschäftsprozessen. Jede Art von Verschwendung steht der Effizienz und Effektivität entgegen und muss daher vermieden werden.
Der Begriff \textit{Lean Management} beschreibt ein System, das vielfältige Ansätze und Philosophien miteinander verbindet, um Geschäftsprozessen im gesamten Unternehmen dadurch wertschöpfender zu gestalten, dass die Verschwendung von Ressourcen erkannt und eliminiert wird.
Lean Management beinhaltet sowohl organisatorische Prinzipien, operative Strategien zur Erfüllung des Unternehmensziele, betriebswirtschaftliche Maßnahmen, aber auch Vorgaben an die einzelnen Angestellten.
\cite{Schell.2017}

Der Ursprung der heutigen Philosophie des \textit{Lean Management} liegt in Japan. Im wirtschaftlich angeschlagenen Japan der Nachkriegszeit des Zweiten Weltkriegs herrschte vor allem in den Industrieunternehmen ein akuter Ressourcenmangel, da Japan in diesen Jahren keine Wirtschaftshilfe von anderen Ländern erhielt. In dieser Notsituation mussten die Industrieunternehmen ihre Ressourcen schonend einsetzen, Verschwendung vermeiden, Prozesse optimieren und gleichzeitig die für den Markt notwendige Qualität liefern.
Mit dieser Herausforderung konfrontiert, entstand bei der Firma Toyota das \textit{Toyota-Produktionssystem}, der Vorreiter von Lean Management.
\cite{Morgan.2006}

Grundsätzlich werden unter Verschwendung alle Aktivitäten und Ressourcen verstanden, die nicht zur Wertschöpfung eines Industriesbetriebs beitragen. Diese nicht wertschöpfenden Tätigkeiten werden als überflüssig angesehen und sind zu eliminieren.
Das \textit{Toyota-Produktionssystem} kennt 7 Arten der Verschwendung. Die unterschiedlichen Arten von Verschwendung werden in Tabelle \ref{tab:Arten von Verschwendung} dargestellt.

\begin{table}[H]
	\centering
	\begin{tabularx}{\textwidth}{l X} 
		\toprule
		\textbf{Art}  &   
		\textbf{Beschreibung}  \\ 		\midrule
		
		Bewegung &   
		Lange Laufwege und umständliche Aktivitäten  \\  \cmidrule(r){1-1} \cmidrule(r){2-2}
		
		Defekte &   
		Qualitätsmängel der Produkte \\ \cmidrule(r){1-1} \cmidrule(r){2-2}
		
		Lagerbestand &   
		Umlaufbestände oder Lager mit fertigen Produkten  \\ \cmidrule(r){1-1} \cmidrule(r){2-2}
		
		Transport &   
		Häufige Transporte von Ressourcen  \\ \cmidrule(r){1-1} \cmidrule(r){2-2}
		
		Überproduktion &   
	    Herstellung nicht benötigter Produkte \\ \cmidrule(r){1-1} \cmidrule(r){2-2}
		
		Unnötige Prozesse &   
		Prozesse oder Fertigungsverfahren ohne Notwendigkeit  \\ \cmidrule(r){1-1} \cmidrule(r){2-2}
		
		Warten &   
		Verzögerung bei den Tätigkeiten  \\
	    \bottomrule
	\end{tabularx}
	\caption{\label{tab:Arten von Verschwendung}Arten von Verschwendung}
\end{table}

\newpage

Aus der Lean-Philosophie lassen sich fünf wesentliche Prinzipien ableiten, die das Fundament für den Gebrauch von Lean Management bilden: \textit{Kundennutzen, Wertstrom, Fluss-Prinzip, Pull-Prinzip und Null Fehler}\begin{itemize}
    \item 
    Der \textbf{Kundennutzen} im Mittelpunkt stellt in der Lean-Philosophie eine fundamentale Leitlinie dar. 
    Es ist somit notwendig, mit den Kunden im Dialog zustehen, um die Bedürfnisse dieser verstehen zu können.
    \cite{Muller.2011} 
    \item
    Die \textbf{Wertstromanalyse} dient zur Identifizierung relevanter Stellen im Geschäftsprozess, die einen Beitrag zur Wertschöpfung leisten. Der Wertstrom wird durch alle Aktivitäten und Ereignisse gekennzeichnet, die für das Produkt notwendig ist. 
    \item
    Im Rahmen des \textbf{Fluss-Prinzips} wird, versucht die einzelnen Aktivitäten in einer optimalen Abfolge anzuordnen. Eine optimale Abfolge von Aktivitäten eines Geschäftsprozesses wird dann als nicht gegeben betrachtet, wenn es zu Wartezeiten oder Engpässen kommt.
    \item
    Das \textbf{Pull-Prinzip} beschreibt denselben Grundsatz wie die Kundeneinzelfertigung in der Produktionsplanung und -steuerung in SAP S/4HANA Cloud. Eine Fertigung darf nur erfolgen, wenn ein konkreter Bedarf durch einen Kundenauftrag existiert.
    \item
    Lean Management definiert einen Fehler als Abweichung von einem definierten Standard. Daher müssen für Geschäftsprozesse solche Toleranzgrenzen als regeln definiert werden, um mögliche Fehler und daraus abgeleitete Qualitätsmerkmale zu ermitteln. Dieser Gedanke wird als \textbf{Null-Fehler-Prinzip} bezeichnet.
    Die Verantwortung für die Fehlererfassung und -behebung liegt bei den verantwortlichen Mitarbeiter.
\end{itemize}

Die Leitgedanken von Lean Management dienen in der vorliegenden Bachelorarbeit als Gedankenstütze bei der Identifizierung geeigneter Maßnahmen gegen Verschwendung und für Automatisierung und Reaktionsfähigkeit in der Fertigungsdurchführung mit SAP S/4HANA Cloud.
\section{Darstellung der methodischen Vorgehensweise}\label{sec:methode}

\todo{einleiten}

\missingfigure{Vorgehen in der Übersicht}

\todo{erläutern}

\section{Beteiligte Rollen}\label{sec:rollen}

\todo{Optional}
\chapter{Anwendung der konzipierten Methode}
\section{Entwurf auf Geschäftsprozessebene}\label{sec:modellierung}

\missingfigure{Geamt prozess ist nach \cite{Beier.2016}s.34}

\todo{Modellierungs guidelines}

\todo{bpmn elemente}
% \todo{Auf das gewählte Modell in Punkt 4 anpassen}
% Um zu überprüfen, ob die im Rahmen dieser Arbeit angedachte Automatisierung von Testfällen möglich ist wird folgende Hypothese aufstellen: 

% \textit{Vorher manuell ausgeführte Testfälle, welche Blackbox-Verfahren anwenden, lassen sich durch Eagle Drones automatisieren und weisen dabei äquivalente Ergebnisse auf}. 

% Geprüft werden soll diese Hypothese für die im Rahmen dieser Arbeit zur Automatisierung bestimmten Testfälle. Für die Überprüfung der Hypothese wird angenommen, dass ein manueller Testfall automatisierbar ist, wenn er Bestandteil eines automatisierten Tests ist. Gestützt wird dieses Annahme auf das Prinzip der schlüsselwortgetriebenen Automatisierung (siehe Kapitel 2.5.4). Auf dieser Basis soll ein komplexer Testfall entworfen werden, der möglichst viele zu automatisierende Testfälle enthält. Ist die Automatisierung dieses komplexen Testfalls im Rahmen der prototypischen Implementierung möglich, so kann die Hypothese für die inkludierten Testfälle als verifiziert angesehen werden. Gelingt eine Automatisierung nicht, ist die Hypothese zu falsifizieren.

\subsection{Erarbeitung und Vorstellung geeigneter Maßnahmen}

Ein Medienbruch ist ein Wechsel des Mediums, das eine Information trägt. Bei- spielsweise stellt das Abschreiben einer Telefonnummer vom Handy auf einen Notizzettel und das anschließende Übertragen der Nummer von diesem Zettel in eine Datei auf dem PC einen doppelten Medienbruch dar. Da Medienbrüche Zeit kosten, Schnittstellen notwendig machen und Fehler in den übertragenen Informationen verursachen können, sollen Medienbrüche durch eine geeignete Gestaltung der Informationssysteme möglichst vermieden werden.

Die technische Prozessimplementierung geht daher einen Schritt weiter: Die Datenflüsse werden direkt an die Orte der Ausführung geleitet, und Rückmeldungen werden direkt von dort zurück in das führende ERP-System gespeist. Die technische Anbindung kann dabei beliebige Komplexitätsgrade annehmen. Scannerlösungen mit Handhelds, die Wareneingangsbuchungen und Entnahmebuchungen automatisiert per Hallenfunk an der Zentralsystem zurückmelden, sind in den meisten Unternehmen seit vielen Jahren Standard.




Die Sicherstellung der Qualität im Produktionsprozess ist ein weiteres Merk- mal der Lean Production. Fehler müssen bei dieser Philosophie direkt im Fer- tigungsprozess erkannt und auf der Stelle beseitigt werden. Diese Vorgehens- weise wird unter Jidoka zusammengefasst (siehe hierzu auch die Definitionen in Abschnitt 1.3.7).
\todo{medienbruch nach \cite{Schell.2017}s.106 und \cite{Hoppe.2011}}
\todo{Schwachstellenidentifikation nach \cite{Kieviet.2019}{S.53}}
\todo{schwachstellen nach \cite{Gerberich.2011}{S.56}}
\todo{Maßnahmen nach \cite{Schell.2017} s.176}

\todo{Anforderungen in konkrete Maßnahmen umformen in tabellenform}

\subsection{Segmentierung und Bildung von Dynamikeinheiten}

\todo{Bildung der Beziehungen Ereignis <=> Aktivität}

\subsection{Aspekte der Ausführungssemantik}

\missingfigure{ Wesentliche Aspekte der Ausführungssemantik}
\section{Konzeption auf  Ereignisverarbeitungsebene}\label{sec:ereignismodell}
Diesesr Abschnitt behandelt die technische Sichtweise auf die Ereignisverarbeitungsebene, die mittels Echtzeitverarbeitung von zur Laufzeit auftretenden Ereignissen die Dynamik für den Ablauf der Geschäftsprozessebene liefert. Für die digitale Verarbeitung der Ereignisse ist eine explizite Spezifikation des Ereignismodells nötig. Im Sinne einer softwaretechnischen Implementierung wird darüber hinaus der Ereignisverarbeitungsdienst konzipert.

Als Spezifikation einer allgemeinen Beschreibung von Ereignisobjekten dient die \textit{CloudEvents}-Spezifikation in der Version 0.3-wip. 
\textit{CloudEvents} bietet eine standardisierte Definition der Struktur und Metadatenbeschreibung von Ereignissen. Diese Definition definiert, wie die in der \textit{CloudEvents}-Spezifikation definierten Elemente im \ac{JSON}-Format dargestellt werden sollen. \cite{CloudEvents.2019}

Im weiteren Verlauf dieser Bachelorarbeit wird nun auch spezifischer auf eine Umsetzung mit SAP S/4HANA Cloud eingegangen. SAP S/4HANA Cloud bietet für die Behandlung von verschiedenen vordefinierten Geschäftsereignissen eigene Dienste zur Integration an. Der auf der SAP Cloud Platform basierte \textit{SAP Enterprise Messaging-Service} bietet Funktionalitäten für die Integration in SAP S/4HANA Cloud, um Ereignisse zu verbreiten und reaktive Geschäftsprozesse über Unternehmenslandschaften hinweg zu ermöglichen.
\cite{Schneider.2018}

\subsection{Spezifizierung des Ereignismodells}
Ein Ereignisobjekt repräsentiert ein in der Realität eingetretenes Ereignis. Es muss somit alle relevanten Informationen beinhalten, die für eine anschließende Verarbeitung dieses Ereignisses notwendig sind.

Unterschieden wird dabei in verschiedene Ereignistypen. Ein Ereignistyp setzt sich laut der \textit{CloudEvents}-Spezifikation aus dem betroffenen Objekt und der Art der Veränderung zusammen. Einfach ausgedrückt werden durch ein Ereignisobjekt die folgenden vier Fragen beantwortet: \textit{Wann hat es sich ereignet? Was hat sich ereignet? Wer war betroffen?}
Im Kontext dieser Bachelorarbeit wird die Antwort auf die erste Frage als der Zeitpunkt des Ereignissen verstanden. Die zweite Frage zielt auf die Veränderung eines Objektes ab, sie enthält somit die Antwort welcher Objekttyp verändert wurde und wie diese Veränderung aussieht. Die letzte Frage identifiziert schließlich ein eindeutiges Objekt des veränderten Objekttyps. \cite{CloudEvents.2019}

Beim Erstellen eines Ereignisobjektes im Rahmen dieser Bachelorarbeit müssen demnach mindestens die folgenden Daten an den Konstruktor der Ereignisgenerierung übergeben werden:
\begin{itemize}
    \item \textbf{Zeitstempel:} Er beschreibt wann das Ereignis ausgelöst wurde. 
    \item \textbf{Objekttyp:} Er spezifiziert die Klasse des veränderten Objekts. 
    \item \textbf{Ereignistyp:} Er definiert die Art und Weise wie ein Objekt verändert wurde. 
    \item \textbf{Objektschlüssel:} Er identifiziert das betroffene Objekt. 
\end{itemize}

Die SAP-spezifischen Geschäftsobjekte , auch \textit{SAP-Objekttyp} genannt, sind die Basis für Ereignisobjekte innerhalb von SAP S/4HANA Cloud. 
Ein SAP-Objekttyp kann Geschäftsereignisse mit einer eindeutigen Semantik für bestimmte Situationen definieren.
Ein Ereignis kann ausgelöst werden, wenn eine bestimmte Aktion für das Geschäftsobjekt ausgeführt wird oder wenn das Geschäftsobjekt einen bestimmten Status annimmt.
Sobald Ereignisse von Geschäftsobjekten ausgelöst werden, werden diese Ereignisse von SAP S/4HANA Cloud an ein Framework zur Ereignisverarbeitungs weitergegeben. 
\cite{Herzig.2018}
Dieses Framework bildet aus der internen technischen Darstellung des Ereignisses eine Menschen-lesbare Darstellung im \ac{JSON}-Format. Diese Darstellung, wie im Quelltext \ref{code:jsonevent}, wird der \textit{CloudEvents}-Spezifikation gerecht und enthält alle notwendigen Daten.
\cite{Herzig.2018}

\begin{algorithm}[H]
\centering 
\inputminted[linenos]{js}{code/sapevent.json}
\caption{Exemplarisches SAP-spezifisches Ereignisobjekt im JSON-Format \cite{Herzig.2018}}
\label{code:jsonevent}
\end{algorithm}

Um diese Integration zu erreichen, werden zwei Komponenten benötigt: Die allgemeine Integration in die ereignisgesteuerte Architektur von SAP S/4HANA Cloud und eine spezifische Integration mit dem SAP Enterprise Messaging-Service, der auf der SAP Cloud Platform angeboten wird. Beide Komponenten kommunizieren über Standard-Nachrichten- und Verbindungsprotokolle. Anwendungssoftware, die auf der SAP Cloud Platform basiert, kann dabei als Ereigniskonsument fungieren.

\subsection{Architektur der Ereignisverarbeitungsdienstes}
Das Schlüsselelement reaktiver Systeme ist die asynchrone nachrichtengesteuerte Kommunikation zwischen den beteiligten Anwendungssystemen. SAP S/4HANA Cloud setzt hier auf das MQTT-Protokoll. MQTT steht für Message Queuing Telemetry Transport und ist ein standardisiertes Protokoll für die Maschine-to-Maschine-Kommunikation.
Es lassen sich jedoch auch Ereignisverarbeitungsdienste nach den Konzepten des \textit{Publish-Subscribe-Paradigmas} (siehe Abschnitt \ref{sec:Ereignisverarbeitung}) damit umsetzen.
\cite{Herzig.2018}

In SAP S/4HANA Cloud werden zur Integration zwei Dienste angeboten:
\begin{itemize}
    \item Der \textbf{SAP Enterprise Messaging-Service} bietet Ereignisverarbeitungs-Funktionen mit Hilfe der SAP Cloud Platform. Der Diest umfasst die sofortige Integration in SAP S/4HANA Cloud, um Ereignisse zu verbreiten und reaktive Geschäftsprozesse über Anwendungssysteme hinweg zu ermöglichen.
    \item Der \textbf{Business Event Handling-Service} in SAP S/4HANA Cloud, bietet einen standardisierten Weg zur Generierung von Ereignissen auf Basis von Geschäftsobjeken.
\end{itemize}

Geschäftsobjekte sind die Basis für die Ereignisverarbeitung in SAP S/4HANA Cloud. Geschäftsobjekte sind im Kontext von SAP ERP und in der SAP S/4HANA Cloud gut definiert. Ein Geschäftsobjekt ist eine Geschäftsfunktionalität, die eigenständig instanziiert werden kann. Während Geschäftsobjekte fast ständig verwendet werden, werden ihre Typen nur selten explizit berücksichtigt. In Geschäftsanwendungen werden die Typen der Geschäftsobjekte aus dem Kontext, in dem sie angezeigt werden, abgeleitet und sind bekannt, aber im Allgemeinen nicht explizit. Ein gemeinsames Typensystem zwischen dem SAP Enterprise Messaging-Service und SAP S/4HANA Cloud ist daher die Voraussetzung für die generische Interaktion zwischen den Systemen
\cite{Herzig.2018}. 

Dieser kurze Abriss zu den Grundlagen der Ereignisverarbeitung in SAP S/4HANA Cloud  soll an dieser Stelle genügen, während für eine detaillierte Betrachtung der existierenden Dienste auf die genannte Literatur in diesem Umfeld verwiesen sei.\cite{Herzig.2018}

Zur Vorbereitung der Integration des Ereignisverarbeitungsdienstes muss folglich der \textit{SAP Enterprise Messaging-Service} auf der SAP Cloud Platform sowie \textit{Business Event Handling-Service} in SAP S/4HANA Cloud konfiguriert werden, diese Verfahren werden in Abschnitt \ref{sec:integration} noch näher beschrieben. 
\section{Modellbasierte Darstellung des Geschäftsprozesses}\label{sec:bpmnabbildung}
\todo{Grundlangen BPMN 2.0 + neues Modell}
% \chapter{Konzeption des Ereignisverarbeitungsmodells}

\section{Spezifizierung des Ereignismodells}

\todo{Welche Ereignisse gibt es?}

\todo{Welche Daten braucht man?}




\section{Entwicklung des Ereignisverarbeitungsdienstes}

\todo{Wie wird kommuniziert?}

\missingfigure{Event Messaging Architecture}
\chapter{Softwaretechnische Implementierung}\label{ch:Implementierung}

\section{Transformation des Ereignisverarbeitungsmodells in ausführbaren Code}\label{sec:Transformation}

\subsection{Implementierte Komponenten}
\todo{Java Coding hier platzieren}

\subsection{Softwarearchitektur}


\subsection{Benutzung}

\todo{JavaScript Coding hier platzieren}

\section{Implementierung der Benutzerschnittstelle}

\subsection{Softwarearchitektur}

\subsection{Benutzung}
\section{Technische Anbindung von Diensten und Ereignisquellen}

\subsection{Integration mit SAP S/4HANA Cloud}

\subsection{Anbindung von SAP Enterprise Messaging}

\subsection{Erweiterung durch Google Cloud Vision}
% \chapter{Anwendung und Bewertung}

\section{Evaluation auf Geschäftsprozessebene}
\section{Konzeption auf  Ereignisverarbeitungsebene}\label{sec:ereignismodell}
Diesesr Abschnitt behandelt die technische Sichtweise auf die Ereignisverarbeitungsebene, die mittels Echtzeitverarbeitung von zur Laufzeit auftretenden Ereignissen die Dynamik für den Ablauf der Geschäftsprozessebene liefert. Für die digitale Verarbeitung der Ereignisse ist eine explizite Spezifikation des Ereignismodells nötig. Im Sinne einer softwaretechnischen Implementierung wird darüber hinaus der Ereignisverarbeitungsdienst konzipert.

Als Spezifikation einer allgemeinen Beschreibung von Ereignisobjekten dient die \textit{CloudEvents}-Spezifikation in der Version 0.3-wip. 
\textit{CloudEvents} bietet eine standardisierte Definition der Struktur und Metadatenbeschreibung von Ereignissen. Diese Definition definiert, wie die in der \textit{CloudEvents}-Spezifikation definierten Elemente im \ac{JSON}-Format dargestellt werden sollen. \cite{CloudEvents.2019}

Im weiteren Verlauf dieser Bachelorarbeit wird nun auch spezifischer auf eine Umsetzung mit SAP S/4HANA Cloud eingegangen. SAP S/4HANA Cloud bietet für die Behandlung von verschiedenen vordefinierten Geschäftsereignissen eigene Dienste zur Integration an. Der auf der SAP Cloud Platform basierte \textit{SAP Enterprise Messaging-Service} bietet Funktionalitäten für die Integration in SAP S/4HANA Cloud, um Ereignisse zu verbreiten und reaktive Geschäftsprozesse über Unternehmenslandschaften hinweg zu ermöglichen.
\cite{Schneider.2018}

\subsection{Spezifizierung des Ereignismodells}
Ein Ereignisobjekt repräsentiert ein in der Realität eingetretenes Ereignis. Es muss somit alle relevanten Informationen beinhalten, die für eine anschließende Verarbeitung dieses Ereignisses notwendig sind.

Unterschieden wird dabei in verschiedene Ereignistypen. Ein Ereignistyp setzt sich laut der \textit{CloudEvents}-Spezifikation aus dem betroffenen Objekt und der Art der Veränderung zusammen. Einfach ausgedrückt werden durch ein Ereignisobjekt die folgenden vier Fragen beantwortet: \textit{Wann hat es sich ereignet? Was hat sich ereignet? Wer war betroffen?}
Im Kontext dieser Bachelorarbeit wird die Antwort auf die erste Frage als der Zeitpunkt des Ereignissen verstanden. Die zweite Frage zielt auf die Veränderung eines Objektes ab, sie enthält somit die Antwort welcher Objekttyp verändert wurde und wie diese Veränderung aussieht. Die letzte Frage identifiziert schließlich ein eindeutiges Objekt des veränderten Objekttyps. \cite{CloudEvents.2019}

Beim Erstellen eines Ereignisobjektes im Rahmen dieser Bachelorarbeit müssen demnach mindestens die folgenden Daten an den Konstruktor der Ereignisgenerierung übergeben werden:
\begin{itemize}
    \item \textbf{Zeitstempel:} Er beschreibt wann das Ereignis ausgelöst wurde. 
    \item \textbf{Objekttyp:} Er spezifiziert die Klasse des veränderten Objekts. 
    \item \textbf{Ereignistyp:} Er definiert die Art und Weise wie ein Objekt verändert wurde. 
    \item \textbf{Objektschlüssel:} Er identifiziert das betroffene Objekt. 
\end{itemize}

Die SAP-spezifischen Geschäftsobjekte , auch \textit{SAP-Objekttyp} genannt, sind die Basis für Ereignisobjekte innerhalb von SAP S/4HANA Cloud. 
Ein SAP-Objekttyp kann Geschäftsereignisse mit einer eindeutigen Semantik für bestimmte Situationen definieren.
Ein Ereignis kann ausgelöst werden, wenn eine bestimmte Aktion für das Geschäftsobjekt ausgeführt wird oder wenn das Geschäftsobjekt einen bestimmten Status annimmt.
Sobald Ereignisse von Geschäftsobjekten ausgelöst werden, werden diese Ereignisse von SAP S/4HANA Cloud an ein Framework zur Ereignisverarbeitungs weitergegeben. 
\cite{Herzig.2018}
Dieses Framework bildet aus der internen technischen Darstellung des Ereignisses eine Menschen-lesbare Darstellung im \ac{JSON}-Format. Diese Darstellung, wie im Quelltext \ref{code:jsonevent}, wird der \textit{CloudEvents}-Spezifikation gerecht und enthält alle notwendigen Daten.
\cite{Herzig.2018}

\begin{algorithm}[H]
\centering 
\inputminted[linenos]{js}{code/sapevent.json}
\caption{Exemplarisches SAP-spezifisches Ereignisobjekt im JSON-Format \cite{Herzig.2018}}
\label{code:jsonevent}
\end{algorithm}

Um diese Integration zu erreichen, werden zwei Komponenten benötigt: Die allgemeine Integration in die ereignisgesteuerte Architektur von SAP S/4HANA Cloud und eine spezifische Integration mit dem SAP Enterprise Messaging-Service, der auf der SAP Cloud Platform angeboten wird. Beide Komponenten kommunizieren über Standard-Nachrichten- und Verbindungsprotokolle. Anwendungssoftware, die auf der SAP Cloud Platform basiert, kann dabei als Ereigniskonsument fungieren.

\subsection{Architektur der Ereignisverarbeitungsdienstes}
Das Schlüsselelement reaktiver Systeme ist die asynchrone nachrichtengesteuerte Kommunikation zwischen den beteiligten Anwendungssystemen. SAP S/4HANA Cloud setzt hier auf das MQTT-Protokoll. MQTT steht für Message Queuing Telemetry Transport und ist ein standardisiertes Protokoll für die Maschine-to-Maschine-Kommunikation.
Es lassen sich jedoch auch Ereignisverarbeitungsdienste nach den Konzepten des \textit{Publish-Subscribe-Paradigmas} (siehe Abschnitt \ref{sec:Ereignisverarbeitung}) damit umsetzen.
\cite{Herzig.2018}

In SAP S/4HANA Cloud werden zur Integration zwei Dienste angeboten:
\begin{itemize}
    \item Der \textbf{SAP Enterprise Messaging-Service} bietet Ereignisverarbeitungs-Funktionen mit Hilfe der SAP Cloud Platform. Der Diest umfasst die sofortige Integration in SAP S/4HANA Cloud, um Ereignisse zu verbreiten und reaktive Geschäftsprozesse über Anwendungssysteme hinweg zu ermöglichen.
    \item Der \textbf{Business Event Handling-Service} in SAP S/4HANA Cloud, bietet einen standardisierten Weg zur Generierung von Ereignissen auf Basis von Geschäftsobjeken.
\end{itemize}

Geschäftsobjekte sind die Basis für die Ereignisverarbeitung in SAP S/4HANA Cloud. Geschäftsobjekte sind im Kontext von SAP ERP und in der SAP S/4HANA Cloud gut definiert. Ein Geschäftsobjekt ist eine Geschäftsfunktionalität, die eigenständig instanziiert werden kann. Während Geschäftsobjekte fast ständig verwendet werden, werden ihre Typen nur selten explizit berücksichtigt. In Geschäftsanwendungen werden die Typen der Geschäftsobjekte aus dem Kontext, in dem sie angezeigt werden, abgeleitet und sind bekannt, aber im Allgemeinen nicht explizit. Ein gemeinsames Typensystem zwischen dem SAP Enterprise Messaging-Service und SAP S/4HANA Cloud ist daher die Voraussetzung für die generische Interaktion zwischen den Systemen
\cite{Herzig.2018}. 

Dieser kurze Abriss zu den Grundlagen der Ereignisverarbeitung in SAP S/4HANA Cloud  soll an dieser Stelle genügen, während für eine detaillierte Betrachtung der existierenden Dienste auf die genannte Literatur in diesem Umfeld verwiesen sei.\cite{Herzig.2018}

Zur Vorbereitung der Integration des Ereignisverarbeitungsdienstes muss folglich der \textit{SAP Enterprise Messaging-Service} auf der SAP Cloud Platform sowie \textit{Business Event Handling-Service} in SAP S/4HANA Cloud konfiguriert werden, diese Verfahren werden in Abschnitt \ref{sec:integration} noch näher beschrieben. 
% \section{Fachliche Anforderungen an die Bachelorarbeit}

\todo{Der Punkt \enquote{Fachliche Anforderungen an die Bachelorarbeit} Sollte in einen anderen Einleitungspunkt integriert werden}

\cite{Staud2006Geschaftsprozessanalyse:Standardsoftware}
\chapter{Schlussbetrachtung}\label{ch:Ende}
Zum Abschluss dieser Arbeit soll der dynamische Geschäftsprozess zur Fertigungsdurchführung auf Basis von Ereignisverarbeitung im Gesamtzusammenhang begutachtet werden.
Zudem erfolgt ein Ausblick auf zukünftige Entwicklungen und Erweiterungsmöglichkeiten.
\section{Bewertung der Anforderungserfüllung}
\section{Kritische Würdigung}
\section{Ausblick}\label{sec:ausblick}
Während der Fokus dieser Bachelorbeit in der schnellen, zielgerichteten Versorgung mit aktuellen Betriebsdaten aus der Fertigungsduchführung und dem Zugriff auf Funktionen des fertigungsnahen Moduls \textit{Qualitätsmanagement} liegt, ergibt sich weiterer Forschungsbedarf in der Anwendung des Ansatzes der dynamischen, ereignisorientierten Geschäftsprozesse in weiteren Phasen des Produktionslebenszyklus. 

Neben der Erweiterung des Ansatzes auf die Anwendung in den Fertigungsdurchführung der diskreten Fertigung ist weiteres Potenzial zur Weiterentwicklung des Ansatzes im Bereich der Industrie 4.0, dementsprechend in der Verwendung in cyberphysischen Systemen, in Produkten und Ressourcen zu sehen. Die identifizierten Anforderungen und Erkenntnisse dieser Bachelorarbeit können hier als Ausgangspunkt dienen, um die Informationen und Zusammenhänge zwischen der physischen und der digitalen Welt zu verstehen und zu adaptieren.




%	Literaturverzeichnis
\ihead{} % Neue Header-Definition
\printbibliography[title=Literaturverzeichnis]
\counterwithin{figure}{chapter}
\counterwithin{table}{chapter}
\counterwithin{figure}{chapter}
\counterwithin{algorithm}{chapter}


\cleardoublepage
% Der Anhang beginnt hier - jedes Kapitel wird alphabetisch aufgezählt. (Anhang A, B usw.)
\appendix
\ihead{\appendixname~\thechapter} % Neue Header-Definition

% appendix.tex einziehen

\chapter{Interviewprotokolle}\label{ah:protokolle}

\spaceparagraph{Vorwort zu den Experteninterviews}
Die folgenden Ausführungen fassen die Resultate der im Rahmen dieser Bachelorarbeit durchgeführten Experteninterviews zusammen.
Die Auswertung erfolgt summarisch und wird nicht wortwörtlich transkribiert.
Dieses Vorgehen ist mit dem wissenschaftlichen Betreuer dieser Bachelorarbeit abgestimmt.
Die Interviewpartnerinnen und -partner haben der Veröffentlichung ihres Interviews in der Audio- oder Textversion zugestimmt.
Aus datenschutzrechtlichen Gründen wurden in den folgenden Interviewprotokollen alle Namen, Unternehmen und Institutionen durch Pseudonyme ersetzt und dementsprechend gekennzeichnet.

\begin{itemize}
  \item
        \textit{Thema der Experteninterviews:} 
        Die Experteninterviews beziehen sich auf die Erhebung von in der Praxis eingesetzten Abläufe einer Fertigungsdurchführung in der diskreten Industrie im Allgemeinen sowie im SAP-Umfeld.
        Welche Faktoren in Form von Verschwendung negativ Einfluss auf den Geschäftsprozess nehmen und wie sich diese potenziell eliminieren ließen, wird im Rahmen der Experteninterviews ebenfalls eruiert.
        Im Fokus steht die Ermittlung der etablierten Aktivitäten und Hilfsmittel einer Fertigungsdurchführung, deren Merkmale und Defizite sowie die Identifikation von Alternativen.
  \item 
        \textit{Ziel der Experteninterviews:}
        Mit den Experteninterviews sollen tatsächliche Gegebenheiten einer Fertigungsdurchführung in der diskreten Industrie begreiflich gemacht werden. 
        Anhand der erarbeiteten Merkmale und Defizite werden anschließend formulierbare Verbesserungspotenziale in Erfahrung gebracht.
        Diese Potenziale sollen als Summarium an Anforderungen einer Fertigungsdurchführung im SAP-Umfeld zur Verfügung stehen und als Fundament der Evaluation der zu betrachtenden Maßnahmen dienen.
\end{itemize}

\newpage
% **************************************************************************************
% * Anwenderperspektive                                                               
% **************************************************************************************
\tocless\section{Interview mit einem Branchenexperten}\label{ah:interviewTUD}

\spaceparagraph{Experte für das Experteninterview}
\textit{Prof. Dr.-Ing. Max Mustermann \footnote{Der Name \textit{Max Mustermann} repräsentiert den Namen des Branchenexperten.}} wird für das Experteninterview befragt.
Er ist Institutsleiter des Instituts für Produktionsmanagement, Technologie und Werkzeugmaschinen der \textit{Technische Universität Musterstadt \footnote{Die \textit{Technische Universität Musterstadt} repräsentiert das Bildungsinstitut des Branchenexperten.}}, in dessen Zuständigkeit die folgenden Forschungsbereiche fallen: Digitalisierung und Vernetzung von Fertigungsprozessen, Echtzeitdatenerfassung und Automatisierungstechnik für die diskrete Fertigung, Werkzeugmaschinen und Industrierobotik \footnote{Die Aufzählung der Forschungsbereiche dient lediglich als Überblick und ist nicht vollständig.}

Die Tätigkeit des Institutsleiters übt \textit{Prof. Dr.-Ing. Max Mustermann} seit dem Jahr 2019 an der \textit{Technischen Universität Musterstadt} aus. Seit dem Jahr 2002 hatte er diverse Leitungsfunktionen mit den Themenschwerpunkten Werkzeugtechnologie, Produktionsplanung, Prototypen- und Serienfertigung, Betriebsmittelkonstruktion und Automatisierungstechnik sowie Engineering und Innovationsmanagement bei der \textit{Musterberger Druckmaschinen AG \footnote{Die \textit{Musterberger Druckmaschinen AG} repräsentiert einen Arbeitgeber des Branchenexperten.}}, sowie der \textit{ERP SE \footnote{Die Firma \textit{ERP SE} repräsentiert einen Arbeitgeber des Branchenexperten.}} inne. Zum Zeitpunkt der Erstellung dieser Bachelorarbeit ist er damit seit 17 Jahren im Fachbereich Produktionsmanagement tätig. Aufgrund dessen besitzt er einen großen Erfahrungsschatz im Umfeld der industrienahen Forschung und ein ausgeprägtes Verständnis für die tatsächlichen Abläufe einer Fertigungsdurchführung in der diskreten Industrie, welches anhand dieses Experteninterviews greifbar gemacht werden soll.

\spaceparagraph{Transkription des Experteninterviews}
\textit{Interviewer}: Guten Tag Herr Mustermann. Danke, dass Sie sich für dieses Experteninterview Zeit genommen haben.
\newline
\textit{Max Mustermann}: Guten Tag Herr Sauer. Sehr gerne.

\textit{Interviewer}: : Das Experteninterview wird sich so strukturieren, dass wir erst formale Aspekte klären. Danach werde ich Ihnen Fragen zu Merkmalen und Defiziten von Aktivitäten in der Fertigungsdurchführung stellen. Für das Experteninterview sind 60 Minuten angesetzt. Falls Sie Pausen oder kurze Unterbrechungen benötigen, geben Sie mir bitte kurz Bescheid. Sind das Vorgehen und der Rahmen aus Ihrer Sicht in Ordnung?
\newline
\textit{Max Mustermann}: Danke Ihrer Nachfrage, das passt.

\textit{Interviewer}: Das Experteninterview führe ich im Rahmen meiner Bachelorarbeit durch. Darf ich alle Aussagen und Erkenntnisse aus dem Experteninterview auswerten und in meine Bachelorarbeit einbeziehen?
\newline
\textit{Max Mustermann}: Ja.

% Herr Bourlauf, auch in der Bankenbranche ist die Industrialisierung von
% Geschäftsprozessen ein viel diskutiertes Thema. Wie schätzen Sie den aktuellen
% Stand ein?

% es  wird viel geschrieben über die Industrialisierung bei Banken. Von
% Standardisierung, Taylorisierung, Automatisierung, Modularisierung, kontinuierlicher
% Verbesserung, Konzentration auf Kernkompetenzen und Reduzierung der Fertigungstiefe ist die Rede. Die Automobilbranche, die unterschiedliche Fabrikate auf einer Fertigungsstraße produziert, wird oft als Vorbild für die Finanzdienstleistungsbranche
% genommen. Green, Yellow, Black Belts und Ähnliches etablieren sich im Rahmen von
% Six Sigma als neue Berufsgruppen. Eine SOA hat heutzutage angeblich jedes Softwarehaus und jede IT-Abteilung, die etwas auf sich hält. Business Process Management (BPM) betreibt inzwischen jede moderne Organisation, sei es mit Visio, ARIS
% oder anderen »Maltools«. Kreditfabriken existieren und eigentlich sind die Banken
% doch schon sehr weit – oder?

% Welchen Weg sind Sie gegangen?

%  Die Degussa Bank ist eine im Geschäftsmodell streng fokussierte Privatbank, die mit über 600 Mitarbeitern in über 210 Zweigstellen das Privatkundengeschäft in Deutschland betreibt. Wir haben in den letzten beiden Jahren schrittweise
% begonnen, zwei Bereiche zu industrialisieren und zu reorganisieren: den Immobilienkreditbereich und erhebliche Teile des Kundenservices mit über 70 Prozessen aus
% Konto- und Kundenservice. Die Geschäftsprozesse beider Bereiche haben unterschiedliche Charakteristika und Anforderungen. Während es im Immobilienkreditbereich um relativ wenige, aber dafür komplexe Prozessarten geht, gibt es im Kundenservice viele unterschiedliche Prozessarten mit hoher Agilität.
% Bei der Umgestaltung haben wir uns von zwei Grundgedanken leiten lassen:
% Erstens werden alle Prozesse »End-to-End« vom Kunden aus betrachtet. Wir sprechen hier von kundenfokussierter Fertigung, zum Kunden hin individuell, in der Fertigung weitestgehend standardisiert. Zweitens denken wir betriebswirtschaftlich – analog zur Automobilindustrie – in Fertigungsstraßen und Standards, damit die Lösungen
% als »Blueprint« einsetzbar sind.
% Im Rahmen unserer serviceorientierten Architektur sind heute alle Prozesse in
% einer BPM-Plattform abgebildet und werden dort orchestriert und überwacht. Die
% BPM-Plattform bildet das Bindeglied zwischen unserer SOA und der Fachabteilung.
% Es gibt nun in allen Bereichen Produktionsleitstände, mit deren Hilfe wir u.a. eine
% deutlich flexiblere Arbeitsorganisation erreicht haben. Je nach tatsächlicher oder
% erwarteter Auslastung der Teams werden Finanzprodukte von einer Fertigungsstraße
% auf eine andere Fertigungsstraße verschoben oder Kollegen aus einem Team virtuell
% einem anderen Team zugeordnet, sofern sie über die entsprechenden Fähigkeiten
% verfügen. Zusätzlich besteht die Möglichkeit, Zweigstellen anderen Betreuungsteams
% zuzuordnen.

% : Mit der Industrialisierung von Geschäftsprozessen verbunden sind sicherlich erhebliche Auswirkungen auf die Organisation, denen aber auch große Nutzenpotenziale gegenüberstehen. Können Sie uns diese kurz beschreiben?


% In der Tat hat die Industrialisierung erhebliche Auswirkungen auf unsere
% Gesamtorganisation und führt zu deutlichen Verbesserungen für unsere Servicequalität und Kostenposition. Die wichtigsten Aspekte aus meiner Sicht sind folgende:
% ■ Geschäftlich relevante Prozesse laufen nun automatisiert in der BPM-Plattform ab
% und sind transparent für jeden verfügbar. Das ist die Grundlage für alle weiteren
% Möglichkeiten, die mit der Industrialisierung einhergehen. Voraussetzung hierfür
% war neben der Einführung der BPM-Plattform in unserem Fall auch die Digitalisierung der gesamten Kundenkorrespondenz und damit die Einführung der elektronischen Kredit- und Kundenakte.
% ■ Da alle Informationen im System transparent vorhanden sind, können wir – egal
% über welchen Vertriebskanal der Kunde uns anspricht – direkt Auskunft über den
% Vorgang geben.
% ■ Alle betroffenen Prozesse wurden standardisiert und können nun leicht verändert
% werden. Das bedeutet, dass wir jederzeit Prozessabschnitte taylorisieren oder
% zusammenfassen können, abhängig vom Mengenaufkommen und den Fähigkeiten
% der verfügbaren Mitarbeiter. Uns steht ein Pool an praxisgetesteten Prozessabschnitten zur Verfügung, die wir als Vorlage nutzen und rasch an sich schnell verändernde Marktbedingungen anpassen können. Außerdem können wir auf diese
% Weise unser Personal flexibler einsetzen, die erforderlichen Skill-Levels in einzelnen Prozessschritten senken und auf Engpässe zeitnah reagieren. Damit wird ein
% einheitlicher Qualitätsstandard gesichert, volle Transparenz erzeugt und die Einhaltung definierter Service Levels über alle Arbeitsprozesse hinweg überprüfbar.
% ■ Die Fertigungstiefe kann die Bank jederzeit selbst bestimmen. Damit können wir
% z.B. einfachere Tätigkeiten im Rahmen der Kampagnenbearbeitung an externe
% Dienstleister auslagern und unsere hoch qualifizierten Mitarbeiter von monotonen
% Tätigkeiten entlasten.
% ■ Die Strukturen der betroffenen Abteilungen wurden im Alignment mit den Prozessen aufgestellt. Durch das Aufsetzen eines kontinuierlichen Verbesserungsprozesses unter Verwendung der statistischen Kennzahlen aus dem BPM-System kann
% die Fachabteilung die Prozesse und ihre eigene Organisation optimieren. Die
% Reorganisation wurde aus der Fachabteilung initiiert, bedurfte keiner externen
% Beratungsgesellschaft, und die Organisationsstruktur passte sich Schritt für Schritt
% dem optimierten Prozessmodell an.
% ■ Das Know-how der Mitarbeiter ist nun in der BPM-Plattform abgebildet und allgemein zugänglich. Dadurch ist die ursprüngliche Profession der Mitarbeiter teilweise
% verloren gegangen, neue Rollen und Fähigkeiten mussten erlernt und ausgebildet
% werden.
% ■ Last but not least: Wir haben durch diese Maßnahmen Effizienzsteigerungen von
% 30 Prozent und mehr erreicht.

% \textit{Interviewer}: Danke dafür. Wie lange beschäftigen Sie sich schon mit den Geschäftsprozessen der Produktionsplanung und -steuerung? Führen Sie ausschließlich Analysen zu kritischen Systemzuständen durch oder setzen Sie Vorschläge Ihrerseits auch direkt an Kundensystemen um?
% \newline
% \textit{Max Mustermann}: Meine Tätigkeit übe ich aus, seitdem die Datenbank SAP HANA 2010 auf den Markt gekommen ist. Das heißt, mittlerweile beschäftige ich mich schon acht Jahre mit der Thematik in Bezug auf die Systemlandschaftsdaten. Dabei arbeite ich im Rahmen des SAP EWA Next Generation des SAP S/4HANA Service and Support. Ich besitze den Status als Datenbankadministrator. Neben meiner Analysetätigkeit behebe ich kritische Sytemzustände in Zusammenarbeit mit dem Kunden.

% \textit{Interviewer}: Guten Tag Herr Mustermann. Danke, dass Sie sich für dieses Experteninterview Zeit genommen haben.
% \newline
% \textit{Max Mustermann}: Guten Tag Herr Sauer. Sehr gerne.

\newpage
% **************************************************************************************
% * Prozessexperten                                                               
% **************************************************************************************
\tocless\section{Interview mit einem Prozessexperten}\label{ah:interviewCON}

\spaceparagraph{Experte für das Experteninterview}
\textit{Herr Otto Normalverbraucher \footnote{Der Name \textit{Otto Normalverbraucher} repräsentiert den Namen des Prozessexperten.}} wird für das Experteninterview befragt.
Er ist einer der Experten des Beratungsbereiches \enquote{Manufacturing Industries} der \textit{ERP SE \footnote{Die Firma \textit{ERP SE} repräsentiert den Arbeitgeber des Prozessexperten.}}, welcher Betriebe, aus der Automobilindustrie, im Maschinen- und Anlagenbau sowie der Hightech Industrie, bei der Gestaltung branchenspezifischer Prozessabläufe und der Umsetzung mit SAP-Software unterstützt.
In seiner Rolle als \enquote{Senior Business Process Consultant} bewertet er Kundenanforderungen, stimmt sie auf die Geschäftsprozesse ab, konzeptioniert die Abbildung mittels SAP Lösungen in einer Gesamtlösungsarchitektur und sorgt für die Implementierung aus Sicht der Lösungsarchitektur.
Basierend auf den ihm vorliegenden Systemlandschaftsdaten erarbeitet er detaillierte Konzepte durchgängiger, branchenspezifischer Geschäftsprozesse und identifiziert dabei Faktoren, die einen negativen Einfluss auf den Geschäftsprozess haben.
Im Dialog mit den Kunden hilft er diesen, die kritischen Faktoren nachhaltig zu beseitigen. 
 
Diese Tätigkeit übt \textit{Herr Otto Normalverbraucher} seit dem Jahr 1999 (seit dem Jahr 2004 bei der \textit{ERP SE}) aus und besitzt zum Zeitpunkt der Erstellung dieser Bachelorarbeit damit eine Berufserfahrung von 20 Jahren.
Aufgrund dieser besitzt er einen großen Erfahrungsschatz hinsichtlich der Merkmale von etablierten Arbeitsprozessen und Hilfsmitteln zur Fertigungsdurchführung im SAP-Umfeld, welche anhand dieses Experteninterviews greifbar gemacht werden soll.

\spaceparagraph{Transkription des Experteninterviews}
\newpage
% **************************************************************************************
% * Technologieexperten                                                               
% **************************************************************************************
\tocless\section{Interview mit einem Technologieexperten}\label{ah:interviewDev}

\spaceparagraph{Experte für das Experteninterview}
\textit{Herr John Doe \footnote{Der Name \textit{John Doe} repräsentiert den Namen des Technologieexperten.}} wird für das Experteninterview befragt.
 Er ist einer der Experten des Entwicklungsbereiches \enquote{S/4HANA Cloud Produce - Manufacturing} der \textit{ERP SE \footnote{Die Firma \textit{ERP SE} repräsentiert den Arbeitgeber des Technologieexperten.}}, in dessen Verantwortungsbereich die Anwendungen und Technologien zur Produktionsplanung und -steuerung der \ac{ERP}-Software SAP S/4HANA entworfen und entwickelt werden.
 In seiner Rolle als \enquote{Development Expert} eruiert er Anforderungen an die \ac{ERP}-Software, transformiert bestehende Geschäftsprozesse in Anwendungen, konzeptioniert die technologischen Hilfsmittel und sorgt für die Implementierung aus Sicht der Systemarchitektur. Basierend auf den ihm vorliegenden Geschäftsprozessen erarbeitet er detaillierte Konzepte standardisierter, prozessspezifischer Anwendungen und evaluiert dabei mögliche Technologien, die zur Unterstützung der Geschäftsprozesse fungieren. Im Dialog mit den Kunden, dem Beratungsbereich und dem Produktmanagement werden geeignete Technologien identifiziert, um die Geschäftsprozesse nachhaltig zu verbessern. 
 
Diese Tätigkeit übt \textit{Herr John Doe} seit dem Jahr 2000 bei der \textit{ERP SE} aus und besitzt zum Zeitpunkt der Erstellung dieser Bachelorarbeit damit eine Berufserfahrung von 19 Jahren. Aufgrund dieser besitzt er einen großen Erfahrungsschatz hinsichtlich der Anforderungen an geeignete Technologien der angebotenen SAP-Software zur Fertigungsdurchführung, welche anhand dieses Experteninterviews greifbar gemacht werden soll.

\paragraph{Kernaussagen des Experteninterviews}

\begin{definitionForm}[KA-T-1]
Die Fertigungsdurchführung in SAP S/4HANA Cloud ist durch \textbf{mangelhafte Datenqualität} gekennzeichnet. Durch den Zeitdruck in der Produktion werden Rückmeldungen so schnell wie möglich erfasst. Warnungen werden hierbei ignoriert. Des weiteren nutzen nicht alle Kunden die Customizing-Funktionalität Toleranzregeln, für die Rückmeldung festzulegen.
\end{definitionForm}

\begin{definitionForm}[KA-T-2]
Die Erfassung von \textbf{Fertigungsrückmeldungen in digitaler Form} wird von Seiten des Entwicklungsbereiches \enquote{S/4HANA Cloud Produce - Manufacturing} empfohlen. Die Strategie einer papierlosen Fertigung findet in der Praxis noch keine weite Verbreitung.
\end{definitionForm}

\begin{definitionForm}[KA-T-3]
Die Rückmeldung von Fertigungsaufträgen erfolgt in der Regel auf der Vorgangsebene. Die minimal notwendigen Informationen sind: \textbf{Auftragsnummer, Vorgangsnummer, Gut-, Ausschuss- und Nacharbeitsmenge, Rüstzeit, Bearbeitungszeit und Abrüstzeit}. Die Angabe von Personalnummern ist eher unüblich und meist nicht notwendig.
\end{definitionForm}

\begin{definitionForm}[KA-T-4]
Der \textbf{Einsatz von mobilen Geräten} hat ein großes Potenzial die Fertigungsdurchführung effizienter zu gestalten. Eine Durchführung der Betriebsdatenerfassung über einen Barcode-Scanner oder über Text- bzw. Formularerkunng ist ein Anwendnungsfall der ebenfalls ein sehr hohes Potenzial für die diskrete Fertigung hat.
\end{definitionForm}

\begin{definitionForm}[KA-T-5]
Die Fertigungsdurchführung in SAP S/4HANA Cloud wird oft nur zur \textbf{Materialbedarfsplanung und Abbrechnung} genutzt. Dabei kommen zur eigentlichen Fertigungsdurchführung sogenannte MES-Systeme oder andere IT-Systeme zum Einsatz.
\end{definitionForm}


% \chapter*{Templates}

% \begin{figure}[htb]
% \centering
% \begin{tikzpicture}
%   \begin{axis}[
%     width=\textwidth,
%     height=9.5cm,
%     ybar=0pt, % Raum zwischen den Balken
%     bar width=40pt,
%     enlarge x limits=0.5,
%     legend style={at={(1,1.15)},anchor=north east,draw=none},% Legende über Abb.
%     legend cell align=left,% Linksbündige Ausrichtung der Legende
%     % xlabel={Gesicht},
%     xtick={data},
%     symbolic x coords={hoch vertrauenswürdig,niedrig vertrauenswürdig},
%     ymin=17,
%     ymax=25,
%     ylabel={Mittlerer Einsatz},
%     %ytick={17,18,...,25}
%   ]
%     \addplot[
%         black,fill=lightgray
%       ]coordinates {
%       (hoch vertrauenswürdig,23.453) 
%       (niedrig vertrauenswürdig,18.797)
%     };
%     % \addlegendentry{~betrügerisches Verhalten};
%     % \addplot[
%     %     black,fill=white,
%     %     postaction={pattern=north east lines,pattern color=gray}
%     %     ] coordinates {
%     %   (hoch vertrauenswürdig,22.891) +- (0,0.410)
%     %   (niedrig vertrauenswürdig,18.844) +- (0,0.407)
%     % };
%     % \addlegendentry{~kooperatives Verhalten};
%   \end{axis}
% \end{tikzpicture}
% \caption{Investitionsverhalten. Die Fehlerbalken stellen die Standardfehler dar.}
% \label{fig:Investitionsverhalten}
% \end{figure}

% \begin{figure}[H]
% 	\centering 
% 	\includegraphics[width=\textwidth]{img/Production_IST.png}	\caption[TEST]{\label{fig:logo}test
% 	}
% \end{figure}


% % \begin{figure}[H]
% % 	\centering 
% % 	\includegraphics[width=\textwidth]{img/Arch_IST.png}	\caption[TEST]{\label{fig:logo}test
% % 	}
% % \end{figure}

% % \begin{figure}[H]
% % 	\centering 
% % 	\includegraphics[angle=270,width=\textwidth]{img/Arch_SOLL.png}	\caption[TEST]{\label{fig:logo}test
% % 	}
% % \end{figure}

% % \begin{figure}[H]
% % 	\centering 
% % 	\includegraphics[angle=270,width=\textwidth]{img/Order_Operating_IST.png}	\caption[TEST]{\label{fig:logo}test
% % 	}
% % \end{figure}

% % \begin{figure}[H]
% % 	\centering 
% % 	\includegraphics[angle=270,width=\textwidth]{img/Order_Operating_SOLL.png}	\caption[TEST]{\label{fig:logo}test
% % 	}
% % \end{figure}

% % \begin{algorithm}[H]
% % \centering 
% % \inputminted[linenos]{java}{code/Application.java}
% % \caption{Application.java}
% % \end{algorithm}

% % \begin{algorithm}[H]
% % \centering 
% % \inputminted[linenos]{java}{code/MessageController.java}
% % \caption{MessageController.java}
% % \end{algorithm}

% % \begin{algorithm}[H]
% % \centering 
% % \inputminted[linenos]{java}{code/ProductionOrderController.java}
% % \caption{ProductionOrderController.java}
% % \end{algorithm}

% % \begin{algorithm}[H]
% % \centering 
% % \inputminted[linenos]{java}{code/ProductionOrderReleasedNotificationListener.java}
% % \caption{ProductionOrderReleasedNotificationListener.java}
% % \end{algorithm}

% % \begin{algorithm}[H]
% % \centering 
% % \inputminted[linenos]{java}{code/EventMessagingService.java}
% % \caption{EventMessagingService.java}
% % \end{algorithm}

% \begin{algorithm}[H]
% \centering 
% % \lstinputlisting[style=JavaStyle, caption=pr.java]{code/test.java}
% \inputminted[linenos]{js}{code/test.js}
% \caption{test.js}
% \end{algorithm}

% % \begin{algorithm}[H]
% % \centering 
% % % \lstinputlisting[language=ABAP, caption=test.abap]{code/test.abap}
% % \inputminted[linenos]{ABAP}{code/RaiseEvent.abap}
% % \caption{Exemplarisches Auslösen eines Ereignisses in SAP S/4HANA}
% % \end{algorithm}

% % \begin{algorithm}[H]
% % \centering 
% % % \lstinputlisting[language=ABAP, caption=test.abap]{code/test.abap}
% % \inputminted[linenos]{js}{code/test.json}
% % \caption{Exemplarisches Auslösen eines Ereignisses in SAP S/4HANA}
% % \end{algorithm}

% \begin{definitionForm}[Definition]
% Diese Hervorhebungen können für deine Arbeit an machen stellen sehr nützlich sein. Besonders bei Definitionen macht es einen guten Eindruck, wenn diese in solch einer Form dargestellt ist. 
% \end{definitionForm}

% DHBW Richtlinie: Laut den aktuellen Angaben der DHBW sind diese Boxen nicht notwendig. Helfen können sie jedoch, um einen Faktor speziell hervorzuheben. Bitte beachte, dass deine Projektarbeit oder auch Bachelorarbeit kein Bilderbuch ist! Alles was eingebunden wird sollte schlicht und dezent dargestellt sein.

% \begin{attentionForm}[SOA und Webservices sind nicht grundsätzlich synonym]
% An dieser Stelle sei jedoch der Hinweis erlaubt, dass das Konzept der Serviceorientierung allgemeiner ist und schon früher existierte als Webservices. Webservices sollten daher nur als eine, wenn auch zum Verfassungszeitpunkt dieses Buches als wahrscheinlich am besten geeignete Möglichkeit zur Realisierung serviceorientierter Architekturen betrachtet werden.
% \end{attentionForm}



% \begin{table}[H]
% 	\centering
% 	\begin{tabularx}{\textwidth}{|l|X|} 
% 		\hline
% 		Auslöser                                     &   
% 		Der Produktionsplaner möchte Engpässe an betroffenen Arbeitsplätzen durch Modifikationen an Kapazitätsangeboten beheben. \\ 
% 		\hline\hline
% 		Input                                         &   
% 		Handlungsbedarf durch die Auswertung der Kapazitätsauslastung \\ 
% 		\hline\hline
% 		Aktivitäten &   
% 		\begin{minipage}{5in}
%     		\begin{enumerate} 
%         		\renewcommand{\labelenumi}{(\arabic{enumi})}
%         		\item Starten der Anwendung
%         		\item Iterative Vorgehensweise bis zur Zielerreichung:
%             		\begin{enumerate} 
%             		\renewcommand{\labelenumi}{(\arabic{enumi})}
%             		\item Wahl einer Arbeitsplatzkapazität
%             		\item Erstellung einer Angebotskapazität
%             		\item Festlegung eines Gültigkeitszeitraums
%             		\item Pflege von Schichten in der Angebotskapazität
%             		\item Speichern der Änderungen
%             		\end{enumerate}
%             	\item Ende des Prozesses
%     		\end{enumerate}
%     		\vspace{1pt}		
% 		\end{minipage} \\
% 		\hline\hline
% 		Output                                        &   
% 		Modifizierte Kapazitätsangebote von Arbeitsplätzen  \\
% 		\hline
% 	\end{tabularx}
% 	\caption{\label{tab:aktivitäten}Ist-Prozessbeschreibung }
% \end{table}

% \begin{table}[H]
% 	\centering
% 	\begin{tabularx}{\textwidth}{l X} 
% 		\toprule
% 		\textbf{Kriterium}  &   
% 		\textbf{Beschreibung}  \\ 
% 		\midrule
% 		OP1 &   
% 		Keine Modifikationsmöglichkeiten  \\  \cmidrule(r){1-1} \cmidrule(r){2-2}
% 		OP2 &   
% 		Unausgereifter Prozessablauf \\ \cmidrule(r){1-1} \cmidrule(r){2-2}
% 		OP3 &   
% 		Mangelhafte Bedienbarkeit  \\ \cmidrule(r){1-1} \cmidrule(r){2-2}
% 		OP4 &   
% 		Kontraintuitives Stammdatenmodell  \\ \cmidrule(r){1-1} \cmidrule(r){2-2}
% 		OP5 &   
% 		Keine Überprüfung der Validität der Eingaben  \\ \cmidrule(r){1-1} \cmidrule(r){2-2}
% 		OP6 &   
% 		Kein Überblick über getätigte Modifikationen  \\
% 	    \bottomrule
% 	\end{tabularx}
% 	\caption{\label{tab:potentiale}Optimierungspotenziale des Ist-Systems}
% \end{table}



% % \resizebox{\textwidth}{!}{%
% % \begin{tikzpicture}

% % % Fragestellungs
% % \node (goal) [goal] {Entwicklung eines dynamischen Gesch{\"a}ftsprozesses zur ereignisgesteuerten Betriebsdatenerfassung mit SAP S/4HANA Cloud};

% % % Wozu?
% % \node (wozu) [question, node distance=7cm and 0.5cm, above left=of goal] {Wohin?};
% % \draw [] (goal) |- (wozu);


% % % Einleitung
% % \node (einleitung)[main,draw,rectangle, minimum height = 6cm, node distance=0.5cm and 0.5cm, below=of wozu] {};

% % \node (Einleitung) at (einleitung.north) [anchor=north] {\ref{ch:einleitung} \nameref{ch:einleitung}};
% % \node (Motivation) [submain,below=of Einleitung] {Motivation}
% % \node (Problemstellung) [submain, below=of Motivation] {Problemstellung};
% % \node (Zielsetzung) [submain, below=of Problemstellung] {Zielsetzung};
% % \node (Vorgehen) [submain, below=of Zielsetzung] {Vorgehen};

% % \draw [->] (Motivation) -- (Motivation |- Problemstellung.north); 
% % \draw [->] (Problemstellung) -- (Problemstellung |- Zielsetzung.north); 
% % \draw [->] (Zielsetzung) -- (Zielsetzung |- Vorgehen.north); 
% % % \draw [->] (gui1) -- (gui1 |- pssub1.north); 

% % % Was?
% % \node (was) [question, node distance=6cm and 0.5cm, above right=of goal] {Was?};
% % \draw [] (goal) |- (was);

% % % Grundlagen
% % \node (grundlagen)[main,draw,rectangle, minimum height = 4cm, node distance=0.5cm and 0.5cm, below=of was] {};

% % \node (Grundlagen) at (grundlagen.north) [anchor=north] {\ref{ch:Grundlagen} Stand der Wissenschaft und Technik};
% % \node (Automatisierung) [submain,below=of Grundlagen] {Automatisierung von Gesch{\"a}ftsprozessen};
% % \node (pssub2) [submain, below=of Automatisierung] {};

% % \end{tikzpicture}
% % }%


% Ehrenwörtliche Erklärung ewerkl.tex einziehen
\input{ewerkl.tex}


\end{document}

